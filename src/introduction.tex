\section*{Introduction}
\phantomsection
\addcontentsline{toc}{section}{Introduction}

When thinking of a restricted class of categories with pleasant properties, \emph{(finitary) varieties} are among the first candidates: \emph{Sets}, \emph{groups}, \emph{rings} all form complete and cocomplete categories, are well-powered, well-copowered, have free objects, closure properties, commutativity ... \\

A variety $\cat K$ is given through a finitary signature $\Sigma$ and a set of term equations over $\Sigma$. Then $\cat K$ is the full subcategory of the algebras $\catname{Alg}(\Sigma)$ that satisfy all equations. Universal algebra has the following famous characterisation of varieties by Birkhoff's HSP-Theorem: Let $\cat K$ be a full subcategory of $\catname{Alg}(\Sigma)$. Then the following are equivalent
\begin{enumerate}
\item $\cat K$ is a variety
\item $\cat K$ is closed under \underline{h}omomorphic images, \underline{s}ubobjects and \underline{p}roducts in $\catname{Alg}(\Sigma)$.
\end{enumerate}
This is a first correspondence of syntactical properties of a theory and properties of its model class. For example, we can deduce that the class of \emph{fields} cannot be described with a universal set of axioms, as the componentwise product of fields fails to be a field. \footnote{Of course, the category of fields cannot be equivalent to a variety either, as it's not complete. \emph{What characteristic would a terminal field have?}}  Still, Birkhoff's theorem is very much an extrinsic characterisation of varieties. Can we tell if a category is \emph{equivalent} to a variety, even when it doesn't look algebraic at all? In other words, what can we infer from categorical properties of $\cat K$ alone? \\

% generalize while maintaining pleasant properties
In this essay, I will introduce the theory of \emph{locally presentable} and \emph{accessible} categories. Both are clean, intrinsically defined categorical properties: Local presentability roughly means that a cocomplete category is generated in a certain sense by ``small objects'' via colimits. This notion already covers a wide range of everyday categories, including all varieties, quasivarieties, but also Banach spaces or posets. \\

Accessibility weakens the requirement of cocompleteness, giving us back wilder categories like the categories of fields, linear orders, sets with monomorphisms or connected graphs. \\

It turns out that again, both locally presentable and accessible categories can be axiomatised as models of certain kinds of theories, e.g. in infinitary propositional logic. Instead of introducing their syntax, we'll take another approach using ``language'' of \emph{sketches}: \\

A sketch is merely a small category $\cat A$ with distinguished cones and cocones. A \emph{model} of the sketch is a functor $F : \cat A \to \catname{Set}$ that turns the distinguished cones and cocones into limits and colimits. For example, consider a sketch $\cat A$ with two objects and three morphisms
\[
\xymatrix{
  & p \ar[rd]^{\pi_2} \ar[d]_{m} \ar[ld]_{\pi_1} & \\
a & a & a
}\]
with the prescribed discrete cone
\[
\xymatrix{
  & p \ar[rd]^{\pi_2} \ar[ld]_{\pi_1} & \\
a & & a
}\]
A model $F : \cat A \to \catname{Set}$ consists of two sets $X=F(a)$ and $P=F(p)$ where
\[
\xymatrix{
  & F(p) \ar[rd]^{F(\pi_2)} \ar[ld]_{F(\pi_1)} & \\
F(A) & & F(A)
}\]
is a limit, thus \[ P \cong X \times X \] 
and a map $F(m) : P \to X$. We can therefore identify the models of $\mathcal A$ as algebras with a binary operation $m$. \\

In fact, sketches eliminate the need to define terms and formulas for our theories entirely. We have a purely categorical concept right at our disposal. Still, limits and colimits provide us with all the expressive power we need. This will be our main theorem:

\begin{Theorem}[Sketchability]\ \\
\begin{enumerate}
\item A category is accessible if and only if it is \emph{sketchable}, i.e. equivalent to the category of models of a sketch. 

\item A category is locally presentable iff it is sketchable by a \emph{limit sketch}, i.e. a sketch that only prescribes \emph{cones}.

\item A category is a (many-sorted) variety iff it is sketchable by only prescribing finite discrete cones.
\end{enumerate}
\end{Theorem}

The main goal of this essay is to give a proof of the accessible $\Leftrightarrow$ sketchable-equivalence. In sections \ref{sec:presentableaccessible} and \ref{sec:sketches}, I will introduce our categorical notions and the formalism of sketches. I'll give a parallel treatment of locally presentable and accesible categories for as long as possible, but eventually focus on the accessible case. \\

In section \ref{seq:catstoskeches}, I will analyse the structure of accessible categories further and show how to turn them into functor categories and models of a sketch. Section \ref{sec:sketchesaccessible} will contain the other direction of the theorem, that model categories of sketches are indeed accessible.