\section{Models are accessible}
\label{sec:sketchesaccessible}

The goal of this section is the remaining direction in the sketchability theorem: To show that $\mod(\mathbb S)$ is accessible for every sketch $\mathbb S$. \\

As an outline, let's look at the following crude proof that the category of groups is $\aleph_1$-accessible. For every infinite regular cardinal $\kappa$, every group $G$ of cardinality less than $\kappa$ is $\kappa$-presentable by \ref{prop:varietypresentable}, because we have a presentation of $G$ by its own elements and multiplication table
\[ G \cong \langle e_g : g \in G \,|\, e_g e_h = e_{gh} \rangle. \]

Every group is the colimit (union) of its subgroups of cardinality less than $\kappa$, because the subgroups generated by each singleton element are at most countable and thus among them. So representatives of these groups up to isomorphism would be a candidate generating set in the definition of accessibility. Write $\sub_\kappa(G)$ for the poset of subgroups of cardinality less than $\kappa$. It remains to find out when this poset is actually $\kappa$-directed. \\

It is certainly not the case for $\kappa=\aleph_0$. Consider the infinite group $G=\langle a, b \,|\, a^2 = b^2 = 1 \rangle$ and the poset of its \emph{finite} subgroups. Then the subgroups $\langle a \rangle$ and $\langle b \rangle$ don't have an upper bound. 
However for $\kappa > \aleph_0$, if $\{H_i\}$ is a family of less than $\kappa$ subgroups of cardinality less than $\kappa$, the set \[ A = \bigcup_i H_i \]
is still smaller than $\kappa$ by regularity. The subgroup $\langle A \rangle$ generated by it has at most as many elements as there are words in $A \cup A^{-1}$, thus
\begin{equation} |\langle A \rangle| \leq 2 \cdot |A|\cdot \aleph_0 < \kappa. \label{eq:skolembound} \end{equation}
Therefore $\sub_\kappa(G)$ is $\kappa$-directed. $\blacksquare$ \\

We'll give an analogous argument for the models of a sketch $\mathbb S$. The hard part will obtaining the bound on the size of ``generated submodels'' like \eqref{eq:skolembound}. The rather explicit construction of the generated subgroup is a special case of the classical Skolem hull-construction in model theory, which builds up a model by inductively including all witnesses for existential formulae. This leads to a formulation of the (infinitary) downward Löwenheim-Skolem theorem for models of sketches. \\

First some terminology:

% Sort the S/A notation out; maybe fix a sketch S
\begin{Definition}
Let $M : \cat S \to \catname{Set}$ be a model of a sketch $\mathbb S$ and $S = \obj(\cat S)$. We will think of $M$ as an $S$-sorted set endowed with unary perations and write $s_M$ for the set $M(s)$ as well as $f_M : s_M \to s'_M$ for $M(f)$ if $f : s \to s'$ is a morphism in $\cat S$.
\begin{enumerate}
\item We define the cardinality of an $S$-sorted set $A$ to be \[ |A| = \sum_s |s_A|. \]
\item Subsets and submodels $B \subseteq A$ and other set-theoretic notions like unions will be considered sort-by-sort.
\item We call a (many-sorted) set $A$ $\lambda$-small if $|A|<\lambda$.
\end{enumerate}
\end{Definition}

\begin{Theorem}[Downward Löwenheim-Skolem for Sketches]\label{thm:ls}
Let $\mathbb S$ be a sketch. Then there is a regular cardinal $\kappa > \mu_{\mathbb S}$, such that for every model $M \in \mod(\mathbb S)$ the following holds: Every subset $A \subseteq M$ of cardinality less than $\kappa$ is contained in a submodel $\bar A \subseteq M$ of cardinality less than $\kappa$.
\end{Theorem}
We'll prove the theorem in the next subsection and describe the cardinal $\kappa$ explicitly. \\

\begin{Theorem}
$\mod(\mathbb S)$ is $\kappa$-accessible for every sketch $\mathbb S$.
\end{Theorem}
\begin{Proof}
We take $\kappa > \mu_\mathbb S$ to be the cardinal from \ref{thm:ls}.
\begin{enumerate}
\item Recall from \ref{prop:modsdirectedcolimits} that $\mod(\mathbb S)$ is closed under $\kappa$-directed colimits, computed sort-by-sort.
\item Every $\kappa$-small model is $\kappa$-presentable. We show that this is true in general for $\fun{\cat S}{\catname{Set}}$, so the same has to hold in the full subcategory $\mod(\mathbb S)$ that computes the same relevant colimits: \\

Let $F : \cat S \to \catname{Set}$ be $\kappa$-small. We know that $F$ is the canonical colimit of representable functors, which are finitely presentable objects by \ref{ex:representablepresentable}. When we can show the bound $|\el(F)| < \kappa$ for the canonical diagram, by \ref{prop:smallcolim}, $F$ will be $\kappa$-presentable.

The morphisms of the canonical diagram $\el(F)$ all have the form $(s,x) \xrightarrow{f} (t,y)$ where
\[ x \in F(s),\, y \in F(t),\, h : s \to t,\, x, y \in S. \]
The number of such morphisms is bounded by
\[ |\el(F)| \leq |F|^2 \cdot |S|^3 = |F|\cdot |S| < \kappa. \]

\item For every model $M$, write $\sub_\kappa(M)$ for the poset of its $\kappa$-small submodels ordered by inclusion. By the Löwenheim-Skolem theorem, this poset is $\kappa$-directed, because if $\{A_i\}$ is a $\kappa$-small family of such submodels, the set
\[ A = \bigcup_i A_i \]
has cardinality smaller than $\kappa$, so $A$ is contained in a submodel from $\sub_\kappa(M)$.

\item $M$ is the $\kappa$-directed colimit (union) of $\sub_\kappa(M)$, as this colimit is computed set-theoretically and again by Löwenheim-Skolem, every singleton from the underlying set of $M$ is contained in a submodel from $\sub_\kappa(M)$. 

\item The class of all $\mathbb S$-models of cardinality less than $\kappa$ is essentially small. Their representatives form a generating set of $\mod(\mathbb S)$ in the definition of $\kappa$-accessibility.
\end{enumerate}
\end{Proof}

\begin{Corollary}
Locally presentable categories are precisely the model categories of limit sketches.
\end{Corollary}
\begin{Proof}
It remains to show that $\mod(\mathbb S)$ is locally presentable for limit sketches $\mathbb S$. But $\mod(\mathbb S)$ is closed under limits in $\fun{\cat A}{\catname{Set}}$ as limits are computed pointwise and limits commute with limits. Therefore it is complete and accessible, thus locally presentable.
\end{Proof}

\subsection{The Löwenheim-Skolem theorem}

I will motivate the cardinal bounds that occur in the infinitary version of Löwenheim-Skolem and then adapt the theorem to models of sketches. \\

Consider a $\lambda$-ary algebra $M$ with $\mu$ operations and a subset $A_0 \subseteq M$. We close $A_0$ under the operations of the algebra by the following transfinite process:

\[
\begin{cases}
A_{i+1} = A_i^+, \\
A_{\alpha} = \cup_{i < \alpha}\,A_i
\end{cases}
\]

where

\[ A^+ = \{ f(\vec a) : \vec a \in A^\nu \, | \, f \text{ operation of arity } \nu \}. \] 

For convenience, we assume that $M$ has an identity operation among it, so $A \subseteq A^+$. The process stabilises at $A_\lambda$, for if $\vec a \in (A_\lambda)^\nu, \nu < \lambda$, each component $a_i$ lies in some $A_{\alpha_i}$. By regularity of $\lambda$, the supremum $\alpha$ of the $\alpha_i$ is $< \lambda$, thus already $\vec a \in (A_\alpha)^\nu$ and $f(\vec a) \in A_{\alpha+1} \subseteq A_\lambda$ by construction. $A_\lambda$ is the smallest subalgebra containing $A$. \\

Because all operations of the algebra have arity $< \lambda$, the set of possible inputs $\vec a$ ranges over the set
\[ A^{< \lambda} := \bigsqcup_{\nu < \lambda} A^\nu. \]

We use the following notation for the cardinality of that set
\begin{Definition}
For a cardinal $\mu$, let
\[ \mu^{< \lambda} := \sum_{\nu < \lambda} \mu^\nu. \]
\end{Definition}

The cardinality of $A^+$ is thus bounded by
\[ |A^+| \leq |\{ (f,\vec a) : \vec a \in A^{< \lambda} \}| = \mu \cdot |A|^{< \lambda}. \]

Just as a finite sequence of finite sequences is essentially still a finite sequence, repeated applications of the isolated step can't add more to the cardinality, as the following lemma shows:

\begin{Lemma}
For $\mu \geq \lambda$ and $\lambda$ regular, 
\[ \left(\mu^{< \lambda}\right)^{< \lambda} = \mu^{< \lambda}. \]
\end{Lemma}

Let $\beta \geq \mu + \lambda$ and say $|A_0| \leq \beta^{< \lambda}$, then we'll show inductively that $|A_i| \leq \beta^{< \lambda}$ for all $i \leq \lambda$.
\begin{description}
\item[Successor step] If $|A_i| \leq \beta^{< \lambda}$, then
\[ |A_i^+| = \mu \cdot |A_i|^{< \lambda} \leq \left(\beta^{< \lambda}\right)^{< \lambda} = \beta^{< \lambda}. \]
\item[Limit step] If $|A_i| \leq \beta^{< \lambda}$ for each $i < \alpha \leq \lambda$ then
\[ |A_\alpha| \leq \sum_{i < \alpha} |A_i| \leq \beta^{< \lambda} \]
by regularity.
\end{description}
Therefore, the cardinal $\kappa = \left[(\mu + \lambda)^{<\lambda}\right]^+$ gives the bound \eqref{eq:skolembound} for the type of algebra we considered. \\

We can use the same closure process for models of a sum-sketch $\mathbb S$. This is because by \ref{prop:sumconnectedlimits}, intersections of submodels are still submodels, therefore the description of the ``generated submodel'' stays the same as for varieties.
\begin{Proposition}
Let $\mathbb S = (\cat S, \dist L, \dist C)$ be a sum sketch. Take the regular cardinals $\lambda = \lambda^+_\mathbb S, \mu = (\lambda + \mu_\mathbb S)^+$ and $\kappa = (\mu^{< \lambda})^+$. Then every subset $A$ of cardinality $< \kappa$ of a model of $\mathbb S$ is contained in a submodel of cardinality $< \kappa$.
\end{Proposition}
\begin{Proof}
Let $M$ be a model of $\mathbb S$ and $A \subseteq M$ a subset. A is a submodel of $M$ if and only if the following three closure conditions hold
\begin{enumerate}
\item\label{item:f} for every $f : s \to t$ and $a \in s_A$, $f_M(a) \in t_A$
\item\label{item:cones} for every distinguished cone $(s \xrightarrow{\pi_i} s_i)$, we have $a \in s_A$ iff $(\pi_i)_M(a) \in (s_i)_A$ for all $i$.
\item\label{item:cocones} for every distinguished cocone $(s_i \xrightarrow{c_i} s)$, we have $a \in (s_i)_A$ iff $(c_i)_M(a) \in s_A$ for some $i$.
\end{enumerate}
This leads to the following closure process: We define $A^+$ as the union of the following elements
\begin{enumerate}
\item $f_M(a)$ where $f : s \to t$, $a \in A$
\item $a \in s_M$ such that $(\pi_i)_M(a) = a_i$ where $(a_i \in (s_i)_A)$ is a compatible collection with respect to a distinguished cone $(s \xrightarrow{\pi_i} s_i)$
\item $a \in (s_i)_M$ where $b \in s_A, (c_i)_M(a) = b$ for a distinguished cocone  $(s_i \xrightarrow{c_i} s)$
\end{enumerate}
This process stabilises for $A_\lambda$ with the smallest submodel containing $A$. We show again by induction that $|A_i| < \kappa$ for all $i \leq \lambda$: 
\begin{description}
\item[Successor step] Let $|A_i|$ be $< \kappa$. The contributions of \ref{item:f} and \ref{item:cocones} are easily bounded $< \kappa$. The contribution of \ref{item:cones} is at most $|\dist L|\cdot |A_i|^{< \lambda}$. But $|A_i| \leq \mu^{< \lambda}$ by assumption, so \[ |A_i|^{< \lambda} \leq \left(\mu^{< \lambda}\right)^{< \lambda} = \mu^{< \lambda} < \kappa. \]
\item[Limit step] follows by regularity of $\kappa$.
\end{description}
\end{Proof}