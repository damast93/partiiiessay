\section{Models are accessible}
\label{sec:sketchesaccessible}

The goal of this section is the remaining direction in the sketchability theorem: To show that $\mod(\mathbb S)$ is accessible for every sketch $\mathbb S$. \\

As an outline, let's look at the following crude proof that the category of groups is $\aleph_1$-accessible. For every infinite regular cardinal $\kappa$, every group $G$ of cardinality less than $\kappa$ is $\kappa$-presentable by \ref{prop:varietypresentable}, because we have a presentation of $G$ by its own elements and multiplication table
\[ G \cong \langle e_g : g \in G \,|\, e_g e_h = e_{gh} \rangle. \]
Every group is the colimit (union) of its subgroups of cardinality less than $\kappa$, so representatives of these groups up to isomorphism would be a candidate generating set in the definition of accessibility. Write $\sub_\kappa(G)$ for the poset of subgroups of cardinality less than $\kappa$. It remains to find out when this poset is actually $\kappa$-directed. \\

It is certainly not the case for $\kappa=\aleph_0$. Consider the infinite group $G=\langle a, b \,|\, a^2 = b^2 = 1 \rangle$ and the poset of its \emph{finite subgroups}. Then the subgroups $\langle a \rangle$ and $\langle b \rangle$ don't have an upper bound. 
However for $\kappa > \aleph_0$, if $\{H_i\}$ is a family of less than $\kappa$ subgroups of cardinality less than $\kappa$, the set \[ A = \bigcup_i H_i \]
is still smaller than $\kappa$ by regularity. The subgroup $\langle A \rangle$ generated by it has at most as many elements as there are words in $A \cup A^{-1}$, thus
\begin{equation} |\langle A \rangle| \leq |A|\cdot \aleph_0 < \kappa. \label{eq:skolembound} \end{equation}
Therefore $\sub_\kappa(G)$ is $\kappa$-directed. $\blacksquare$ \\

We'll give an analogous argument for the models of a sketch $\mathbb S$. The hard part will obtaining the bound on the size of ``generated submodels'' like \eqref{eq:skolembound}. The rather explicit construction of the generated subgroup is a special case of the classical Skolem hull-construction in model theory, which builds up a model by inductively including all witnesses for existential formulae. This leads to a formulation of the (infinitary) downward Löwenheim-Skolem theorem for models of sketches. \\

First some terminology:

% Sort the S/A notation out; maybe fix a sketch S
\begin{Definition}
Let $M : \cat S \to \catname{Set}$ be a model of a sketch $\mathbb S$ and $S = \obj(\cat S)$. We will think of $M$ as an $S$-sorted set endowed with unary perations and write $s_M$ for the set $M(s)$ as well as $f_M : s_M \to s'_M$ for $M(f)$ if $f : s \to s'$ is a morphism in $\cat S$.
\begin{enumerate}
\item We define the cardinality of an $S$-sorted set $A$ to be \[ |A| = \sum_s |s_A|. \]
\item Subsets and submodels $B \subseteq A$ and other set-theoretic notions like unions will be considered sort-by-sort.
\item We call a (many-sorted) set $A$ $\lambda$-small if $|A|<\lambda$.
\end{enumerate}
\end{Definition}

\begin{Theorem}[Downward Löwenheim-Skolem for Sketches]\label{thm:ls}
Let $\mathbb S$ be a sketch. Then there is a regular cardinal $\kappa > \mu_{\mathbb S}$, such that for every model $M \in \mod(\mathbb S)$ the following holds: Every subset $A \subseteq M$ of cardinality less than $\kappa$ is contained in a submodel $\bar A \subseteq M$ of cardinality less than $\kappa$.
\end{Theorem}
We'll prove the theorem in the next subsection and describe the cardinal $\kappa$ explicitly. \\

\begin{Theorem}
$\mod(\mathbb S)$ is accessible for every sketch $\mathbb S$
\end{Theorem}
\begin{Proof}
Let $\kappa > \mu_\mathbb S$ be the cardinal from \ref{thm:ls}.
\begin{enumerate}
\item Recall from \ref{prop:modsdirectedcolimits} that $\mod(\mathbb S)$ is closed under $\kappa$-directed colimits, computed sort-by-sort.
\item Every $\kappa$-small model is $\kappa$-presentable. We show that this is true in general for $\fun{\cat S}{\catname{Set}}$, so the same has to hold in the full subcategory $\mod(\mathbb S)$ that computes the same relevant colimits: \\

Let $F : \cat S \to \catname{Set}$ be $\kappa$-small. We know that $F$ is the canonical colimit of representable functors, which are finitely presentable objects by \ref{ex:representablepresentable}. When we can show the bound $|\el(F)| < \kappa$ for the canonical diagram, by \ref{prop:smallcolim}, $F$ will be $\kappa$-presentable.

The morphisms of the canonical diagram $\el(F)$ all have the form $(s,x) \xrightarrow{f} (t,y)$ where
\[ x \in F(s),\, y \in F(t),\, h : s \to t,\, x, y \in S. \]
The number of such morphisms is bounded by
\[ |\el(F)| \leq |F|^2 \cdot |S|^3 = |F|\cdot |S| < \kappa. \]

\item For every model $M$, write $\sub_\kappa(M)$ for the poset of its $\kappa$-small submodels. By the Löwenheim-Skolem theorem, this poset is $\kappa$-directed, because if $\{A_i\}$ is a $\kappa$-small family of such submodels, the set
\[ A = \bigcup_i A_i \]
has cardinality smaller than $\kappa$, so $A$ is contained in a submodel from $\sub_\kappa(M)$. Now $M$ is the (set-theoretic) $\kappa$-directed colimit of $\sub_\kappa(M)$, and representatives of the essentially small collection of all models of cardinality less than $\kappa$ form a generating set of $\mod(\mathbb S)$ in the definition of accessibility. % Is this incomplete? 2nd LS in Pare
\end{enumerate}
\end{Proof}

\begin{Corollary}
Locally presentable categories are precisely the model categories of limit sketches.
\end{Corollary}
\begin{Proof}
It remains to show that $\mod(\mathbb S)$ is locally presentable for limit sketches $\mathbb S$. But $\mod(\mathbb S)$ is closed under limits in $\fun{\cat A}{\catname{Set}}$ as limits are computed pointwise and limits commute with limits. Therefore it is complete and accessible, thus locally presentable.
\end{Proof}

\subsection{The Löwenheim-Skolem theorem}

I will motivate the cardinal bounds that occur in the infinitary version of Löwenheim-Skolem and then adapt the theorem to models of sketches. \\

Consider a $\lambda$-ary algebra with $\mu$ operations (wlog. include an identity operation). Take an algebra $M$ and a subset $A_0 \subseteq M$. We close $A_0$ under the operations of the algebra by the following transfinite process:

\[
\begin{cases}
A_{i+1} = A_i^+, \\
A_{\alpha} = \cup_{i < \alpha} A_i
\end{cases}
\]

where

\[ A^+ = \{ f(\vec a) : \vec a \in A^\nu \text{ where } f \text{ operation of arity } \nu \}. \] 

The set $A_\lambda$ is a closed under all operations and thus a subalgebra of $M$. 

\begin{Definition}
For a set $A$ and a cardinal $\lambda$, let
\[ A^{< \lambda} = \bigsqcup_{\nu < \lambda} A^\nu. \]
Likewise, for a cardinal $\mu$, let
\[ \mu^{< \lambda} = \sum_{\nu < \lambda} \mu^\nu. \]
\end{Definition}

As all operations of the algebra have arity $< \lambda$, the cardinality of the set $A^+$ is bounded by
\[ |A^+| \leq |\{ (f,\vec a) : \vec a \in A^{< \lambda} \}| = \mu \cdot |A|^{< \lambda}. \]
Repeated applications of the step operation can't add more to the cardinality, as the following lemma shows:

\begin{Lemma}
For $\mu \geq \lambda$ and $\lambda$ regular, 
\[ \left(\mu^{< \lambda}\right)^{< \lambda} = \mu^{< \lambda}. \]
\end{Lemma}

Let $\beta \geq \mu + \lambda$ and say $|A_0| \leq \beta^{< \lambda}$, then we'll show inductively that $|A_i| \leq \beta^{< \lambda}$ for all $i \leq \lambda$.
\begin{enumerate}
\item If $|A_i| \leq \beta^{< \lambda}$, then
\[ |A_i^+| = \mu \cdot |A_i|^{< \lambda} \leq \left(\beta^{< \lambda}\right)^{< \lambda} = \beta^{< \lambda}. \]
\item If $|A_i| \leq \beta^{< \lambda}$ for each $i < \alpha < \lambda$ then
\[ |A_\alpha| \leq \sum_{i < \alpha} |A_i| \leq \beta^{< \lambda} \]
by regularity.
\end{enumerate}
Therefore, the cardinal $\kappa = \left[(\mu + \lambda)^{<\lambda}\right]^+$ gives the bound \eqref{eq:skolembound} for the type of algebra we considered.