\section{Models are accessible}
\label{sec:sketchesaccessible}

The goal of this section is the remaining direction in the sketchability theorem: To show that $\mod(\mathbb S)$ is accessible for every sketch $\mathbb S$. \\

As an outline, let's look at the following crude proof that the category of groups is $\aleph_1$-accessible. For every infinite regular cardinal $\lambda$, every group $G$ of cardinality less than $\lambda$ is $\lambda$-presentable by \ref{prop:varietypresentable}, because we have a presentation of $G$ by its own elements and multiplication table
\[ G \cong \langle e_g : g \in G \,|\, e_g e_h = e_{gh} \rangle. \]
Every group is the colimit (union) of its subgroups of cardinality less than $\lambda$, so representatives of these groups up to isomorphism would be a candidate generating set in the definition of accessibility. Write $\sub_\lambda(G)$ for the poset of subgroups of cardinality less than $\lambda$. It remains to find out when this poset is actually $\lambda$-directed. \\

It is certainly not the case for $\lambda=\aleph_0$. Consider the infinite group $G=\langle a, b \,|\, a^2 = b^2 = 1 \rangle$ and the poset of its finite subgroups. Then the subgroups $\langle a \rangle$ and $\langle b \rangle$ don't have an upper bound.

However for $\lambda > \aleph_0$, if $\{H_i\}$ is a family of less than $\lambda$ subgroups of cardinality less than $\lambda$, the set \[ A = \bigcup_i H_i \]
is still smaller than $\lambda$ by regularity. The subgroup $\langle A \rangle$ generated by it has at most as many elements as there are words in $A \cup A^{-1}$, thus
\begin{equation} |\langle A \rangle| \leq |A|\cdot \aleph_0 < \lambda. \label{eq:skolembound} \end{equation}
Therefore $\sub_\lambda(G)$ is $\lambda$-directed. $\blacksquare$ \\

We'll give an analogous argument for models of a sketch $\mathbb S$. The hard part will obtaining the bound on the size of ``generated submodels'' like \eqref{eq:skolembound}. Note that the rather explicit construction of the generated subgroup is a special case of the classical Skolem hull-construction in model theory, which builds up a model by inductively including all witnesses for existential formulae 
\begin{Theorem}[Downward Löwenheim-Skolem theorem]
Given a set $\Phi$ of formulae in a first-order language $L$, a model $M \models \Phi$ and a subset $A \subseteq M$, the Skolem hull
\[ \bar A := \mathcal H^N(A) \]
is a submodel of $N$, containing $A$. Furthermore $|\bar A| \leq |A| + |\Phi| + \aleph_0$, independently of $|N|$. % Bound formulation?
\end{Theorem}

We need a formulation of this theorem for models of sketches. Note that this will have to be analogous to an infinitary version of Löwenheim-Skolem. Some terminology:
\begin{Definition}
Let $M : \cat S \to \catname{Set}$ be a model of a sketch $\mathbb S$ and $S = \obj(\cat S)$. We will think of $M$ as an $S$-sorted set endowed unary with operations and write $s_M$ for the set $M(s)$ as well as $f_M : s_M \to s'_M$ for $M(f)$ if $f : s \to s'$ is a morphism in $\cat S$.
\begin{enumerate}
\item We define the cardinality of an $S$-sorted set $A$ to be \[ |A| = \sum_s |s_A|. \]
\item Subsets and submodels $B \subseteq A$ and other set-theoretic notions like unions will be considered sort-by-sort.
\item We call a (many-sorted) set $A$ $\lambda$-small if $|A|<\lambda$.
\end{enumerate}
\end{Definition}

\begin{Theorem}[Downward Löwenheim-Skolem for Sketches]
Let $\mathbb S$ be a sketch. Then there is a regular cardinal $\kappa$, depending on $\mathbb S$, such that for every model $M \in \mod(\mathbb S)$ the following holds: Every subset $A \subseteq M$ of cardinality less than $\kappa$ is contained in a submodel $\bar A \subseteq M$ of cardinality less than $\kappa$.
\end{Theorem}
We'll prove the theorem in [...] and describe the cardinal $\kappa$ explicitly. \\

\begin{Theorem}
$\mod(\mathbb S)$ is accessible for every sketch $\mathbb S$
\end{Theorem}
\begin{Proof}
Let $\kappa$ be the cardinal from LS.
\begin{enumerate}
\item Recall from \ref{prop:modsdirectedcolimits} that $\mod(\mathbb S)$ is closed under $\kappa$-directed colimits, computed sort-by-sort.
\item Every $\kappa$-small model is $\kappa$-presentable. We show that this is true in general for $\fun{\cat S}{\catname{Set}}$, so the same has to hold in the full subcategory $\mod(\mathbb S)$ that computes the same relevant colimits: \\

Let $F : \cat S \to \catname{Set}$ be $\kappa$-small. We know that $F$ is the canonical colimit of representable functors, which are finitely presentable objects by \ref{ex:representablepresentable}. The morphisms of the canonical diagram $y \downarrow F$ are given by \[ (s,x) \to (s',x'), x \in F(s), x' \in F(x'), h : s \to s' \]
thus we can bound its size as
\[ |y \downarrow F| \leq |F|^2|S|^3 = |F||S| < \kappa. \]
By \ref{prop:smallcolim}, $F$ is $\kappa$-presentable.

\item For every model $M$, write $\sub_\kappa(M)$ for the poset of its $\kappa$-small submodels. By the Löwenheim-Skolem theorem, this poset is $\kappa$-directed, because if $\{A_i\}$ is a $\kappa$-small family of such submodels, the set
\[ A = \bigcup_i A_i \]
has cardinality smaller than $\kappa$, so $A$ is contained in a submodel from $\sub_\kappa(M)$. Now $M$ is the (set-theoretic) $\kappa$-directed colimit of $\sub_\kappa(M)$, and the essentially small collection of all models of cardinality less than $\kappa$ form a generating set of $\mod(\mathbb S)$ in the definition of accessibility.
\end{enumerate}
\end{Proof}



\begin{Corollary}
Locally presentable categories are precisely the model categories of limit sketches.
\end{Corollary}
\begin{Proof}
It remains to show that $\mod(\mathbb S)$ is locally presentable for limit sketches $\mathbb S$. But $\mod(\mathbb S)$ is closed under limits in $\fun{\cat A}{\catname{Set}}$ as limits are computed pointwise and limits commute with limits. Therefore it is complete and accessible, thus locally presentable.
\end{Proof}