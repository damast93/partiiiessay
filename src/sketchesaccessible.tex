\section{Models are accessible}
\label{sec:sketchesaccessible}

% Some properties of Mod(S) already in the section about sketeches

The goal of this section is the remaining direction in the sketchability theorem: To show that $\mod(\mathbb S)$ is accessible for every sketch $\mathbb S$. \\

As an outline, let's look at the following crude proof that the category of groups is $\aleph_1$-accessible. For every infinite regular cardinal $\lambda$, every group $G$ of cardinality less than $\lambda$ is $\lambda$-presentable by \ref{prop:varietypresentable}, because we have a presentation of $G$ by its own elements and multiplication table
\[ G \cong \langle e_g : g \in G \,|\, e_g e_h = e_{gh} \rangle. \]
Every group is the colimit (union) of its subgroups of cardinality less than $\lambda$, so representatives of these groups up to isomorphism would be a candidate generating set in the definition of accessibility. Write $\sub_\lambda(G)$ for the poset of subgroups of cardinality less than $\lambda$. It remains to find out when this poset is actually $\lambda$-directed. \\

It is certainly not the case for $\lambda=\aleph_0$. Consider the infinite group $G=\langle a, b \,|\, a^2 = b^2 = 1 \rangle$ and the poset of its finite subgroups. Then the subgroups $\langle a \rangle$ and $\langle b \rangle$ don't have an upper bound.

However for $\lambda > \aleph_0$, if $\{H_i\}$ is a family of less than $\lambda$ subgroups of cardinality less than $\lambda$, the set \[ A = \bigcup_i H_i \]
is still smaller than $\lambda$ by regularity. The subgroup $\langle A \rangle$ generated by it has at most as many elements as there are words in $A \cup A^{-1}$, thus
\begin{equation} |\langle A \rangle| \leq |A|\cdot \aleph_0 < \lambda. \label{eq:skolembound} \end{equation}
Therefore $\sub_\lambda(G)$ is $\lambda$-directed. $\blacksquare$ \\

We'll give an analogous argument for models of a sketch $\mathbb S$. The hard part will obtaining the bound on the size of ``generated submodels'' like \eqref{eq:skolembound}. Note that the rather explicit construction of the generated subgroup is a special case of the classical Skolem hull-construction in model theory
\begin{Theorem}[Downward Löwenheim-Skolem theorem]
Given a first-order language $L$, a model $N \models \Phi$ and a subset $A \subseteq N$, the Skolem hull
\[ M := \mathcal H^N(A) \]
is a submodel of $N$ of $\Phi$, containing $A$. Furthermore $|M| \leq |A| + |\Phi| + \aleph_0$. 
\end{Theorem}

\begin{Corollary}
Locally presentable categories are precisely the model categories of limit sketches.
\end{Corollary}
\begin{Proof}
It remains to show that $\mod(\mathbb S)$ is locally presentable for limit sketches $\mathbb S$. But $\mod(\mathbb S)$ is closed under limits in $\fun{\cat A}{\catname{Set}}$ as limits are computed pointwise and limits commute with limits. Therefore it is complete and accessible, thus locally presentable.
\end{Proof}