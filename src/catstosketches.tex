\section{From categories to sketches}
\label{seq:catstoskeches}

In this section, we will see how to embed accessible categories into functor categories and turn them into models of a sketch: \\

\subsection{Canonical colimits}

Let $\cat K$ be a $\lambda$-accessible category. We know that every object is $\lambda$-filtered colimit of some set $\cat A$ of $\lambda$-presentable objects. In fact, we can pick a canonical such set generating set by considering all $\lambda$-presentable objects.

\begin{Proposition}
For every $\lambda$-accessible category $\cat K$, the full subcategory of $\lambda$-presentable objects is essentially small. 
\end{Proposition}
\begin{Proof}
Take any $\lambda$-presentable object $B$ and write it as a $\lambda$-filtered colimit $B \xrightarrow{\sim} \colim_i A_i$ of objects in $\cat A$. The isomorphism factors through one of the $A_i$, exhibiting $B$ as a retract $B \xrightarrow{j} A_i \xrightarrow{r} B$ of $A_i$. Note that $jr$ is an idempotent endomorphism of $A_i$, and retracts that induce the same endomorphism are isomorphic. Thus we have an class injection
\[ \{ \text{retracts of } A_i \}/\text{isomorphism} \hookrightarrow \hom(A_i,A_i) \]
where the codomain is a set as $\cat K$ is locally small. Now take the union of the sets of representatives for each $A_i \in \cat A$.
\end{Proof}

Let's write $\cat K_\lambda$ for any set of representatives of the $\lambda$-presentable objects up to isomorphism. Given an object $K$ and small full subcategory $\cat A$ of $\cat K$, the canonical diagram with respect to $\cat A$ is the diagram of all morphisms of $\cat A$-objects into $\cat K$, formally given by the forgetful functor
\[ U : \cat A/K \to \cat K. \]
We say that $K$ is the canonical colimit of $\cat A$-objects if the canonical cocone
\[ (A \xrightarrow{f} K) \xrightarrow{f} K \]
is colimiting. We call $\cat A$ \emph{dense} in $\cat K$ if every object of $\cat K$ is the canonical colimit of $\cat A$-objects.

\begin{Proposition}\label{prop:presdense}
For every $\lambda$-accessible category $\cat K$ and $\cat A = K_\lambda$, every object is canonical colimit of $\cat A$-objects and the canonical diagram is $\lambda$-filtered.
\end{Proposition}
\begin{Proof}
Let $K$ be an object of $\cat K$. We know that it is $\lambda$-filtered colimit of some $\lambda$-presentable objects $(B_i)$. Looking at the canonical diagram $A : \cat A/K \to \cat K$, we see that the diagram $B$ sits inside of it like this
\[
\xymatrix{
  & & K & & \\
& B_i \ar[rr] \ar[ru] & & B_j \ar[lu] & \\
A_r \ar[rrrr] \ar@/^2pc/[rruu] \ar@{.>}[ru]_{\exists} & & & & A_s \ar@{.>}[lu]^{\exists} \ar@/_2pc/[lluu]
}\]
Because $A_r$ are all locally $\lambda$-presentable, we get the dashed morphisms into some $B_i$. The larger diagram therefore factors through $B$ in a compatible way, in the terminology of [...AR], $B$ is cofinal in $A$, and the diagrams have the same colimit. \\

For $\lambda$-filteredness, note that each $\lambda$-small subcategory of $\cat A/K$ admits factorisations into less than $\lambda$ $B$-objects, but by $\lambda$-directedness of $B$, everything in fact factors through a single $B$-object, so the subcategory has a cocone over it. 
\end{Proof}

% Use actual confinal diag B in A/K, index stuff with U_f

Given a small full subcategory $\cat A$ of $\cat K$, we have a functor 
\[ E : \cat K \to \fun{\cat A^\op}{\catname{Set}} \]
that sends an object $K$ to the domain restriction of $\hom(-, K)$ to $\cat A^\op$. We can restate the properties of $\cat A$ in terms of $E$ in the following way:

\begin{Proposition}\
\label{prop:canonicalproperties}
\begin{enumerate}
\item $E$ is fully faithful if $\cat A$ is dense in $\cat K$. \label{item:fullyfaithful}
\item $E$ preserves $\lambda$-filtered colimits if all objects of $\cat A$ are $\lambda$-presentable. \label{item:limits}
\end{enumerate}
\end{Proposition}
% Make stuff iff again

\begin{Proof}
\ref{item:fullyfaithful}: Let $U : \cat A/K \to \cat K$ be the canonical diagram; we claim that a cocone $(U_f \to K')$ corresponds to a natural transformation $EK \to EK'$. Indeed such a cocone has for each $A \xrightarrow{f} K$ a morphism $A \xrightarrow{\hat f} K'$, such that for all commutative triangles
\[
\xymatrix{
A \ar[rr]^{h} \ar[rd]_{f} & & A' \ar[ld]^{g} \\
& K
}\]
we have $\hat f = \hat g \cdot h$. This amounts precisely to a family of maps $(\widehat{-})_A: \hom(A,K) \to \hom(A,K')$, natural in $A$. We get the desired equation
\[ \hom(EK,EK') \cong \textrm{Cocones}(U, K') \cong \hom(K,K') \] 
if and only if the cocone into $K$ is universal, thus $\cat A$ is dense. \\

For \ref{item:limits}, note that colimits of functors are computed pointwise. For all $\lambda$-filtered colimits, we have
\[ E(\colim_i K_i)(A) \cong \hom(A,\colim_i K_i) \cong \colim_i \hom(A,K_i) \]
as $A$ is $\lambda$-presentable, therefore $E(\colim_i K_i) \cong \colim_i EK_i$.
\end{Proof}

Together with Proposition \ref{prop:presdense}, we see that for $\lambda$-accessible categories $\cat K$, we get a canonical embedding
\[ E : \cat K \to \fun{\cat A}{\catname{Set}} \text{ where } \cat A = \cat K^\op_\lambda. \]

\subsection{Flat functors}
We want to characterise precisely which subcategories of $\fun{\cat A}{\catname{Set}}$ occur via embeddings of accessible categories.

\begin{Definition}\
We call a functor $F : \cat A \to \catname{Set}$
\begin{enumerate}
\item \emph{$\lambda$-flat} if it is $\lambda$-filtered colimit of representable functors
\item \emph{$\lambda$-continuous} if it preserves $\lambda$-small limits
\end{enumerate}
and we denote the full subcategories of $\fun {\cat A}{\catname{Set}}$ of $\lambda$-flat and $\lambda$-continuous functors as 
\[ \flat_\lambda(\cat A), \cont_\lambda(\cat A) \text{ respectively}. \]
\end{Definition}

\begin{Proposition}
Let $\cat K$ be $\lambda$-accessible and $\cat A = \cat K^\op_\lambda$. Then the essential image of the canonical embedding consists precisely of the $\lambda$-flat functors, and we have an equivalence
\[ \cat K \simeq \flat_\lambda(\cat A). \]
Conversely, for $\cat A$ small, the category $\flat_\lambda(\cat A)$ is $\lambda$-accessible.
\end{Proposition}
\begin{Proof}
The first statement follows from \ref{prop:canonicalproperties}; $E$ preserves $\lambda$-filtered colimits, and every object $K$ can be written as one, so
\[ EK \cong E(\colim_i A_i) \cong \colim_i \hom(-,A_i), \]
where the latter is a colimit of representables in $\fun{\cat A^\op}{\catname{Set}}$.
% comment on both directions
Conversely, by [MP], $\flat_\lambda(\cat A)$ is closed under $\lambda$-filtered colimits in $\fun{\cat A^\op}{\catname{Set}}$. Furthermore representable functors are $\lambda$-presentable objects and by definition every $\lambda$-flat functor is $\lambda$-filtered colimits of these. So $\flat_\lambda(\cat A)$ is $\lambda$-accessible. % proof
\end{Proof}

% orthogonality small vs. filtered
Note that this description leads to a strong characterisation in the locally presentable case.
\begin{Proposition}
Let $\cat K$ be $\lambda$-accessible \emph{and cocomplete}, i.e. locally $\lambda$-presentable, and $\cat A = \cat K^\op_\lambda$. Then the essential image of the canonical embedding restricts to the $\lambda$-continuous functors and
\[ \cat K \simeq \cont_\lambda(\cat A). \]
\end{Proposition}
\begin{Proof}
The category $\cat A$ is closed under $\lambda$-small colimits in $\cat K$, because the colimit exists in $\cat K$ but by \ref{prop:smallcolim} actually lies in $\cat A$. \\

By continuity of representables, the functor $\hom(-, K)$ preserves all limits in $\cat K^\op$. However the $\lambda$-small limits of $\cat A^\op$ agree with those from $\cat K^\op$, so they preserved as well.

% Show other implication
Now let $F$ be $\lambda$-continuous.
\end{Proof}

\begin{Corollary}\
Let $\cat K$ be a locally $\lambda$-presentable category. Then
\begin{enumerate}
\item $\cat K$ is complete, and well-powered \label{item:cw}
\item $\cat K$ is sketchable by a $\lambda$-small limit sketch \label{item:limsketch}
\end{enumerate}
\end{Corollary}
\begin{Proof}
We have an equivalence $K \simeq \cont_\lambda(\cat A)$. The latter category is closed under limits in $\fun{\cat A}{\catname{Set}}$, as limits commute with limits; thus it is complete. Every category with pullbacks and a strong generator is well-powered, showing \ref{item:cw}. For \ref{item:limsketch}, note that by definition
\[ \cont_\lambda(\cat A) \simeq \mod(\mathbb S) \]
where the sketch $\mathbb S$ collects all $\lambda$-small limit cones in $\cat A$.
\end{Proof}