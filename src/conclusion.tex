\section{Conclusion \& Outlook}
\label{sec:conclusion}

I have given two characterisations of a broad and important class of categories -- one syntactical and one semantical -- and proven that they agree. I will now give a brief outlook on interesting aspects of the theory. All of those statements can be found across \cite{AdamekRosicky}. \\

From the various characterisations, interesting examples can be deduced: For example the category $\catname{Ban}$ of Banach spaces is locally $\aleph_1$-presentable while the category $\catname{Hilb}$ of Hilbert spaces is $\aleph_1$-accessible. Furthermore, for every infinitary sentence $\Phi$, the category $\catname{Elem}(\Phi)$ of models of $\Phi$ with \emph{elementary embeddings} as morphisms is accessible, essentially again by the Löwenheim-Skolem theorem. \\

Locally presentable and accessible categories retain many pleasant properties of $\catname{Set}$ and are closed under important categorical constructions. For example
\begin{enumerate}
\item Locally presentable categories are co-well-powered
\item In every locally $\lambda$-presentable category, $\lambda$-small limits commute with $\lambda$-filtered colimits.
\item Slice categories $\cat K/K$, $K/\cat K$ and functor categories $\fun{\cat A}{\cat K}$ are locally presentable for $\cat K$ locally presentable and $\cat A$ small. $\fun{\cat A}{\cat K}$ is accessible for $\cat K$ accessible.
\end{enumerate}
There is a natural notion of \emph{accessible functors} between $\lambda$-accessible categories, namely functors preserving $\lambda$-filtered colimits. Importantly
\begin{enumerate}[resume]
\item Comma categories $F \downarrow G$ of accessible functors are accessible.
\item The solution set condition for accessible functors is automatic (we used this in \ref{coro:completecocomplete}).
\end{enumerate}
Curiously, the only locally presentable categories whose opposite categories are also locally presentable are complete lattices. For example, unlike $\catname{Ban}$, $\catname{Hilb}$ cannot be a locally presentable category as it is self-dual by Riesz' representation theorem. Furthermore, opposites like $\catname{Set}^\op$ are examples of non-accessible categories. \\

The presentation of a theory by a sketch $\mathbb S$ is a step towards the \emph{internalisation} of that theory, as we can look at the models of $\mathbb S$ not only in $\catname{Set}$ but in other categories $\cat K$. However, the set-valued models are already a sufficiently general case:
\begin{enumerate}[resume]
\item For every locally presentable category $\cat K$, $\mod(\mathbb S, \cat K)$ is accessible.
\item If $\mathbb S$ is a limit sketch, $\mod(\mathbb S, \cat K)$ is locally presentable. More explicitly, given $\cat K \cong \mod(\mathbb T)$, we can define a tensor product $\mathbb S \otimes \mathbb T$ of sketches such that 
\[ \mod(\mathbb S, \mod(\mathbb T)) \cong \mod(\mathbb S \otimes \mathbb T, \catname{Set}). \] 
\end{enumerate}

At last, it is interesting to note that many structural questions about locally presentable and accessible categories are dependent on the underlying set theory. For example, the strengthenings of the facts above to the accessible case

\begin{enumerate}[resume]
\item Are all accessible categories co-well-powered?
\item Is $\mod(\mathbb S, \cat K)$ accessible for every \emph{accessible} category $\cat K$?
\end{enumerate} 

are independent of ZFC. They do in fact follow from assuming a powerful large cardinal axiom called \emph{Vopěnka's principle}: \\

We call a class $G$ of objects of a category \emph{rigid} if the full subcategory induced by $G$ is discrete, that is there are no non-identity morphisms between the objects. Vopěnka's principle states that the category of simple graphs contains no large rigid class. \\

Vopěnka's principle can be strengthened to the assertion that no accessible category contains a large rigid class. As $\catname{Elem}(\Phi)$ is accessible, we can use this to state an embedding principle in logic: ``For every large class of models of $\Phi$, one model has an elementary embedding into another one.'' \\

Lastly, Vopěnka's principle allows us to simplify the definition of local presentability to a point where there is no reference to presentable objects at all: A category is locally presentable if and only if is cocomplete and has a small full subcategory $\cat A$ such that every object is a colimit of $\cat A$-objects.