\section{Sketches}
\label{sec:sketches}

\subsection{Definition of sketches}

The idea of presenting theories by sketches originated from C. Ehresmann and his French school of category theorists \cite{Ehresmann}. \\

In this section, I will introduce the formalism and give examples of different types of sketches. We will see how the geometric properties of the sketch relate to categorical properties of its models and make connections to algebra and the syntax of the corresponding logical theories. \\

% More precise references
Sketches are covered in various books including \cite{MakkaiPare, TTT, CTCS, elephant}, though precise definitions vary slightly. Most examples follow \cite{AdamekRosicky}, who are also most explicit about the equivalent logical formulation (introducing \emph{essentially algebraic theories}).

\begin{Definition}[Sketch]
A sketch is tuple $\mathbb S = (\cat A, \dist L, \dist C)$ where $\cat A$ is a small category, $\dist L$ is a set of cones in $\cat A$ and $\dist C$ a set of cocones in $\cat A$.
\end{Definition}

For convenience, we treat our cones not as natural transformations but as diagrams $D : \cat I^+ \to \cat A$ of a special form. $\cat I^+$ consists of a distinguished apex $I^+$ and a basis diagram $\cat I$ with unique morphisms $(I^+ \to B)_{B \in \cat I}$ and no morphisms from the basis into $\cat I^+$; analogously for cocones. \\

Sketches form a category $\catname{Sk}$ taking morphisms $(\cat A, \dist L, \dist C) \to (\cat A', \dist{L'}, \dist{C'})$ to be functors $F : \cat A \to \cat A'$ that respect the distinguished cones and cocones, that is for all $D \in \dist L$, we have $F \circ D \in \dist{L'}$ and similarly $F \circ D \in \dist{C'}$ for all $D \in \dist C$. \\

\textbf{Remark:} The practical way to write down sketches for presenting a theory will look slightly different: Instead of the category $\cat A$, we will usually specify an underlying graph $G$, together with a set $D$ of commutativity conditions between pairs of paths of the same starting- and endpoint. We will then prescribe the cones and cocones accordingly. Note that we retrieve our original definition by taking $\cat A$ to be the category freely presented by $G$ modulo the congruence given by $D$. 

\begin{Definition}[Model of a sketch]
Given a sketch $\mathbb S = (\cat A, \dist L, \dist C)$ and a category $\cat K$, a \emph{model} of $\mathbb S$ in $\cat K$ is a functor
\[ F : \cat A \to \cat K \]
that turns all distinguished cones into limit cones and cocones into colimit cocones in $\cat K$:\begin{itemize}
\item For all $D \in \dist L$, $F \circ D$ is limiting,
\item For all $D \in \dist C$, $F \circ D$ is colimiting.
\end{itemize}

A \emph{morphism of models} is just a natural transformation between the functors, making the category \[ \mod(\mathbb S, \cat K) \]
of $\mathbb S$-models in $\cat K$ a full subcategory of $\fun{\cat A}{\cat K}$. Set-valued models of sketches will be the most important special case, and we define
\[ \mod(\mathbb S) := \mod(\mathbb S, \catname{Set}) \]
A category is \emph{sketchable} if it is equivalent to $\mod(\mathbb S)$ for a sketch $\mathbb S$.
\end{Definition}

\textbf{Remark:} By ignoring set-theoretic restrictions, we can think of the model category construction as a functor
\begin{equation}
\label{eq:modfunctor} \mod(-, \cat K) : \catname{Sk}^\op \to \catname{CAT}
\end{equation}
Isomorphism of sketches is a stronger notion than equivalence of their categories of models, though. Note that we can assign to every category $\cat K$ an ``underlying large sketch'' 
\[ U(\cat K) = (\cat K, \dist L, \dist C) \]
where $\dist L$ collects all limit cones and $\dist C$ all colimit cones in $\cat K$. Then a model $F : \mathbb S \to \cat K$ is nothing but a sketch morphism $F : \mathbb S \to U(\cat K)$ and (\ref{eq:modfunctor}) is just given by composition of morphisms. \\

We can classify sketches based on the geometric properties of their shapes
\begin{Definition}
We call $\mathbb S = (\cat A, \dist L, \dist C)$
\begin{enumerate}
\item a \emph{limit sketch} if $\dist C$ is empty
\item a \emph{product sketch} if $\dist C$ is empty and all cones in $\dist L$ are \emph{discrete}, i.e. an apex connected to a discrete diagram
\item a \emph{finite product sketch} if $\dist C$ is empty and all cones in $\dist L$ are discrete and finite
\item a \emph{sum sketch} if all cocones are discrete
\end{enumerate}
We furthermore say \emph{mixed sketch} to stress that both $\dist L$ and $\dist C$ are nonempty. \\

We attach two infinite regular cardinal $\lambda_\mathbb S, \mu_\mathbb S$ to the sketch $\mathbb S$ that measure its size:
\begin{enumerate}
\item The \emph{limit size} $\lambda_\mathbb S$ is the least infinite regular cardinal $\geq$ the size of all cones in $\dist L$.\footnote{The size of a cone or cocone is understood to be the number of morphisms of its underlying diagram.}

\item The \emph{total size} $\mu_\mathbb S$ is the least infinite regular cardinal that is $\geq$ the size of all cones in $\dist L$, cocones in $\dist C$ and the cardinalities $|\dist L|$ and $|\dist C|$.
\end{enumerate}
\end{Definition}


\subsection{Limit sketches}

In this section, we will see examples on how to sketch theories for known some categories. We have to be careful to distinguish between the vertices and edges of the sketch, and their ``meaning'' as objects and morphisms in a model. \\

We have already seen the sketch for binary algebras in the introduction. We can extend it in the following way:

\begin{Example}[Unital algebras]\label{ex:unitality}
To sketch the variety $\cat V$ with a neutral element $e$ such that
\[ m(x, e) = x, \]
we start with a graph on vertices $1, a, a^2$ and edges as pictured
\[
\xymatrix{
  & a^2 \ar[rd]^{\pi_2} \ar[d]_{m} \ar[ld]_{\pi_1} & & 1 \ar[d]_{e} \\
a & a & a & a
}\]
Even though $1$ and $a^2$ are suggestively named, these are just symbols at the moment; we need to fix their intended meaning in the models by limits and colimits. To make $a^2$ become the product of two copies of $a$, we added
\[
\xymatrix{
  & p \ar[rd]^{\pi_2} \ar[ld]_{\pi_1} & \\
a & & a
}\]
to $\dist L$. Furthermore, we add to $\dist L$ the empty cone with apex $1$, making it a terminal object. We add an edge
\[ (\id,e) : a \to a^2 \]
and now need to pin down its meaning. To this end, we introduce another edge $! : a \to 1$. As $1$ will be a terminal object, its meaning is already uniquely defined\footnote{and I will by slight abuse of notation write $!$ for all edges into terminal objects}. Now we can add commutativity relations
\[
\xymatrix{
  & & a \ar[d]^{(\id,e)} \ar@/_/[lldd]^{\id} \ar@/^/[rd]^{!}  & & \\
  & & a^2 \ar[lld]^{\pi_1} \ar[rrd]_{\pi_2} & 1 \ar@/^/[rd]^{e} & \\
a & & & & a
}\]
As $a^2$ will be a limit, the morphism $(\id,e)$ into it will be determined through its projections. Lastly, we can add a commutativity relation
\[ m \cdot (\id,e) = \id_a, \]
where the identity morphism $\id_a$ is formally given by the empty path $()$ from $a$ to $a$. \\

As natural transformations between models correspond precisely to the usual algebraic notion of homomorphisms, we found a finite product sketch $\mathbb S$ with two cones such that
\[ \mod(\mathbb S) \simeq \cat V. \]
\end{Example}

\begin{Example}[Groups]
I will just sketch how to sketch the category of groups. For the associative law, we require double and triple products
\[
\xymatrix{
  & a^2 \ar[ld]_{\pi_1} \ar[rd]^{\pi_2} &    & & a^3 \ar[ld]_{p_1} \ar[d]^{p_2} \ar[rd]^{p_3} & \\
a &     & a & a & a & a
}\]
which are defined via discrete limit cones. We want to write down the commutativity 
\[ m \circ (m \circ (p_1, p_2), p_3) = m \circ (p_1, m \circ (p_2,p_3)) \]
between paths $a^3 \to a$. All the subterms have to be represented in the sketch by new morphisms and and their meaning pinned down via their projections, for example by the following commuting diagram
\[
\xymatrix{
& & a^3 \ar[d]^{(p_1,p_2)} \ar@/_2pc/[lldd]_{p_1} \ar@/^2pc/[rrdd]^{p_2} & \\
& & a^2 \ar[lld]_{\pi_1} \ar[rrd]^{\pi_2} & \\
a & & & & a
}\]
Unitality was discussed in \ref{ex:unitality}, and the inversion operation amounts to an edge $\iota : a \to a$ such that \[ m \cdot (\iota, \id) = e \cdot ! = m \cdot (\id, \iota) \]
as paths $a^2 \to a$, where again $!: a^2 \to 1$. 
\end{Example}

This procedure gives an outline of how to turn any set of algebraic identities into a finite product sketch. There is a inverse construction to this.

\begin{Definition}
Let $S$ be a set. An \emph{$S$-sorted set} $A$ is just a functor $A : S \to \catname{Set}$ with $S$ viewed as a discrete category.
\end{Definition}

When $S$ is a non-discrete category, functors $A : S \to \catname{Set}$ can be viewed as $S$-sorted sets endowed with (unary) operations between the sorts. \\

Prescribing product cones for $A$ additionally declares certain sorts to be products of other sorts, allowing us to express operations of higher arities in terms of just unary ones.
 
\begin{Proposition}\label{prop:modelsarevarieties}
Model categories of finite product sketches are precisely the many-sorted varieties.
\end{Proposition}
\begin{Proof}
Given a finite product sketch $\mathbb S = (\cat A, \dist L)$, we define a variety equivalent to its models: Start by defining a signature with one sort for each object of $\cat A$ and unary operation symbols $f : s \to t$ for every morphism $f : s \to t$ in $\cat A$. \\

Add equations
\[ \id_r(x) = x \]
and
\[ f(g(x)) = (f \circ g)(x)  \]
for all morphisms $g : r \to s, f : s \to t$ in $\cat A$ where $x$ is a variable of sort $r$. We can now think of algebras satisfying these equations as functors $F : \cat A \to \catname{Set}$. \\

The limit conditions require us to exhibit certain sorts as the products of others. Thus for each cone $C = (s \xrightarrow{\pi_i} s_i) \in \dist L$, add a ``packing operation'' operation $\hat c$ of arity
\[ \hat c : s_1 \times \cdots \times s_n \to s \]
that is inverse to the projections in the sense that
\begin{align*}
\hat c(\pi_1(x), \ldots, \pi_n(x)) &= x \\
\pi_i(\hat c(x_1, \ldots, x_n)) &= x_i
\end{align*}
for all $i=1,\ldots,n$ where $x : s, x_i : s_i$.
Every functor $F : \cat A \to \catname{Set}$ that satisfies these equations will induce an isomorphism
\[ F(s) \xrightarrow{(F\pi_1,\ldots, F\pi_n)} F(s_1) \times \cdots \times F(s_n) \]
in $\catname{Set}$ and thus be a model of the sketch.
\end{Proof}

Analogously, product sketches with limit size $\lambda$ correspond to $\lambda$-ary varieties. \\

We gain more expressive power by prescribing non-discrete cones:

\begin{Example}[Torsion-free groups]
Consider a quasivariety like torsion-free groups and look at a single implicational axiom like
\[ x + x = 0 \quad \Rightarrow \quad x = 0. \]
We can sketch this using equalisers: Take again vertices $1,a,a^2$ as before and $0_a : 1 \to a$. The set
\[ e = \{ x : x + x = 0 \} \]
is an equalizer
\[
\xymatrix{
& e \ar@/_1pc/[ld]_{j} \ar@/^1pc/[rd] & \\
a \ar@/^/[rr]^{x + x} \ar@/_/[rr]_{0_a \circ !} & & a
}\]
where \[ x + x = + \circ (\id,\id). \]
Now, we add a commutativity condition to express the fact that $e = \{0\}$ by \[ j = 0_a\cdot !. \]
\end{Example}

\begin{Example}[Simple graphs]
A simple graph is a graph, namely a two-sorted algebra on sorts $v,e$ with operations
\[ s, t : e \to v, \]
where there is at most one edge between two vertices, leading to the implicational condition
\begin{equation}\label{eq:graph} s(x) = s(y),\, t(x) = t(y) \quad \Rightarrow \quad x = y \end{equation}
The right hand side can be expressed as a pullback of the map \[ (s,t) : e \to v^. \]
In fact the condition (\ref{eq:graph}) just says that $(s,t)$ is injective, which is a pullback condition
\[
\xymatrix{
 & e \ar[ld]_{\id} \ar[rd]^{id} & \\
e \ar[rd]_{(s,t)} &  & e \ar[ld]^{(s,t)} \\
 & v^2 &
}\]
\end{Example}

\textbf{Remark:} Recall that we could think of models of product sketches as many-sorted algebras where certain sorts encoded products of other sorts. Prescribing more general cones determines certain sorts as \emph{limits} of other sorts, thus subsets of a product subject to equations. For example, the pullback above means that we have a ``partial packing operation'' $\hat c : e \times e \to e$ which is only defined on \emph{compatible inputs}, i.e.
\[ \hat c(x,y) \text{ defined iff } (s,t)(x) = (s,t)(y). \]
This leads to the equivalence of limit sketches with so called \emph{essentially algebraic theories}, which are equational theories with total and partial operations, where the domains of the partial operations are given by equations in the total ones. \cite[Chapter 3.D]{AdamekRosicky} \\

For example, the theories of posets and small categories are essentially algebraic and can be obtained from the theories of simple graphs and graphs. The relevant notion of transitivity or composition can be given as a partial operation $o : e \times e \to e$ with
\[ o(f, g) \text{ defined if } t(f) = s(g). \]

In view of the sketchability theorem, we can identify the locally presentable with essentially algebraic categories. Note also that models of a finite limit sketches can be axiomatised in finitary first-order logic. This will contrast finite mixed sketches, see \ref{ex:cxgraph}. \\

Categories of models of essentially algebraic theories are complete, as their limits can be computed on the level of the underlying sets, sort-by-sort. For limit sketches, this translates back to the following observation:

\begin{Proposition}
For every limit sketch $\mathbb S = (\cat A, \dist L)$, $\mod(S)$ is closed under limits in $\fun{\cat A}{\catname{Set}}$.
\end{Proposition}
\begin{Proof}
Limits in the functor category are computed pointwise and limits commute with limits.
\end{Proof}

\subsection{Mixed sketches}
The introduction of cocones allows for more complicated theories:

\begin{Example}[Fields]
We can almost sketch fields as varieties on the signature $\{0_k,1_k,+,\times,(-)^{-1}\}$ apart from the pathological axiom
\[ x = 0_k \,\vee\,x \times x^{-1} = 1_k. \] 

Introduce a sketch as usual on vertices $1,k,k^2,k^3$ but also add a symbol $k^*$ which shall represent the nonzero elements of the field. This of course means
\[
\xymatrix{
& k & \\
k^* \ar[ru]^{j} & & 1 \ar[lu]_{0_k}
}\]
has to become a coproduct, so add it to $\dist C$. Then define a map 
\[ (x,x^{-1}) : k^* \to k^2 \]
in the obvious way and add a condition on paths $k^* \to k$
\[ \times \circ (x,x^{-1}) = 1_k \circ !. \]
\end{Example}

The category of fields can be neither complete nor cocomplete, as there are only homomorphisms between fields of the same characteristic. However, it still has \emph{connected limits}, so for example the intersection of two subfields is again a subfield.\footnote{The connectedness requirement precisely takes care of the restriction on the characteristics.} \\

This can be traced to the fact that the sketch we just gave for $\catname{Fld}$ is a sum sketch. 

\begin{Proposition}\label{prop:sumconnectedlimits}
For every sum sketch $\mathbb S = (\cat A, \dist L, \dist C)$, $\mod(\mathbb S)$ is closed under connected limits in $\fun{\cat A}{\catname{Set}}$.
\end{Proposition}
\begin{Proof}
Coproducts commute with connected limits in $\catname{Set}$. \cite[Theorem 3.3.2]{MakkaiPare}
\end{Proof}

This is not the case any more for arbitrary cocones in $\dist C$.
\begin{Example}[Connected graphs] 
\label{ex:cxgraph}
Recall the coequaliser cocone in the sketch for connected graphs from the introduction
\[
\xymatrix{
& 1 \\
e \ar@/^/[rr]^{s} \ar@/_/[rr]_{t} \ar[ru] & & v \ar[lu] \\
}\]
\end{Example}
Intersections of connected subgraphs need not be connected. \\

Also, while models of finite limit sketches could be described in finitary first-order logic, this is no longer true for finite mixed sketches. Connected graphs cannot be axiomatised in that way by the compactness theorem. Instead, a countable disjunction
\[ \forall x y \left(x \sim y \vee \exists z(x \sim z \wedge z \sim y) \vee \cdots\right) \] would be needed.

% Notation Consistency for S/S

\begin{Proposition}\label{prop:modsdirectedcolimits}
For any sketch $\mathbb S$, $\mod(\mathbb S)$ is closed under $\lambda_\mathbb S$-filtered colimits in $\fun{\cat S}{\catname{Set}}$.
\end{Proposition}
\begin{Proof}
Consider the pointwise $\lambda_\mathbb S$-filtered colimit $F$ of models $F_i$. Then $F$ will turn all distinguished cocones into colimits because each $F_i$ does and colimits commute with colimits. \\

It will also turn all distinguished cones into limits because by construction, all of those are $\lambda_\mathbb S$-small and $\lambda_\mathbb S$-filtered colimits commute with $\lambda_\mathbb S$-small limits.
\end{Proof}