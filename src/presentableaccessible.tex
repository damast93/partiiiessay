\section{Locally presentable and accessible categories}
\label{sec:presentableaccessible}

% Moral introduction

Every set is the union of its finite subsets. Furthermore, every algebra is the union of its finitely presented subalgebras. We will transfer these properties into a categorical setting. \\

\textbf{Convention: } All categories in this essay will be locally small. All cardinals will be infinite regular cardinals.

\subsection{Directed and filtered colimits}

Let's recall the definitions of directed colimits (confusingly known as ``direct limits'' in algebra):
\begin{Definition}[Directed set]
A nonempty poset is called \emph{$\lambda$-directed} if every subset of cardinality less than $\lambda$ has an upper bound. \footnote{Note that if $\lambda$ was a singular cardinal, a poset is $\lambda$-directed iff it is $\lambda^+$-directed for the cardinal successor $\lambda^+$, which is always regular.}
\end{Definition}

We can generalise the notion of directedness to arbitrary categories.

\begin{Definition}[$\lambda$-small category]
A category is called \emph{$\lambda$-small} if has less than $\lambda$ morphisms.
\end{Definition}

\begin{Definition}[Filtered category]
A nonempty category is called \emph{$\lambda$-filtered} if every $\lambda$-small subcategory has a cocone over it. 
\end{Definition}

A colimit of a diagram $D : \cat I \to \cat K$ is called \emph{$\lambda$-filtered ($\lambda$-directed) colimits} if $\cat I$ is a $\lambda$-filtered category (or a $\lambda$-directed poset considered as a category). It turns out that both notions are equivalent, and we will use them interchangably.

\begin{Proposition}A category $\cat C$ has $\lambda$-filtered colimits iff it has $\lambda$-directed colimits. For such categories, a functor $F : \cat C \to \cat D$ preserves $\lambda$-filtered colimits iff it preserves $\lambda$-directed ones.
\end{Proposition}

\begin{Example}\label{ex:Q}
In $\catname{Grp}$, we have 
\[ "\mathbb Q = \varinjlim \frac 1 n \mathbb Z". \]
More precisely, $\mathbb Q$ is the $\aleph_0$-directed colimit of the diagram
\begin{align*}
 D : (\mathbb N, |) &\to \catname{Grp}, n \mapsto \mathbb Z \\
 (n\, | \,m) &\mapsto \left(\mathbb Z \xrightarrow{m/n} \mathbb Z\right)
\end{align*}
\end{Example}

Note that unlike e.g. coproducts, the underlying set of a filtered colimit is easy to describe. In all varieties, the forgetful functor to $\catname{Set}$ creates filtered colimits.

\subsection{Presentable objects}
\begin{Definition}[Presentable object]
An object $A$ of a category $\cat K$ is called \emph{$\lambda$-presentable} if its covariant Hom-functor 
\[ \hom(A,-) : \cat K \to \catname{Set} \]
preserves $\lambda$-filtered colimits.
\end{Definition}
Let's unravel this definition. Take a $\lambda$-filtered diagram
\[ D : \cat I \to \cat K \]
with colimit $C$ and coprojections $D_i \xrightarrow{c_i} C$, then we get an induced diagram in $\catname{Set}$:
\[
\xymatrix{
\hom(A,D_i) \ar[rd] \ar[dd]_{(D(i \to j))_*} \ar@/^2pc/[rrd]^{(c_i)_*} \\
& \colim_i \hom(A,D_i) \ar[r]^{\kappa} & \hom(A,C) \\
\hom(A,D_j) \ar[ru] \ar@/_2pc/[rru]_{(c_j)_*} \\
}\]
Preserving the colimit means that $\kappa$ is a bijection. We read off the conditions:

%% TODO Adapt to filteredness
\textbf{Surjectivity} Every morphism $f : A \to C$ factors through one of the $D_i$ as
\begin{equation}
f = c_i \circ f_i \text { for some } f_i : A \to D_i.
\end{equation}
\textbf{Injectivity} The factorization is essentially unique, i.e. if $f = c_i\circ  f_i = c_j\circ  f_j$ then $f_i = f_j$ in the colimit, so there is $k \geq i,j$ such that
\begin{equation}
D(i \to k) \circ f_i = D(j \to k) \circ f_j.
\end{equation}

Note that every $\lambda$-presentable object is $\lambda'$-presentable for $\lambda' \geq \lambda$. In the case $\lambda=\aleph_0$, we simply say \emph{finitely presentable}. \\

\textbf{Remark: } The definition of presentable objects really captures what the name suggests: In a variety, an object $A$ is $\lambda$-presentable iff it has a presentation with less than $\lambda$ generators and relations.
\begin{Proof}
\end{Proof}

\begin{Proposition}\label{prop:smallcolim} A $\lambda$-small colimit of $\lambda$-presentable objects is again $\lambda$-presentable.
\end{Proposition}

\begin{Example}\ \\
\begin{itemize}
\item A set $A$ is $\lambda$-presentable iff $|A| < \lambda$.
\item An element $c$ of a lattice $L$ is finitely presentable iff it is a \emph{compact element}, i.e. whenever
\[ c \leq \bigvee_i d_i \]
for a directed join, we have $c \leq d_i$ for some $i$. This really generalizes the usual compactness from the semilattice of open sets of a topological space (every open cover has a finite subcover).
\item The group $\mathbb Z$ is finitely presentable and by Example \ref{ex:Q}, $\mathbb Q$ is a $\aleph_1$-small colimit of these, thus $\aleph_1$-presentable by Proposition \ref{prop:smallcolim}; in fact, the construction gives the presentation
\[ \mathbb Q \cong \left\langle x_n : n \in \mathbb N | x_n = k \cdot x_{nk} \right \rangle\]
\end{itemize}
\end{Example}

\subsection{Locally presentable and accessible categories}

%% Coole einleitende Sätze!

% Change things to filteredness
\begin{Definition}[Locally presentable category]
A category $\cat K$ is \emph{locally $\lambda$-presentable} if
\begin{enumerate}
\item $\cat K$ is cocomplete
\item there is a set $\mathcal A$ of $\lambda$-presentable objects such that every object of $\cat K$ is a $\lambda$-filtered colimit of objects of $\mathcal A$.
\end{enumerate}
\end{Definition}

Accessibility is a weakening on condition (1):

\begin{Definition}[Accessible category]
A category $\cat K$ is \emph{$\lambda$-accessible} if
\begin{enumerate}
\item $\cat K$ is has $\lambda$-filtered colimits
\item there is a set $\mathcal A$ of $\lambda$-presentable objects such that every object of $\cat K$ is a $\lambda$-filtered colimit of objects of $\mathcal A$.
\end{enumerate}
\end{Definition}

We say a category is \emph{accessible} (\emph{locally presentable}) if it is $\lambda$-accessible (locally $\lambda$-presentable) for some $\lambda$. Again, for $\lambda = \aleph_0$, we say locally finitely presentable and finitely accessible. \\

Recall that a set $G$ of $\cat K$-objects is called a \emph{strong generator} if morphisms out of $G$-objects can distinguish morphisms and proper subobjects in $\cat K$. If $\cat K$ is $\lambda$-accessible, the set $\mathcal A$ is a strong generator. For locally presentable categories, we have a converse that allows for a simpler definition
\begin{Proposition}
A cocomplete category $\mathcal K$ is locally $\lambda$-presentable iff it has a strong generator of $\lambda$-presentable objects. 
\end{Proposition}

\begin{Corollary}\label{coro:raise}
Every locally $\lambda$-presentable category is also locally $\lambda'$-presentable for $\lambda' \geq \lambda$.
\end{Corollary}
\begin{Proof}
Take the same strong generator.
\end{Proof}

Note that we can \emph{not} take the same set $\mathcal A$ from the definition. For example, every set is directed colimit of its finite subsets, but not $\aleph_1$-directed colimit of these. Instead we have to add new $\aleph_1$-small colimits that will now form a set of $\aleph_1$-presentable sets, so that every set is $\aleph_1$-directed colimit of its countable subsets. \\

Corollary \ref{coro:raise} is in interesting contrast to accessible categories. We only get the following
\begin{Proposition}
For every regular cardinal $\lambda$ there are arbitrarily large regular cardinals $\mu \geq \lambda$ such that every $\lambda$-accessible category is $\mu$-accessible.
\end{Proposition}

We can now start discussing some examples:

% Example for lambda > w?
\begin{Example} The category $\catname{Set}$ is locally finitely presentable as $\{1\}$ is a strong generator by a single finitely presentable set. \\

Every variety is locally finitely presentable by the strong generator $\{ F(x) \}$ on the one-generator free algebra. \\

% The category of simple graphs

To give an example that is not equivalent to a variety: The category $\catname{Pos}$ of posets is locally finitely presentable, as $\{\mathbf 2\}$ is a strong generator. % Why isn't that a variety?
\end{Example}

\begin{Example}
For every small category $\cat A$, the category of presheaves $\pre{\cat A} = \fun{\cat A^\op}{\catname{Set}}$ is locally finitely presentable. We'll show that the representable presheaves are finitely presentable. Let \[ y : \cat A \to \pre{\cat A}, A \mapsto \hom(-,A) \] denote the Yoneda embedding. By the Yoneda lemma, we have
\begin{align*}
\hom(yA, \colim_i F_i) &\cong (\colim_i F_i)(A) \\
&\cong \colim_i F_i(A) \\
&\cong \colim_i \hom(yA, F_i)
\end{align*}
so $\hom(yA,-)$ preserves in fact all colimits. Now $\pre{\cat A}$ is cocomplete and representable presheaves form a strong generator as well. %proof
\end{Example}

