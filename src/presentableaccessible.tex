\section{Locally presentable and accessible categories}
\label{sec:presentableaccessible}

Every set is the union of its finite subsets. More generally, every algebra in a variety is a filtered colimit of finitely presented algebras and there is, up to isomorphism, only a set of such algebras. These observations will be the starting point for the definiton of locally presentable and accessible categories. \\

Locally presentable categories have been introduced by Gabriel and Ulmer in \cite{GabrielUlmer}. The more recent concept of accessible categories is due to Makkai and Paré. In this introduction, I will follow Chapters 1-2 of their book \cite{MakkaiPare} as well as Chapter 1 of \cite{AdamekRosicky}.\\

\textbf{Convention: } All categories in this essay will be locally small. All cardinals $\mu,\lambda$ will be infinite regular cardinals.

\subsection{Directed and filtered colimits}

Let us recall the definition of directed colimits (confusingly known as ``direct limits'' in algebra):
\begin{Definition}
A nonempty poset is called \emph{$\lambda$-directed} if every subset of cardinality less than $\lambda$ has an upper bound.
\end{Definition}

The notion of directedness generalises from posets to arbitrary categories:

\begin{Definition}\
\begin{enumerate}
\item A small category $\cat C$ is called \emph{$\lambda$-small} if has less than $\lambda$ morphisms. 
\item A small category is called \emph{$\lambda$-filtered} if it is nonempty and every $\lambda$-small subcategory admits a cocone over it. 
\end{enumerate}
A colimit of a diagram $D : \cat I \to \cat K$ is called \emph{$\lambda$-filtered (resp. $\lambda$-directed)} if $\cat I$ is a $\lambda$-filtered category (resp. a $\lambda$-directed poset). We say \emph{filtered (resp. directed)} in the case $\lambda = \aleph_0$.
\end{Definition}

It turns out that both types of colimits are equivalent, and we will use them interchangably.

\begin{Proposition}A category $\cat C$ has $\lambda$-filtered colimits if and only if it has $\lambda$-directed colimits. If $\cat C$ has $\lambda$-filtered colimits, a functor $F : \cat C \to \cat D$ preserves them if and only if it just preserves $\lambda$-directed ones (see \cite[Remark 1.21]{AdamekRosicky}).
\end{Proposition}

\begin{Example}\label{ex:Q}
In $\catname{Grp}$, we have 
\[ \text{``}\mathbb Q = \varinjlim \frac 1 n \mathbb Z\text{''}. \]
More precisely, $\mathbb Q$ is the colimit of the directed diagram
\begin{align*}
 D : (\mathbb N, |) &\to \catname{Grp}, n \mapsto \mathbb Z \\
 (n\, | \,m) &\mapsto \left(\mathbb Z \xrightarrow{m/n} \mathbb Z\right)
\end{align*}
Unlike arbitrary colimits (for example coproducts), the underlying set of a filtered colimit is easy to describe. In all varieties, the forgetful functor to $\catname{Set}$ creates filtered colimits.
\end{Example}

The trademark feature of filtered colimits is that they commute with small limits in $\catname{Set}$.

\begin{Lemma}\label{prop:smallvsfiltered}
For every diagram $D : \cat I \times \cat J \to \catname{Set}$ with $\cat I$ $\lambda$-small and $\cat J$ $\lambda$-filtered, the induced morphism
\[ \colim_j \lim_i D(i,j) \to \lim_i \colim_j D(i,j) \]
is an isomorphism (see \cite[Theorem 1.2.1]{MakkaiPare}).
\end{Lemma}

\subsection{Presentable objects}
\begin{Definition}[Presentable object]\label{def:presentableobject}
An object $A$ of a category $\cat K$ is called \emph{$\lambda$-presentable} if its covariant Hom-functor 
\[ \hom(A,-) : \cat K \to \catname{Set} \]
preserves $\lambda$-filtered colimits.
\end{Definition}
Spelled out, this means that every morphism $A \xrightarrow{f} C = \colim_i D_i$ into a $\lambda$-filtered colimit already factors through one of the coprojections
\[ f = A \xrightarrow{f_i} D_i \xrightarrow{c_i} C \]
and the factorisation is essentially unique, so if \[ f = c_i\cdot f_i = c_j \cdot f_j, \]
there is an index $k$ in the diagram and maps $i \to k, j \to k$ such that
\[ D(i \to k)\cdot f_i = D(j \to k) \cdot f_j. \]

Every $\lambda$-presentable object is $\lambda'$-presentable for $\lambda' \geq \lambda$. In the case $\lambda=\aleph_0$, we simply say \emph{finitely presentable}. \\

\begin{Lemma}\label{prop:smallcolim} A $\lambda$-small colimit of $\lambda$-presentable objects is again $\lambda$-presentable.
\end{Lemma}
\begin{Proof}
Take a $\lambda$-small diagram $(A_i)$ of $\lambda$-presentable objects, and a $\lambda$-filtered diagram $(D_j)$. By property \ref{prop:smallvsfiltered},
\begin{align*}
\hom(\colim_i A_i, \colim_j D_j) &\cong \lim_i \hom(A_i, \colim_j D_j) \\
&\cong \lim_i \colim_j \hom(A_i, D_j) \\
&\cong \colim_j \lim_i \hom(A_i, D_j) \\
&\cong \colim_j \hom(\colim_i A_i, D_j),
\end{align*}
therefore $\hom(\colim_i A_i, -)$ preserves $\lambda$-filtered colimits as claimed.
\end{Proof}

\begin{Example}
In $\catname{Set}$, the singleton set $1$ is finitely presentable as $\hom(1,X) \cong X$ for every set, therefore
\[ \hom(1,\colim_i A_i) \cong \colim_i \hom(1,A_i) \]
for every diagram $(A_i)$. By \ref{prop:smallcolim}, sets of cardinality $< \lambda$ are $\lambda$-presentable (the converse is also true).
\end{Example}

The definition of presentable objects really captures what the name suggests: 

\begin{Proposition}\label{prop:varietypresentable} In a variety, an object $A$ is $\lambda$-presentable if and only if it has a presentation with less than $\lambda$ generators and relations in the usual algebraic sense (see \cite[Chapter 3.A]{AdamekRosicky}).
\end{Proposition}
\textit{Sketch of proof:} Take $A$ with presentation on less than $\lambda$ generators and relations and let $f : A \to C = \colim_i D_i$ be a map into a $\lambda$-filtered colimit. All generators of $A$ have to map set-theoretically into some $D_i$. By filteredness, there is a cocone over these, so all generators map into a single $D$; however the relations do not have to hold in $D$ yet, so we cannot extend the map to a homomorphism of algebras. However the relations do hold in $C$, so each relation has to start holding after some map $D \to D_j$. Take a cocone over these to some $D^*$, through which $f$ now factors. \\

Conversely, let $A$ be $\lambda$-presentable.
\begin{enumerate}
\item \label{item:varietygen} $A$ is generated by less than $\lambda$ generators: $A$ is the $\lambda$-directed union of its subalgebras with less than $\lambda$ generators, ordered by inclusion. The identity $A \xrightarrow{\id} A$ factors through one of these subalgebras as $A \to A_0 \to A$, but $A_0 \to A$ is an inclusion, thus it has to be an isomorphism. Denote the corresponding set of generators by $X$.
\item Let \[ E = \{ (t_1 = t_2) \,|\, A \models (t_1 = t_2) \} \] be the set of equations in the generators $X$ that hold in $A$. $A$ is the $\lambda$-directed colimit of the diagram
\[ D = \{ \langle X|E_0\rangle : E_0 \subseteq E, |E_0| < \lambda \} \]
where the maps $\langle X|E_i\rangle \xrightarrow{\pi_{ij}} \langle X|E_j\rangle$ are the canonical projections corresponding to the inclusions $E_i \subseteq E_j$. Again we get a factorisation 
\[ A \xrightarrow{u} \langle X|E_0 \rangle \xrightarrow{\kappa_0} A \]
of the identity. By \ref{item:varietygen}, $\langle X|E_0 \rangle$ is itself a $\lambda$-presentable object and we have two factorisations $\kappa_0 = \kappa_0 \cdot u\kappa_0$ of a map into $A$. By the essential uniqueness in \ref{def:presentableobject}, we get a factorisation of $\kappa_0$ as

\[ \langle X|E_0 \rangle \xrightarrow{\pi} \langle X|E_1 \rangle \xrightarrow{\kappa_1} A \] satisfying $\pi u \kappa_0 = \pi$. Now, $\kappa_1$ is an isomorphism as
\begin{align*}
\kappa_1 (\pi u) &= \kappa_0 u = \id_A \\
(\pi u) \kappa_1 \cdot \pi &= \pi u \kappa_0 = \id_A \cdot \pi
\end{align*}
and thus $(\pi u) \kappa_1 = \id_A$ as $\pi$ is an epimorphism.
\end{enumerate}

\begin{Example}
The group $\mathbb Z$ is finitely presentable and by Example \ref{ex:Q}, $\mathbb Q$ is a $\aleph_1$-small colimit of copies of $\mathbb Z$, thus it is $\aleph_1$-presentable by Proposition \ref{prop:smallcolim}; indeed it has a corresponding countable presentation as
\[ \mathbb Q \cong \left\langle x_n : n \in \mathbb N\,|\,x_n = k \cdot x_{nk} \right \rangle\]
\end{Example}

\begin{Example}
An element $c$ of a lattice $L$ is finitely presentable if and only if it is a \emph{compact element}, that is whenever
\[ c \leq \bigvee_i d_i \]
for a directed join, there is some $i$ such that $c \leq d_i$. This generalises the topological notion of a compact set, where $L$ is the lattice of open subsets.
\end{Example}


\subsection{Locally presentable and accessible categories}

\begin{Definition}[Locally presentable category]
A category $\cat K$ is \emph{locally $\lambda$-presentable} if
\begin{enumerate}
\item \label{item:kcocomplete} $\cat K$ is cocomplete
\item there is a set $\mathcal A$ of $\lambda$-presentable objects such that every object of $\cat K$ is a $\lambda$-filtered colimit of objects of $\mathcal A$.
\end{enumerate}
\end{Definition}

Accessibility is a weakening of condition \ref{item:kcocomplete}:

\begin{Definition}[Accessible category]
A category $\cat K$ is \emph{$\lambda$-accessible} if
\begin{enumerate}
\item $\cat K$ is has $\lambda$-filtered colimits
\item there is a set $\mathcal A$ of $\lambda$-presentable objects such that every object of $\cat K$ is a $\lambda$-filtered colimit of objects of $\mathcal A$.
\end{enumerate}
\end{Definition}

A category is \emph{accessible} (\emph{locally presentable}) if it is $\lambda$-accessible (locally $\lambda$-presentable) for some $\lambda$. Again, for $\lambda = \aleph_0$, we say locally finitely presentable and finitely accessible. 

\begin{Example}
The category $\catname{Set}$ is locally finitely presentable, as every set is $\aleph_0$-directed colimit of its finite subsets by inclusion.
\end{Example}

Recall that in a category $\cat K$, a set $G$ of objects is called a \emph{strong generator} if morphisms out of $G$-objects can separate morphisms and proper subobjects in $\cat K$. If $\cat K$ is $\lambda$-accessible, the set $\mathcal A$ forms a strong generator for $\cat K$. For locally presentable categories, the converse holds and allows for a simpler definition:

\begin{Lemma}\label{lemma:stronggen}
A cocomplete category $\mathcal K$ is locally $\lambda$-presentable if and only if it has a strong generator of $\lambda$-presentable objects (see \cite[Theorem 1.20]{AdamekRosicky}).
\end{Lemma}

As an immediate corollary, every locally $\lambda$-presentable category is also locally $\lambda'$-presentable for $\lambda' \geq \lambda$; the same strong generator will do. This stands in interesting contrast to the accessible case, where only the following holds: For every regular cardinal $\lambda$ there are arbitrarily large regular cardinals $\mu \geq \lambda$ such that every $\lambda$-accessible category is $\mu$-accessible (see \cite[Theorem 2.14]{AdamekRosicky}). \\

\begin{Example}\
\begin{enumerate}
\item Every variety is locally finitely presentable by the strong generator $\{ F(x) \}$ on the one-generator free algebra.

\item The category $\catname{Pos}$ of posets is locally finitely presentable, as it is cocomplete and $\{\mathbf 2\}$ is a strong generator. 

\item An infinitary ($\lambda$-ary) variety allows operations to have infinite arities $< \lambda$. For example, semilattices with countable suprema can be axiomatised by an equational theory with an operation
\[ \bigvee(x_1, x_2, \ldots), \]
thus forming an $\aleph_1$-ary variety. Analogously to finitary case, $\lambda$-ary varieties are locally $\lambda$-presentable categories. The only difference is that, while infinitary varieties are also cocomplete, $\kappa$-filtered colimits are generally \emph{only} created by the forgetful functor if $\kappa \geq \lambda$.

\item The category $\catname{Fld}$ of fields is finitely accessible. Even though the category is not cocomplete, it has \emph{filtered} colimits created by the forgetful functor to $\catname{Set}$. \\

We captured the algebraic property that $F$ is finitely presentable as $\hom(F,-)$ preserving filtered colimits. In the same way, the statement that $F$ is finitely \emph{generated} translates into $\hom(F,-)$ preserving filtered colimits \emph{of monomorphisms}. As every morphism in $\catname{Fld}$ is a monomorphism, this is no condition and the two notions coincide. Every field is the directed colimit of its finitely generated subfields by inclusion. \\

It remains to show that the class of finitely generated fields is essentially small: If $F$ is finitely generated, we can use transcendence bases to show that there are extensions $F/L/F_p$ where $F_p$ is the prime subfield of $F$, $L=F_p(\alpha_1,\ldots,\alpha_n)$ is purely transcendental and $F/L$ is finitely generated and algebraic. Thus $F/L$ is finite and the ring homomorphism
\[ L[x_1,\ldots,x_m] \to F \]
is surjective. Therefore the finitely generated fields are up to isomorphism quotients of the rings $F_p(\alpha_1, \ldots,\alpha_n)[x_1,\ldots,x_m]$ for the prime fields $F_0 = \mathbb Q$ or $F_p = \mathbb F_p$ for $p$ prime.
\end{enumerate}
\end{Example}

Another important example is the following:
\begin{Example}\label{ex:representablepresentable}
The functor category $\fun{\cat A}{\catname{Set}}$ is locally finitely presentable for every small category $\cat A$. 
\end{Example}
\begin{Proof}
Representable functors are finitely presentable objects. Let $yA = \hom(A,-)$, then $\hom(yA,-)$ actually preserves \emph{all} colimits by the Yoneda lemma, as 
\[ \hom(yA, \colim_i F_i) \cong (\colim_i F_i)(A) \cong \colim_i F_i(A) \cong \colim_i \hom(yA, F_i). \]
Representables form a generator of the functor category. To see that this generator is strong, note that the monomorphisms $F \xrightarrow{\alpha} G$ are pointwise-injective natural transformations because $\catname{Set}$ has pullbacks. Thus a proper subobject $\alpha$ leads to some element $x \in GA \setminus \alpha_A(FA)$ and the natural transformation $yA \xrightarrow{x} G$ doesn't factor through $F$. \\

Note that $\fun{\cat A}{\catname{Set}}$ is equivalent to a variety, as we will see in \ref{prop:modelsarevarieties}, so local presentability is no surprise.
\end{Proof}

