\section{Locally presentable and accessible categories}
\label{sec:presentableaccessible}

% Moral introduction

Every set is the union of its finite subsets. Furthermore, every algebra is the union of its finitely presented subalgebras. We will transfer these properties into a categorical setting. \\

\textbf{Convention: } All categories in this essay will be locally small. All cardinals will be infinite regular cardinals.

\subsection{Directed and filtered colimits}

Let's recall the definitions of directed colimits (confusingly known as ``direct limits'' in algebra):
\begin{Definition}
A nonempty poset is called \emph{$\lambda$-directed} if every subset of cardinality less than $\lambda$ has an upper bound. \footnote{Note that if $\lambda$ was a singular cardinal, a poset is $\lambda$-directed iff it is $\lambda^+$-directed for the cardinal successor $\lambda^+$, which is always regular.}
\end{Definition}

We can generalise the notion of directedness to arbitrary categories.

\begin{Definition}\
\begin{enumerate}
\item A category is called \emph{$\lambda$-small} if has less than $\lambda$ morphisms.
\item A category is called \emph{$\lambda$-filtered} if it is nonempty and every $\lambda$-small subcategory has a cocone over it. 
\end{enumerate}
A colimit of a diagram $D : \cat I \to \cat K$ is called \emph{$\lambda$-filtered ($\lambda$-directed)} if $\cat I$ is a $\lambda$-filtered category (or a $\lambda$-directed poset considered as a category). 
\end{Definition}

It turns out that both notions are equivalent, and we will use them interchangably.

\begin{Proposition}A category $\cat C$ has $\lambda$-filtered colimits iff it has $\lambda$-directed colimits. For such categories, a functor $F : \cat C \to \cat D$ preserves $\lambda$-filtered colimits iff it preserves $\lambda$-directed ones.
\end{Proposition}

\begin{Example}\label{ex:Q}
In $\catname{Grp}$, we have 
\[ \text{``}\mathbb Q = \varinjlim \frac 1 n \mathbb Z\text{''}. \]
More precisely, $\mathbb Q$ is the $\aleph_0$-directed colimit of the diagram
\begin{align*}
 D : (\mathbb N, |) &\to \catname{Grp}, n \mapsto \mathbb Z \\
 (n\, | \,m) &\mapsto \left(\mathbb Z \xrightarrow{m/n} \mathbb Z\right)
\end{align*}
Note that unlike e.g. coproducts, the underlying set of a filtered colimit is easy to describe. In all varieties, the forgetful functor to $\catname{Set}$ creates filtered colimits.
\end{Example}

The trademark feature of filtered colimits is that they commute with small limits in $\catname{Set}$.

\begin{Lemma}\label{prop:smallvsfiltered}
For every diagram $D : \cat I \times \cat J \to \catname{Set}$ with $\cat I$ $\lambda$-small and $\cat J$ $\lambda$-filtered, the induced morphism
\[ \colim_j \lim_i D(i,j) \to \lim_i \colim_j D(i,j) \]
is an isomorphism.
\end{Lemma}

\subsection{Presentable objects}
\begin{Definition}[Presentable object]\label{def:presentableobject}
An object $A$ of a category $\cat K$ is called \emph{$\lambda$-presentable} if its covariant Hom-functor 
\[ \hom(A,-) : \cat K \to \catname{Set} \]
preserves $\lambda$-filtered colimits.
\end{Definition}
Spelled out, this means that every morphism $A \xrightarrow{f} C = \colim_i D_i$ into a $\lambda$-filtered colimit already factors through one of the coprojections
\[ f = A \xrightarrow{f_i} D_i \xrightarrow{c_i} C \]
and the factorisation is essentially unique, so if \[ f = c_i\cdot f_i = c_j \cdot f_j, \]
there is an index $k$ in the diagram and maps $i \to k, j \to k$ such that
\[ D(i \to k)\cdot f_i = D(j \to k) \cdot f_j. \]

Note that every $\lambda$-presentable object is $\lambda'$-presentable for $\lambda' \geq \lambda$. In the case $\lambda=\aleph_0$, we simply say \emph{finitely presentable}. \\

\begin{Lemma}\label{prop:smallcolim} A $\lambda$-small colimit of $\lambda$-presentable objects is again $\lambda$-presentable.
\end{Lemma}
\begin{Proof}
Take a $\lambda$-small diagram $(A_i)$ of $\lambda$-presentable objects, and a $\lambda$-filtered diagram $(D_j)$. By the property \ref{prop:smallvsfiltered}, we get
\begin{align*}
\hom(\colim_i A_i, \colim_j D_j) &\cong \lim_i \hom(A_i, \colim_j D_j) \\
&\cong \lim_i \colim_j \hom(A_i, D_j) \\
&\cong \colim_j \lim_i \hom(A_i, D_j) \\
&\cong \colim_j \hom(\colim_i A_i, D_j),
\end{align*}
therefore $\hom(\colim_i A_i, -)$ preserves $\lambda$-filtered colimits as claimed.
\end{Proof}

\begin{Example}
In $\catname{Set}$, the singleton set $1$ is finitely presentable as $\hom(1,X) \cong X$ for every set, therefore
\[ \hom(1,\colim_i A_i) \cong \colim_i \hom(1,A_i) \]
for every diagram $(A_i)$. By \ref{prop:smallcolim}, a nonempty set $A$ is $\lambda$-presentable iff $|A| < \lambda$.
\end{Example}

The definition of presentable objects really captures what the name suggests: 

\begin{Proposition}\label{prop:varietypresentable} In a variety, an object $A$ is $\lambda$-presentable iff it has a presentation with less than $\lambda$ generators and relations in the usual algebraic sense.
\end{Proposition}
\textit{Idea: } Take $A$ with presentation on less than $\lambda$ generators and relations and take map $f : A \to C = \colim_i D_i$. Then all generators have to map set-theoretically into some $D_i$. Take a cocone over these, so all generators map into a single $D$, but the relations don't have to hold in $D$ yet, so we can't extend to a homomorphism of algebras. However the relations hold in $C$, so each relation has to start holding after some $D \to D_j$. Take a cocone over these to some $D^*$, now we can factor $f$ through it. \\

Conversely, let $A$ be $\lambda$-presentable.
\begin{enumerate}
\item \label{item:varietygen} $A$ is generated by less than $\lambda$ generators. Note that $A$ is the $\lambda$-directed union of its subalgebras with less than $\lambda$ generators, ordered by inclusion. The identity $A \xrightarrow{\id} A$ factors through one of these subalgebras as $A \to A_0 \to A$, but $A_0 \to A$ is an inclusion, thus we have an isomorphism. Denote the set of generators by $X$ where $|X| < \lambda$
\item Let \[ E = \{ (t_1 = t_2) \,|\, A \models t_1 = t_2 \} \] be the set of equations in the generators $X$ that hold in $A$. $A$ is the $\lambda$-directed colimit of the diagram
\[ D = \{ \langle X|E_0\rangle : E_0 \subseteq E, |E_0| < \lambda \} \]
where the maps $\langle X|E_i\rangle \xrightarrow{\pi_{ij}} \langle X|E_j\rangle$ are the canonical projections corresponding to the inclusions $E_i \subseteq E_j$. Again we get a factorisation 
\[ A \xrightarrow{u} \langle X|E_0 \rangle \xrightarrow{\kappa_0} A \]
of the identity. Note that by \ref{item:varietygen}, $\langle X|E_0 \rangle$ is itself $\lambda$-presentable object and we have two factorisations $\kappa_0 = \kappa_0 \cdot u\kappa_0$ of a map into $A$. By the essential uniqueness in \ref{def:presentableobject}, we get a factorisation of $\kappa_0$ through some $\langle X|E_1 \rangle \xrightarrow{\kappa_1} A$ satisfying $\pi u \kappa_0 = \pi$. Now, $\kappa_1$ is an isomorphism as
\begin{align*}
\kappa_1 (\pi u) &= \kappa_0 u = \id_A \\
(\pi u) \kappa_1 \cdot \pi &= \pi u \kappa_0 = \id_A \cdot \pi
\end{align*}
thus $(\pi u) \kappa_1 = \id_A$ as $\pi$ is an epimorphism.
\end{enumerate}

\begin{Example}
The group $\mathbb Z$ is finitely presentable and by Example \ref{ex:Q}, $\mathbb Q$ is a $\aleph_1$-small colimit of these, thus $\aleph_1$-presentable by Proposition \ref{prop:smallcolim}; note that indeed we have a presentation
\[ \mathbb Q \cong \left\langle x_n : n \in \mathbb N\,|\,x_n = k \cdot x_{nk} \right \rangle\]
\end{Example}

% Remove this?!
\begin{Example}
An element $c$ of a lattice $L$ is finitely presentable iff it is a \emph{compact element}, i.e. whenever
\[ c \leq \bigvee_i d_i \]
for a directed join, we have $c \leq d_i$ for some $i$. This really generalizes the usual compactness from the semilattice of open sets of a topological space (every open cover has a finite subcover).
\end{Example}


\subsection{Locally presentable and accessible categories}

%% Coole einleitende Sätze!

% Change things to filteredness
\begin{Definition}[Locally presentable category]
A category $\cat K$ is \emph{locally $\lambda$-presentable} if
\begin{enumerate}
\item $\cat K$ is cocomplete \label{item:cocomplete}
\item there is a set $\mathcal A$ of $\lambda$-presentable objects such that every object of $\cat K$ is a $\lambda$-filtered colimit of objects of $\mathcal A$.
\end{enumerate}
\end{Definition}

Accessibility is a weakening on condition \ref{item:cocomplete}:

\begin{Definition}[Accessible category]
A category $\cat K$ is \emph{$\lambda$-accessible} if
\begin{enumerate}
\item $\cat K$ is has $\lambda$-filtered colimits
\item there is a set $\mathcal A$ of $\lambda$-presentable objects such that every object of $\cat K$ is a $\lambda$-filtered colimit of objects of $\mathcal A$.
\end{enumerate}
\end{Definition}

We say a category is \emph{accessible} (\emph{locally presentable}) if it is $\lambda$-accessible (locally $\lambda$-presentable) for some $\lambda$. Again, for $\lambda = \aleph_0$, we say locally finitely presentable and finitely accessible. 

\begin{Example}
The category $\catname{Set}$ is locally finitely presentable, as every set is $\aleph_0$-directed colimit of its finite subsets.
\end{Example}

Recall that a set $G$ of $\cat K$-objects is called a \emph{strong generator} if morphisms out of $G$-objects can distinguish morphisms and proper subobjects in $\cat K$. If $\cat K$ is $\lambda$-accessible, the set $\mathcal A$ is a strong generator. For locally presentable categories, we have a converse that allows for a simpler definition:

\begin{Lemma}\label{lemma:stronggen}
A cocomplete category $\mathcal K$ is locally $\lambda$-presentable iff it has a strong generator of $\lambda$-presentable objects. 
\end{Lemma}

As an immediate corollary, we get that every locally $\lambda$-presentable category is also locally $\lambda'$-presentable for $\lambda' \geq \lambda$. The same strong generator will do! This situation stands in interesting contrast with accessible categories, where we only get the following: For every regular cardinal $\lambda$ there are arbitrarily large regular cardinals $\mu \geq \lambda$ such that every $\lambda$-accessible category is $\mu$-accessible. \\

Using \ref{lemma:stronggen} we can now give more examples

\begin{Example}
% Is this really true??
Every variety is locally finitely presentable by the strong generator $\{ F(x) \}$ on the one-generator free algebra. \\

To give an example that is not equivalent to a variety: The category $\catname{Pos}$ of posets is locally finitely presentable, as $\{\mathbf 2\}$ is a strong generator. % Why isn't that a variety?
\end{Example}

% Functor categories?
% Examples for accessible categories
% Example for lambda > w? Accessible example?

