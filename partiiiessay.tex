\documentclass[11pt,a4paper]{article}

% Typography
\usepackage[T1]{fontenc}
\usepackage[english]{babel}

%\usepackage[condensed,math]{anttor}
%\usepackage{libertine}
%\setmainfont[Ligatures=TeX]{Linux Libertine O}
 % \usepackage[light]{kpfonts}
%\usepackage{mathpazo}

\usepackage{fontspec}
\usepackage{microtype}

\usepackage{etoolbox}
%\newfontfamily\quotefont{Antykwa Torunska Condensed}
%\AtBeginEnvironment{quote}{\quotefont}

% Maths symbols
\usepackage{amsmath}
\usepackage{amsfonts}
\usepackage{amssymb}

% Commutative diagrams
\usepackage[all,cmtip]{xy}

% Bibliography
%\usepackage{doc}
\usepackage[style=alphabetic, backend=biber]{biblatex} 
%\renewcommand*{\intitlepunct}{}

\addbibresource{src//bibliography.bib}

% Layout setup
\usepackage[left=3cm,right=3cm,top=1.5cm,bottom=1.2cm,includeheadfoot]{geometry}

% Creating graphics
\usepackage{tikz}
\usepackage{graphicx}

% Proof enviroments
\usepackage[standard,amsmath,thmmarks]{ntheorem}

% Enumeration style
\renewcommand\labelenumi{(\roman{enumi})}
\renewcommand\theenumi\labelenumi

\usepackage{enumitem} % Resume enumerations

% Incremental numbering per section
\theoremstyle{margin}
\theorembodyfont{\normalfont}
\newtheorem{counter}{counter}[section]

% TODO Check font 
%\theorembodyfont{\slshape}

\renewtheorem{Example}[counter]{Example}
\renewtheorem{Definition}[counter]{Definition}

\renewtheorem{Corollary}[counter]{Corollary}
\renewtheorem{Lemma}[counter]{Lemma}
\renewtheorem{Proposition}[counter]{Proposition}
\renewtheorem{Theorem}[counter]{Theorem}

\theoremstyle{nonumberplain}
\theoremsymbol{\ensuremath{\,\hfill\square}}
\theoremheaderfont{\normalfont\scshape}
\theorembodyfont{\normalfont}
\theoremseparator{:} 
\renewtheorem{Proof}{Proof}

% Custom maths & symbol definitions
\newcommand{\id}{\textrm{id}}
\renewcommand{\circ}{\cdot}
\newcommand{\op}{\textrm{op}}
\newcommand{\el}{\textrm{el}}
\renewcommand{\flat}{\mathbf{Flat}}
\newcommand{\cont}{\mathbf{Cont}}
\newcommand{\obj}{\textrm{ob}}
\newcommand{\sub}{\mathbf{Sub}}
\newcommand{\dist}[1]{\textbf{#1}}
\newcommand{\pre}[1]{\widehat{#1}}


\newcommand{\fun}[2]{[#1,#2]}

\renewcommand{\mod}{\mathbf{Mod}}
\renewcommand{\hom}{\textrm{Hom}}
\newcommand{\cat}[1]{\mathcal{#1}}
\newcommand{\catname}[1]{\mathbf{#1}}
\renewcommand{\lim}{\operatorname{lim}}
\newcommand{\colim}{\operatorname{colim}}
\DeclareMathOperator*{\bigcolim}{colim}

%\def\[#1\]{%
%  \begin{equation}#1\end{equation}%
% }

% Hyperlinks & PDF setup
\providecommand\phantomsection{} 
\usepackage[colorlinks=true, linkcolor=blue, citecolor=blue]{hyperref}
\hypersetup{
    hypertexnames=false,
    pdftitle={Locally presentable and accessible categories},
    pdfauthor={Dario Stein},
    pdfsubject={Part III Essay},
    pdfkeywords={locally presentable category, accessible category, category theory, universal algebra, sketch, part iii essay},
}

%%%%%%%%%%%%%%%%%%%%%%%%%%%%%%%%%%%%%%%%%%%%%%%%

\title{Locally presentable and accessible categories}

\begin{document}

\maketitle

\pagebreak
\tableofcontents
\pagebreak

% Start numbering with 0
% \setcounter{section}{-1}

\parindent0cm

% Uncomment to display

\section*{Introduction}
\phantomsection
\addcontentsline{toc}{section}{Introduction}

When thinking of a restricted class of categories with pleasant properties, \emph{(finitary) varieties} are among the first candidates: \emph{Sets}, \emph{groups}, \emph{rings} all form complete and cocomplete categories, are well-powered, well-copowered, have free objects, closure properties, commutativity ... \\

A variety $\cat K$ is given through a finitary signature $\Sigma$ and a set of term equations over $\Sigma$. Then $\cat K$ is the full subcategory of the algebras $\catname{Alg}(\Sigma)$ that satisfy all equations. Universal algebra has the following famous characterisation of varieties by Birkhoff's HSP-Theorem: Let $\cat K$ be a full subcategory of $\catname{Alg}(\Sigma)$. Then the following are equivalent
\begin{enumerate}
\item $\cat K$ is a variety
\item $\cat K$ is closed under \underline{h}omomorphic images, \underline{s}ubobjects and \underline{p}roducts in $\catname{Alg}(\Sigma)$.
\end{enumerate}
This is a first correspondence of syntactical properties of a theory and properties of its model class. For example, we can deduce that the class of \emph{fields} cannot be described with a universal set of axioms, as the componentwise product of fields fails to be a field. \footnote{Of course, the category of fields cannot be equivalent to a variety either, as it's not complete. \emph{What characteristic would a terminal field have?}}  Still, Birkhoff's theorem is very much an extrinsic characterisation of varieties. Can we tell if a category is \emph{equivalent} to a variety, even when it doesn't look algebraic at all? In other words, what can we infer from categorical properties of $\cat K$ alone? \\

% generalize while maintaining pleasant properties
In this essay, I will introduce the theory of \emph{locally presentable} and \emph{accessible} categories. Both are clean, intrinsically defined categorical properties: Local presentability roughly means that a cocomplete category is generated in a certain sense by ``small objects'' via colimits. This notion already covers a wide range of everyday categories, including all varieties, quasivarieties, but also Banach spaces or posets. \\

Accessibility weakens the requirement of cocompleteness, giving us back wilder categories like the categories of fields, linear orders, sets with monomorphisms or connected graphs. \\

It turns out that again, both locally presentable and accessible categories can be axiomatised as models of certain kinds of theories, e.g. in infinitary propositional logic. Instead of introducing their syntax, we'll take another approach using ``language'' of \emph{sketches}: \\

A sketch is merely a small category $\cat A$ with distinguished cones and cocones. A \emph{model} of the sketch is a functor $F : \cat A \to \catname{Set}$ that turns the distinguished cones and cocones into limits and colimits. For example, consider a sketch $\cat A$ with two objects and three morphisms
\[
\xymatrix{
  & p \ar[rd]^{\pi_2} \ar[d]_{m} \ar[ld]_{\pi_1} & \\
a & a & a
}\]
with the prescribed discrete cone
\[
\xymatrix{
  & p \ar[rd]^{\pi_2} \ar[ld]_{\pi_1} & \\
a & & a
}\]
A model $F : \cat A \to \catname{Set}$ consists of two sets $X=F(a)$ and $P=F(p)$ where
\[
\xymatrix{
  & F(p) \ar[rd]^{F(\pi_2)} \ar[ld]_{F(\pi_1)} & \\
F(A) & & F(A)
}\]
is a limit, thus \[ P \cong X \times X \] 
and a map $F(m) : P \to X$. We can therefore identify the models of $\mathcal A$ as algebras with a binary operation $m$. \\

In fact, sketches eliminate the need to define terms and formulas for our theories entirely. We have a purely categorical concept right at our disposal. Still, limits and colimits provide us with all the expressive power we need. This will be our main theorem:

\begin{Theorem}[Sketchability]\ \\
\begin{enumerate}
\item A category is accessible if and only if it is \emph{sketchable}, i.e. equivalent to the category of models of a sketch. 

\item A category is locally presentable iff it is sketchable by a \emph{limit sketch}, i.e. a sketch that only prescribes \emph{cones}.

\item A category is a (many-sorted) variety iff it is sketchable by only prescribing finite discrete cones.
\end{enumerate}
\end{Theorem}

The main goal of this essay is to give a proof of the accessible $\Leftrightarrow$ sketchable-equivalence. In sections \ref{sec:presentableaccessible} and \ref{sec:sketches}, I will introduce our categorical notions and the formalism of sketches. I'll give a parallel treatment of locally presentable and accesible categories for as long as possible, but eventually focus on the accessible case. \\

In section \ref{seq:catstoskeches}, I will analyse the structure of accessible categories further and show how to turn them into functor categories and models of a sketch. Section \ref{sec:sketchesaccessible} will contain the other direction of the theorem, that model categories of sketches are indeed accessible.% \pagebreak
\section{Locally presentable and accessible categories}
\label{sec:presentableaccessible}

All categories in this essay will be locally small. All cardinals will be infinite regular cardinals.

\subsection{Directed and filtered colimits}

Let's recall the definition of a directed set
\begin{Definition}[Directed set]
A poset is called \emph{$\lambda$-directed} if every subset of cardinality less than $\lambda$ has an upper bound.
\end{Definition}
Noe that if $\lambda$ was a singular cardinal, a poset is $\lambda$-directed iff it is $\lambda^+$-directed for the cardinal successor $\lambda^+$, which is always regular. \\

We can generalise the notion of directedness to arbitrary categories.

\begin{Definition}[Small category]
A category is called \emph{$\lambda$-small} if has less than $\lambda$ morphisms.
\end{Definition}

\begin{Definition}[Filtered category]
A category is called \emph{$\lambda$-filtered} if every $\lambda$-small subcategory has a cocone over it. 
\end{Definition}

We are interested in \emph{$\lambda$-filtered ($\lambda$-directed) colimits}, i.e. colimits of diagrams of shape $\cat I$, where $\cat I$ is a $\lambda$-filtered category (or a $\lambda$-directed poset considered as a category). It turns out these notions are equivalent, hence we can reduce to the simpler directed case.

\begin{Proposition}A category $\cat C$ has $\lambda$-filtered colimits iff it has $\lambda$-directed colimits. For such categories, a functor $F : \cat C \to \cat D$ preserves $\lambda$-filtered colimits iff it preserves $\lambda$-directed ones.
\end{Proposition}

\begin{Example}\label{ex:Q}
In $\catname{Grp}$, we have 
\[ "\mathbb Q = \varinjlim \frac 1 n \mathbb Z". \]
More precisely, $\mathbb Q$ is the $\aleph_0$-directed colimit of the diagram
\begin{align*}
 D : (\mathbb N, |) &\to \catname{Grp}, n \mapsto \mathbb Z \\
 (n\, | \,m) &\mapsto \left(\mathbb Z \xrightarrow{m/n} \mathbb Z\right)
\end{align*}
\end{Example}

Note that unlike e.g. coproducts, the underlying set of a directed colimit is easy to describe. For all varieties, the forgetful functor to $\catname{Set}$ creates directed colimits.

\subsection{Presentable objects}
\begin{Definition}[Presentable object]
An object $A$ of a category $\cat K$ is called \emph{$\lambda$-presentable} if its covariant Hom-functor 
\[ \hom(A,-) : \cat K \to \catname{Set} \]
preserves $\lambda$-directed colimits.
\end{Definition}
Let's unravel this definition. Take a $\lambda$-directed diagram
\[ D : (\cat I,\leq) \to \cat K \]
with colimit $C$ and coprojections $D_i \xrightarrow{c_i} C$, then we get an induced diagram in $\catname{Set}$:
\[
\xymatrix{
\hom(A,D_i) \ar[rd] \ar[dd]_{(D(i \to j))_*} \ar@/^2pc/[rrd]^{(c_i)_*} \\
& \colim_i \hom(A,D_i) \ar[r]^{\kappa} & \hom(A,C) \\
\hom(A,D_j) \ar[ru] \ar@/_2pc/[rru]_{(c_j)_*} \\
}\]
Preserving the colimit means that $\kappa$ is a bijection. We read off the conditions

\textbf{Surjectivity} Every morphism $f : A \to C$ factors through one of the $D_i$ as
\begin{equation}
f = c_i \circ f_i \text { for some } f_i : A \to D_i.
\end{equation}
\textbf{Injectivity} The factorization is essentially unique, i.e. if $f = c_i\circ  f_i = c_j\circ  f_j$ then $f_i = f_j$ in $\colim_i \hom(A,D_i)$, so there is $k \geq i,j$ such that
\begin{equation}
D(i \to k) \circ f_i = D(j \to k) \circ f_j.
\end{equation}

\textbf{Remark: } For $\lambda \leq \lambda'$, $\lambda$-presentable objects are $\lambda'$-presentable. For the case $\lambda=\aleph_0$, we simply say \emph{finitely presentable}. \\

Presentable objects really capture what the name suggests: In a variety, an object $A$ is $\lambda$-presentable iff it has a presentation with less than $\lambda$ generators and relations.

\begin{Proposition}A $\lambda$-small colimit of $\lambda$-presentable objects is again $\lambda$-presentable.
\end{Proposition}

\begin{Example}\ \\
\begin{itemize}
\item A set $A$ is $\lambda$-presentable iff $|A| < \lambda$.
\item An element $c$ of a lattice $L$ is finitely presentable iff it is a \emph{compact element}, i.e. whenever
\[ c \leq \bigvee_i d_i \]
for a directed join, we have $c \leq d_i$ for some $i$. This really generalizes the usual compactness from the semilattice of open sets of a topological space (every open cover has a finite subcover).
\item The group $\mathbb Z$ is finitely presentable and by Example \ref{ex:Q}, $\mathbb Q$ is a $\aleph_1$-small colimit of these, thus $\aleph_1$-presentable; this corresponds to the countable presentation
\[ \mathbb Q \cong \left\langle x_n : n \in \mathbb N | x_n = k \cdot x_{nk} \right \rangle\]
\end{itemize}
\end{Example}

\subsection{Locally presentable and accessible categories}

\begin{Definition}[Locally presentable category]
A category $\cat K$ is \emph{locally $\lambda$-presentable} if
\begin{enumerate}
\item $\cat K$ is cocomplete
\item there is a set $\mathcal A$ of $\lambda$-presentable objects such that every object of $\cat K$ is a $\lambda$-directed colimit of objects of $\mathcal A$.
\end{enumerate}
We say $\cat K$ is \emph{locally presentable} if it is locally $\lambda$-presentable for some $\lambda$.
\end{Definition}

Accessibility is a weakening on condition (1)

\begin{Definition}[Accessible category]
A category $\cat K$ is \emph{$\lambda$-accessible} if
\begin{enumerate}
\item $\cat K$ is has $\lambda$-directed colimits
\item there is a set $\mathcal A$ of $\lambda$-presentable objects such that every object of $\cat K$ is a $\lambda$-directed colimit of objects of $\mathcal A$.
\end{enumerate}
We say $\cat K$ is accessible if it is $\lambda$-accessible for some $\lambda$.
\end{Definition}

Again, for $\lambda = \aleph_0$, we say locally finitely presentable and finitely accessible. 

Recall that a set $G$ of $\cat K$-objects is called a \emph{strong generator} if morphisms out of $G$-objects can distinguish morphisms and proper subobjects in $\cat K$. If $\cat K$ is $\lambda$-accessible, the set $\mathcal A$ is a strong generator. For locally presentable categories, we have a converse that allows for a simpler definition
\begin{Proposition}
A category $\mathcal K$ is locally $\lambda$-presentable iff it has a strong generator of $\lambda$-presentable objects. 
\end{Proposition}
\begin{Corollary} The category $\catname{Set}$ is locally finitely presentable as $\{1\}$ is a strong generator by a single finitely presentable set. In fact, every set is directed colimit of its finite subsets. \\

Every variety is locally finitely presentable by the strong generator $\{ F(x) \}$ on the one-generator free algebra. 
\end{Corollary}

\begin{Corollary}\label{coro:raise}
Every locally $\lambda$-presentable category is also locally $\lambda'$-presentable for $\lambda' \geq \lambda$.
\end{Corollary}
\begin{Proof}
Take the same strong generator.
\end{Proof}

Note that we cannot take the same set $\mathcal A$ from the definition. For example, every set is directed colimit of its finite subsets, but not $\aleph_1$-directed colimit of these. Instead we have to add new $\aleph_1$-small colimits that will form a set of $\aleph_1$-presentable sets, so that every set is $\aleph_1$-directed colimit of its countable subsets. \\

Corollary \ref{coro:raise} is in interesting contrast to accessible categories. We only get the following
\begin{Proposition}
For every regular cardinal $\lambda$ there are arbitrarily large regular cardinals $\mu \geq \lambda$ such that every $\lambda$-accessible category is $\mu$-accessible.
\end{Proposition}

\subsection{Canonical colimits}
\begin{Definition}
If $\cat K$ is $\lambda$-accessible, the set of $\lambda$-presentable objects of $\cat K$ is essentially small. We write $\cat K_\lambda$ for any set of representatives of the isomorphism classes.
\end{Definition}


The set $\cat K_\lambda$ is a fortiori a strong generator of $\cat K$ and every object of $\mathcal K$ will be a colimit of $\cat K_\lambda$-objects in a canonical way.

\begin{Definition}[Canonical colimit]
Given a set $\mathcal A \subseteq \cat K$ and an object $K$ of $\cat K$, the \emph{canonical diagram} with respect to $\mathcal A$ is the diagram of all morphisms $A \to K$ with $A \in \mathcal A$, formally given by the forgetful functor
\[ U : \mathcal A/K \to \cat K. \]
We say $K$ is a \emph{canonical colimit} of $\mathcal A$-objects if the canonical cocone 
\[ (A \xrightarrow{f} K) \xrightarrow{f} K \]
is colimiting. 
\end{Definition}
\begin{Definition}[Dense subcategory]
A small full subcategory $\cat A \subseteq \cat K$ is called \emph{dense} if every object of $\cat K$ is a canonical colimit of $\cat A$-objects.
\end{Definition}

\begin{Proposition}
If $\cat K$ is $\lambda$-accessible and $K$ a $\cat K$-object, then the canonical diagram wrt. $\cat K_\lambda$ is $\lambda$-filtered and $K$ its canonical colimit. In particular, $\cat K_\lambda$ is dense.
\end{Proposition} \pagebreak
\section{Sketches}
\label{sec:sketches}

\subsection{Definition of sketches}

Sketches have been introduces by Ehresmann. \\

\begin{Definition}[Sketch]
A sketch is tuple $\mathbb S = (\cat A, \dist L, \dist C)$ where $\cat A$ is a small category, $\dist L$ is a set of cones in $\cat A$ and $\dist C$ a set of cocones in $\cat A$.
\end{Definition}
Note that for convenience, we treat our cones not as natural transformations but as special diagrams $D : \cat I \to \cat A$ where $I$ has a distinguished apex $\cat I^+$ and a bottom $\cat I^\bot$ with morphisms $\cat I^+ \to B$ for all $B \in \cat I^\bot$ and no morphisms from $B$ into the apex; analogously for cocones. \\

Sketches form a category $\catname{Sk}$ by letting morphisms $(\cat A, \dist L, \dist C) \to (\cat A', \dist{L'}, \dist{C'})$ be functors $F : \cat A \to \cat A'$ that respect the distinguished cones and cocones, i.e. for all $D \in \dist L$, we have $F \circ D \in \dist{L'}$ and similarly $F \circ D \in \dist{C'}$ for all $D \in \dist C$. \\

\textbf{Remark: } The way we will actually write down sketches to present a theory will look slightly different: Instead of the category $\cat A$, we will usually specify an underlying graph\footnote{directed multigraph} $G$, together with a set $D$ of commutativity conditions between pairs of paths of the same starting- and endpoint. We will then prescribe the cones and cocones accordingly. Note that we get back to our original definition by taking $\cat A$ to be the category freely presented by the $G$ modulo the congruence given by $D$. 

\begin{Definition}[Model of a sketch]
Given a sketch $\mathbb S = (\cat A, \dist L, \dist C)$ and a category $\cat K$, a \emph{model} of $\mathbb S$ in $\cat K$ is a functor
\[ F : \cat A \to \cat K \]
that turns all distinguished cones into limit cones and cocones into colimit cones in $\cat K$, i.e. \begin{itemize}
\item for all $D \in \dist L$, $F \circ D$ is limiting,
\item for all $D \in \dist C$, $F \circ D$ is colimiting.
\end{itemize}
A \emph{morphism of models} is just a natural transformation between the functors, making the category \[ \mod(\mathbb S, \cat K) \]
of $\mathbb S$-models in $\cat K$ a full subcategory of $\fun{\cat A}{\cat K}$. We write
\[ \mod(\mathbb S) := \mod(\mathbb S, \catname{Set}) \]
for the category of set-valued models of $\mathbb S$. Set-valued models will in fact be so important that we call a category \emph{sketchable} if it is equivalent to $\mod(\mathbb S)$ for a sketch $\mathbb S$.
\end{Definition}

\textbf{Remark: } By ignoring set-theoretic restrictions, we could think of taking model categories as a functor
\begin{equation}
\label{eq:modfunctor} \mod(-, \cat K) : \catname{Sk}^\op \to \catname{CAT}
\end{equation}
In general, non-isomorphic sketches can still have equivalent categories of models though. Note that we can assign to every category $\cat K$ an ``underlying large sketch'' 
\[ U(\cat K) = (\cat K, \dist L, \dist C) \]
where $\dist L$ collects all limit cones, $\dist C$ all colimit cones in $\cat K$. Then a model $F : \mathbb S \to \cat K$ is nothing but a sketch morphism $F : \mathbb S \to U(\cat K)$ and (\ref{eq:modfunctor}) is just given by composition of morphisms. 

We can now give names to the the syntactic (better: geometric?) properties of sketches.
\begin{Definition}
We call $\mathbb S = (\cat A, \dist L, \dist C)$
\begin{enumerate}
\item a \emph{limit sketch} if $\dist C$ is empty
\item a \emph{product sketch} if $\dist C$ is empty and all cones in $\dist L$ are \emph{discrete}, i.e. an apex connected to a discrete diagram
\item a \emph{finite product sketch} if $\dist C$ is empty and all cones in $\dist L$ are discrete and finite
\item \emph{$\lambda$-small} if all cones and cocones have underlying $\lambda$-small diagrams
\end{enumerate}
We furthermore say \emph{mixed sketch} to stress that both $\dist L$ and $\dist C$ are nonempty.
\end{Definition}

\subsection{Examples}

In this section, we will see examples on how to sketch theories for known some categories. Note that we have to be careful to distinguish between the vertices and edges of the sketch, and their ``meaning'' as objects and morphisms in a model. \\

 We have already seen the sketch for a binary algebra in the introduction. Let's extend it in the following way:

\begin{Example}[Unital algebras]\label{ex:unitality}
We want to sketch the variety $\cat V$ with a neutral element $e$ such that
\[ m(x, e) = x. \]
We start with a graph on vertices $1, a, a^2$ and edges as described
\[
\xymatrix{
  & a^2 \ar[rd]^{\pi_2} \ar[d]_{m} \ar[ld]_{\pi_1} & & 1 \ar[d]_{e} \\
a & a & a & a
}\]
Note that even though $1$ and $a^2$ are suggestively named, these are just symbols at the moment. We will make them get their intended meaning in the models of the sketch. To make $a^2$ become the actual product of $a$, we needed to add
\[
\xymatrix{
  & p \ar[rd]^{\pi_2} \ar[ld]_{\pi_1} & \\
a & & a
}\]
to $\dist L$. Furthermore, we add to $\dist L$ the empty cone with apex $1$, making it a terminal object. We add an edge
\[ (\id,e) : a \to a^2 \]
and now need to pin down its meaning. Let's introduce another edge $! : a \to 1$. As $1$ will be a terminal object, its meaning is already uniquely defined\footnote{and I will by slight abuse of notation write $!$ for all edges into terminal objects}. Now we can add commutativity relations
\[
\xymatrix{
  & & a \ar[d]^{(\id,e)} \ar@/_/[lldd]^{\id} \ar@/^/[rd]^{!}  & & \\
  & & a^2 \ar[lld]^{\pi_1} \ar[rrd]_{\pi_2} & 1 \ar@/^/[rd]^{e} & \\
a & & & & a
}\]
As $a^2$ will be a limit, the morphism $(\id,e)$ into it will be determined through its projections and thus be uniquely defined as well. Lastly, we can add a commutativity relation\footnote{note that the identity morphism $\id_a$ is uniquely given through the empty path $()$ from $a$ to $a$}
\[ m \cdot (\id,e) = \id_a. \]
Note that natural transformations between models will precisely correspond to the algebraic notion of homomorphisms. By this construction, we have a finite product sketch $\mathbb S$ with two cones such that
\[ \mod(\mathbb S) \simeq \cat V. \]
\end{Example}

\begin{Example}[Groups]
I will just sketch how to sketch the category of groups. For the associative law, we construct ourselves double and triple products
\[
\xymatrix{
  & a^2 \ar[ld]_{\pi_1} \ar[rd]^{\pi_2} &    & & a^3 \ar[ld]_{p_1} \ar[d]^{p_2} \ar[rd]^{p_3} & \\
a &     & a & a & a & a
}\]
with the corresponding discrete limit cones. We want to write down the commutativity
\[ m \circ (m \circ (p_1, p_2), p_3) = m \circ (p_1, m \circ (p_2,p_3)) \]
between paths $a^3 \to a$. All the intermediate edges have to be added to the sketch and their meaning defined uniquely via their projections, e.g by the following commuting diagram
\[
\xymatrix{
& & a^3 \ar[d]^{(p_1,p_2)} \ar@/_2pc/[lldd]_{p_1} \ar@/^2pc/[rrdd]^{p_2} & \\
& & a^2 \ar[lld]_{\pi_1} \ar[rrd]^{\pi_2} & \\
a & & & & a
}\]
Unitality was discussed in \ref{ex:unitality}, and the inversion operation amounts to an edge $\iota : a \to a$ such that \[ m \cdot (\iota, \id) = e \cdot ! = m \cdot (\id, \iota) \]
as paths $a^2 \to a$, where again $!: a^2 \to 1$. 
\end{Example}

This procedure actually shows how to turn any set of identities into a finite product sketch.
\begin{Proposition}
Model categories of finite product sketches are precisely the many-sorted varieties.
\end{Proposition}
\begin{Proof}
For any variety, we have seen how to turn it into a model category of a sketch. \\

Given a finite product sketch $\mathbb S = (\cat A, \dist L)$, we define a variety in the following way: Start by defining a signature with sorts $\obj(\cat A)$ and add unary operation symbols $f : s \to t$ for every morphism $f : s \to t$ in $\cat A$. \\

Add equations
\[ \id_r(x) = x \]
and
\[ f(g(x)) = (f \circ g)(x)  \]
for all morphisms $g : r \to s, f : s \to t$ in $\cat A$ where $x$ is a variable of sort $r$. We can now think of algebras as functors $F : \cat A \to \catname{Set}$. Now encode the limit conditions. For each cone $C = (s \xrightarrow{\pi_i} s_i) \in \dist L$, add an operation symbol $\hat c$ of arity
\[ \hat c : s_1 \times \cdots \times s_n \to s \]
and add equations
\[ \hat c(\pi_1(x), \ldots, \pi_n(x)) = x \]
for $x : s$ and 
\[ \pi_i(\hat c(x_1, \ldots, x_n)) = x_i \]
for all $i=1,\ldots,n$ where $x_i : s_i$.
Every functor $F : \cat A \to \catname{Set}$ that satisfies these equations will induce an isomorphism
\[ F(s) \xrightarrow{(F\pi_1,\ldots, F\pi_n)} F(s_1) \times \cdots \times F(s_n) \]
in $\catname{Set}$, i.e. be a model of the sketch.
\end{Proof}

\begin{Example}[Torsion-free groups]
Let's take a quasivariety like torsion-free groups and look at the implicational axiom
\[ x + x = 0 \quad \Rightarrow \quad x = 0. \]
We can sketch this using equalisers. Take again vertices $1,a,a^2$ as before and $0_a : 1 \to a$. We can describe the set
\[ e = \{ x : x + x = 0 \} \]
as an equalizer cone
\[
\xymatrix{
& e \ar@/_1pc/[ld]_{j} \ar@/^1pc/[rd] & \\
a \ar@/^/[rr]^{x + x} \ar@/_/[rr]_{0_a \circ !} & & a
}\]
where \[ x + x = + \circ (\id,\id). \]
Now we add commutativity condition expressing $e = \{0\}$ by \[ j = 0_a\cdot !. \]
\end{Example}

\begin{Example}[Simple graphs]
A simple graph is a graph, i.e. a two-sorted variety on sort $v,e$ with operations
\[ s, t : e \to v \]
where there is at most one edge between two vertices, leading to the implicational condition
\begin{equation}\label{eq:graph} s(x) = s(y),\, t(x) = t(y) \quad \Rightarrow \quad x = y \end{equation}
for $x,y : e$. The right hand side can be expressed as a pullback of the map \[ (s,t) : e \to v^2, \]
in fact the condition (\ref{eq:graph}) just says that $(s,t)$ is injective, which is a pullback condition
\[
\xymatrix{
 & e \ar[ld]_{\id} \ar[rd]^{id} & \\
e \ar[rd]_{(s,t)} &  & e \ar[ld]^{(s,t)} \\
 & v^2 &
}\]
\end{Example}

The introduction of cocones allows for more interesting categories
\subsection{Mixed sketches}

\begin{Example}[Fields]
We can almost sketch fields as varieties on the signature $\{0_k,1_k,+,\times,(-)^{-1}\}$ apart from the pathological axiom
\[ x = 0_k \,\vee\,x \times x^{-1} = 1_k. \] 
Introduce a sketch as usual on vertices $1,k,k^2,k^3$ but also add a symbol $k^*$ that we want to represent the nonzero elements of the field. This of course means
\[
\xymatrix{
& k & \\
k^* \ar[ru]^{j} & & 1 \ar[lu]_{0_k}
}\]
has to become a coproduct cone, so add it to $\dist C$. Then define a map 
\[ (x,x^{-1}) : k^* \to k^2 \]
in the obvious way and add a condition on paths $k^* \to k$
\[ \times \circ (x,x^{-1}) = 1_k \circ !. \]
\end{Example}

% Put this in the introduction?
\begin{Example}[Connected graphs]
A surprisingly elegant example is the sketch $\mathbb S$ on vertices $1,e,v$ with terminal object $1$ \[
\xymatrix{
& 1 \\
e \ar@/^/[rr]^{s} \ar@/_/[rr]_{t} \ar[ru] & & v \ar[lu] \\
}\]
Models of $\mathbb S$ will be graphs $(V,E)$, but the distinguished cocone becomes a coequaliser 
\[ 1 \cong V/(s(e)\sim t(e)). \]
The equivalence relation identifies all vertices that are connected by an edge, so there has to be precisely one connected component. $\mathbb S$ axiomatises the category of connected graphs and homomorphisms.
\end{Example}

% Linear orders machen %\pagebreak
\section{From categories to sketches}
\label{seq:catstoskeches}

In this section, we will see how to embed accessible categories into functor categories and turn them into models of a sketch: \\

\subsection{Canonical colimits}

Let $\cat K$ be a $\lambda$-accessible category. We know that every object is $\lambda$-filtered colimit of some set $\cat A$ of $\lambda$-presentable objects. In fact, we can pick a canonical such set generating set by considering all $\lambda$-presentable objects.

\begin{Proposition}
For every $\lambda$-accessible category $\cat K$, the full subcategory of $\lambda$-presentable objects is essentially small. 
\end{Proposition}
\begin{Proof}
Take any $\lambda$-presentable object $B$ and write it as a $\lambda$-filtered colimit $B \xrightarrow{\sim} \colim_i A_i$ of objects in $\cat A$. The isomorphism factors through one of the $A_i$, exhibiting $B$ as a retract $B \xrightarrow{j} A_i \xrightarrow{r} B$ of $A_i$. Note that $jr$ is an idempotent endomorphism of $A_i$, and retracts that induce the same endomorphism are isomorphic. Thus we have an class injection
\[ \{ \text{retracts of } A_i \}/\text{isomorphism} \hookrightarrow \hom(A_i,A_i) \]
where the codomain is a set as $\cat K$ is locally small. Now take the union of the sets of representatives for each $A_i \in \cat A$.
\end{Proof}

Let's write $\cat K_\lambda$ for any set of representatives of the $\lambda$-presentable objects up to isomorphism. Given an object $K$ and small full subcategory $\cat A$ of $\cat K$, the canonical diagram with respect to $\cat A$ is the diagram of all morphisms of $\cat A$-objects into $\cat K$, formally given by the forgetful functor
\[ U : (\cat A/K) \to \cat K. \]
We say that $K$ is the canonical colimit of $\cat A$-objects if the canonical cocone
\[ (A \xrightarrow{f} K) \xrightarrow{f} K \]
is colimiting. We call $\cat A$ \emph{dense} in $\cat K$ if every object of $\cat K$ is the canonical colimit of $\cat A$-objects.

\begin{Proposition}\label{prop:presdense}
For every $\lambda$-accessible category $\cat K$ and $\cat A = K_\lambda$, every object is canonical colimit of $\cat A$-objects and the canonical diagram is $\lambda$-filtered.
\end{Proposition}
\begin{Proof}
\end{Proof}

Given a small full subcategory $\cat A$ of $\cat K$, we have a functor 
\[ E : \cat K \to \fun{\cat A^\op}{\catname{Set}} \]
that sends an object $K$ to the domain restriction of $\hom(-, K)$ to $\cat A^\op$. We can restate the properties of $\cat A$ in terms of $E$ in the following way:
\begin{enumerate}
\item $E$ is fully faithful iff $\cat A$ is dense in $\cat K$.
\item $E$ preserves $\lambda$-filtered colimits iff all objects of $\cat A$ are $\lambda$-presentable.
\end{enumerate}

Together with Proposition \ref{prop:presdense}, we see that for $\lambda$-accessible categories $\cat K$, we get a canonical embedding
\[ E : \cat K \to \fun{\cat A}{\catname{Set}} \text{ where } \cat A = \cat K^\op_\lambda. \]

\subsection{Flat functors}
We want to characterise precisely which subcategories of $\fun{\cat A}{\catname{Set}}$ occur via embeddings of accessible categories.

\begin{Definition}\
We call a functor $F : \cat A \to \catname{Set}$
\begin{enumerate}
\item \emph{$\lambda$-flat} if it is $\lambda$-filtered colimit of representable functors
\item \emph{$\lambda$-continuous} if it preserves $\lambda$-small limits
\end{enumerate}
and we denote the full subcategories of $\fun {\cat A}{\catname{Set}}$ of $\lambda$-flat and $\lambda$-continuous functors as 
\[ \flat_\lambda(\cat A), \cont_\lambda(\cat A) \text{ respectively}. \]
\end{Definition}

\begin{Proposition}
Let $\cat K$ be $\lambda$-accessible and $\cat A = \cat K^\op_\lambda$. Then the essential image of the canonical embedding consists precisely of the $\lambda$-flat functors, and we have an equivalence
\[ \cat K \simeq \flat_\lambda(\cat A). \]
Conversely, for $\cat A$ small, the category $\flat_\lambda(\cat A)$ is $\lambda$-accessible.
\end{Proposition}
\begin{Proof}
\end{Proof}

Note that this description leads to a strong characterisation in the locally presentable case.
\begin{Proposition}
Let $\cat K$ be $\lambda$-accessible \emph{and cocomplete}, i.e. locally $\lambda$-presentable, and $\cat A = \cat K^\op_\lambda$. Then the essential image of the canonical embedding restricts to the $\lambda$-continuous functors and
\[ \cat K \simeq \cont_\lambda(\cat A). \]
\end{Proposition}
\begin{Proof}
\end{Proof}

\begin{Corollary}\
Let $\cat K$ be a locally $\lambda$-presentable category. Then
\begin{enumerate}
\item $\cat K$ is complete, and well-powered \label{item:cw}
\item $\cat K$ is sketchable by a $\lambda$-small limit sketch \label{item:limsketch}
\end{enumerate}
\end{Corollary}
\begin{Proof}
We have an equivalence $K \simeq \cont_\lambda(\cat A)$. The latter category is closed under limits in $\fun{\cat A}{\catname{Set}}$, as limits commute with limits; thus it is complete. Every category with pullbacks and a strong generator is well-powered, showing \ref{item:cw}. For \ref{item:limsketch}, note that by definition
\[ \cont_\lambda(\cat A) \simeq \mod(\mathbb S) \]
where the sketch $\mathbb S$ collects all $\lambda$-small limit cones in $\cat A$.
\end{Proof} %\pagebreak
\section{Models are accessible}
\label{sec:sketchesaccessible} \pagebreak
\section{Conclusion}
\label{sec:conclusion}

% Acknowledge: AK, Ohad Kammar %\pagebreak

% Acknowledgement
\section*{Acknowledgement}
I'd like to thank Ohad Kammar, Sean Moss, Andreas Kostakiotis and Moritz Möller for valuable comments and the interesting discussions that sometimes follow from innocent questions (``is there an algebraic theory that has precisely three models up to isomorphism?'').

% Print bibliography

% Print all of it
\nocite{*}


\addcontentsline{toc}{section}{Bibliography}
\printbibliography[title=Bibliography]

\end{document}
