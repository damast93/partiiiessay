\documentclass[11pt,a4paper]{article}

% Typography
\usepackage[T1]{fontenc}
\usepackage[english]{babel}

%\usepackage[condensed,math]{anttor}
%\usepackage{libertine}
%\setmainfont[Ligatures=TeX]{Linux Libertine O}
 % \usepackage[light]{kpfonts}
%\usepackage{mathpazo}

\usepackage{fontspec}
\usepackage{microtype}

\usepackage{etoolbox}
%\newfontfamily\quotefont{Antykwa Torunska Condensed}
%\AtBeginEnvironment{quote}{\quotefont}

% Maths symbols
\usepackage{amsmath}
\usepackage{amsfonts}
\usepackage{amssymb}

% Commutative diagrams
\usepackage[all,cmtip]{xy}

% Bibliography
%\usepackage{doc}
\usepackage[style=alphabetic, backend=biber]{biblatex} 
%\renewcommand*{\intitlepunct}{}

\addbibresource{src//bibliography.bib}

% Layout setup
\usepackage[left=3cm,right=3cm,top=1.5cm,bottom=1.2cm,includeheadfoot]{geometry}

% Creating graphics
\usepackage{tikz}
\usepackage{graphicx}

% Proof enviroments
\usepackage[standard,amsmath,thmmarks]{ntheorem}

% Enumeration style
\renewcommand\labelenumi{(\roman{enumi})}
\renewcommand\theenumi\labelenumi

\usepackage{enumitem} % Resume enumerations

% Incremental numbering per section
\theoremstyle{margin}
\theorembodyfont{\normalfont}
\newtheorem{counter}{counter}[section]

% TODO Check font 
%\theorembodyfont{\slshape}

\renewtheorem{Example}[counter]{Example}
\renewtheorem{Definition}[counter]{Definition}

\renewtheorem{Corollary}[counter]{Corollary}
\renewtheorem{Lemma}[counter]{Lemma}
\renewtheorem{Proposition}[counter]{Proposition}
\renewtheorem{Theorem}[counter]{Theorem}

\theoremstyle{nonumberplain}
\theoremsymbol{\ensuremath{\,\hfill\square}}
\theoremheaderfont{\normalfont\scshape}
\theorembodyfont{\normalfont}
\theoremseparator{:} 
\renewtheorem{Proof}{Proof}

% Custom maths & symbol definitions
\newcommand{\id}{\textrm{id}}
\renewcommand{\circ}{\cdot}
\newcommand{\op}{\textrm{op}}
\newcommand{\el}{\textrm{el}}
\renewcommand{\flat}{\mathbf{Flat}}
\newcommand{\cont}{\mathbf{Cont}}
\newcommand{\obj}{\textrm{ob}}
\newcommand{\sub}{\mathbf{Sub}}
\newcommand{\dist}[1]{\textbf{#1}}
\newcommand{\pre}[1]{\widehat{#1}}


\newcommand{\fun}[2]{[#1,#2]}

\renewcommand{\mod}{\mathbf{Mod}}
\renewcommand{\hom}{\textrm{Hom}}
\newcommand{\cat}[1]{\mathcal{#1}}
\newcommand{\catname}[1]{\mathbf{#1}}
\renewcommand{\lim}{\operatorname{lim}}
\newcommand{\colim}{\operatorname{colim}}
\DeclareMathOperator*{\bigcolim}{colim}

%\def\[#1\]{%
%  \begin{equation}#1\end{equation}%
% }

% Hyperlinks & PDF setup
\providecommand\phantomsection{} 
\usepackage[colorlinks=true, linkcolor=blue, citecolor=blue]{hyperref}
\hypersetup{
    hypertexnames=false,
    pdftitle={Locally presentable and accessible categories},
    pdfauthor={Dario Stein},
    pdfsubject={Part III Essay},
    pdfkeywords={locally presentable category, accessible category, category theory, universal algebra, sketch, part iii essay},
}

%%%%%%%%%%%%%%%%%%%%%%%%%%%%%%%%%%%%%%%%%%%%%%%%

\title{Locally presentable and accessible categories}

\begin{document}

\maketitle

\pagebreak
\tableofcontents
\pagebreak

% Start numbering with 0
% \setcounter{section}{-1}

\parindent0cm

% Uncomment to display

\section*{Introduction}
\phantomsection
\addcontentsline{toc}{section}{Introduction}

When thinking of a restricted class of categories with pleasant properties, \emph{(finitary) varieties} are among the first candidates: \emph{Sets}, \emph{groups}, \emph{rings} all form complete and cocomplete categories, are well-powered, well-copowered, have free objects, closure properties, commutativity ... \\

A variety $\cat K$ is given through a finitary signature $\Sigma$ and a set of term equations over $\Sigma$. Then $\cat K$ is the full subcategory of the algebras $\catname{Alg}(\Sigma)$ that satisfy all equations. Universal algebra has the following famous characterisation of varieties by Birkhoff's HSP-Theorem: Let $\cat K$ be a full subcategory of $\catname{Alg}(\Sigma)$. Then the following are equivalent
\begin{enumerate}
\item $\cat K$ is a variety
\item $\cat K$ is closed under \underline{h}omomorphic images, \underline{s}ubobjects and \underline{p}roducts in $\catname{Alg}(\Sigma)$.
\end{enumerate}
This is a first correspondence of syntactical properties of a theory and properties of its model class. For example, we can deduce that the class of \emph{fields} cannot be described with a universal set of axioms, as the componentwise product of fields fails to be a field. \footnote{Of course, the category of fields cannot be equivalent to a variety either, as it's not complete. \emph{What characteristic would a terminal field have?}}  Still, Birkhoff's theorem is very much an extrinsic characterisation of varieties. Can we tell if a category is \emph{equivalent} to a variety, even when it doesn't look algebraic at all? In other words, what can we infer from categorical properties of $\cat K$ alone? \\

In this essay, I will introduce the theory of \emph{locally presentable} and \emph{accessible} categories. Both are clean, intrinsically defined categorical properties: Local presentability roughly means that a cocomplete category is generated in a certain sense by ``small objects'' via colimits. This notion already covers a wide range of everyday categories, including all varieties, quasivarieties, but also Banach spaces or posets. \\

Accessibility weakens the requirement of cocompleteness, giving us back wilder categories like the categories of fields, linear orders, sets with monomorphisms or connected graphs. \\

It turns out that again, both locally presentable and accessible categories can be axiomatised as models of certain kinds of theories, e.g. in infinitary propositional logic. Instead of introducing their syntax, we'll take another approach using ``language'' of \emph{sketches}: \\

A sketch is merely a small category $\cat A$ with distinguished cones and cocones. A \emph{model} of the sketch is a functor $F : \cat A \to \catname{Set}$ that turns the distinguished cones and cocones into limits and colimits. For example, consider a sketch $\cat A$ with two objects and three morphisms
\[
\xymatrix{
  & p \ar[rd]^{\pi_2} \ar[d]_{m} \ar[ld]_{\pi_1} & \\
a & a & a
}\]
with the prescribed discrete cone
\[
\xymatrix{
  & p \ar[rd]^{\pi_2} \ar[ld]_{\pi_1} & \\
a & & a
}\]
A model $F : \cat A \to \catname{Set}$ consists of two sets $X=F(a)$ and $P=F(p)$ where
\[
\xymatrix{
  & F(p) \ar[rd]^{F(\pi_2)} \ar[ld]_{F(\pi_1)} & \\
F(A) & & F(A)
}\]
is a limit, thus \[ P \cong X \times X \] 
and a map $F(m) : P \to X$. We can therefore identify the models of $\mathcal A$ as algebras with a binary operation $m$. \\

In fact, sketches eliminate the need to define terms and formulas for our theories entirely. We have a purely categorical concept right at our disposal. Still, limits and colimits provide us with all the expressive power we need. This will be our main theorem:

\begin{Theorem}[Sketchability]\ \\
\begin{enumerate}
\item A category is accessible if and only if it is \emph{sketchable}, i.e. equivalent to the category of models of a sketch. 

\item A category is locally presentable iff it is sketchable by a \emph{limit sketch}, i.e. a sketch that only prescribes \emph{cones}.

\item A category is a (many-sorted) variety iff it is sketchable by only prescribing finite discrete cones.
\end{enumerate}
\end{Theorem}

The main goal of this essay is to give a proof of the accessible $\Leftrightarrow$ sketchable-equivalence. In sections \ref{sec:presentableaccessible} and \ref{sec:sketches}, I will introduce our categorical notions and the formalism of sketches. I'll give a parallel treatment of locally presentable and accesible categories for as long as possible, but eventually focus on the accessible case. \\

In section \ref{seq:catstoskeches}, I will analyse the structure of accessible categories further and show how to turn them into functor categories and models of a sketch. Section \ref{sec:sketchesaccessible} will contain the other direction of the theorem, that model categories of sketches are indeed accessible. \pagebreak
\section{Locally presentable and accessible categories}
\label{sec:presentableaccessible}

Every set is the union of its finite subsets. More generally, every algebra in a variety $\cat K$ is a certain colimit of finitely presented algebras and there is, up to isomorphism, only a set of finitely presented algebras. We want to make precise in what sense this set ``generates'' the category $\cat K$ and then generalise these notions. \\

\textbf{Convention: } All categories in this essay will be locally small. All cardinals will be infinite regular cardinals.

\subsection{Directed and filtered colimits}

Let us recall the definition of directed colimits (confusingly known as ``direct limits'' in algebra):
\begin{Definition}
A nonempty poset is called \emph{$\lambda$-directed} if every subset of cardinality less than $\lambda$ has an upper bound.
\end{Definition}

We can generalise the notion of directedness from posets to arbitrary categories.

\begin{Definition}\
\begin{enumerate}
\item A category is called \emph{$\lambda$-small} if has less than $\lambda$ morphisms.
\item A category is called \emph{$\lambda$-filtered} if it is nonempty and every $\lambda$-small subcategory admits a cocone over it. 
\end{enumerate}
A colimit of a diagram $D : \cat I \to \cat K$ is called \emph{$\lambda$-filtered (resp. $\lambda$-directed)} if $\cat I$ is a $\lambda$-filtered category (resp. a $\lambda$-directed poset). We'll just say \emph{filtered (resp. directed)} for $\lambda = \aleph_0$.
\end{Definition}

It turns out that both notions are equivalent, and we will use them interchangably.

\begin{Proposition}A category $\cat C$ has $\lambda$-filtered colimits iff it has $\lambda$-directed colimits. For such categories, a functor $F : \cat C \to \cat D$ preserves $\lambda$-filtered colimits iff it preserves $\lambda$-directed ones.
\end{Proposition}

\begin{Example}\label{ex:Q}
In $\catname{Grp}$, we have 
\[ \text{``}\mathbb Q = \varinjlim \frac 1 n \mathbb Z\text{''}. \]
More precisely, $\mathbb Q$ is the colimit of the directed diagram
\begin{align*}
 D : (\mathbb N, |) &\to \catname{Grp}, n \mapsto \mathbb Z \\
 (n\, | \,m) &\mapsto \left(\mathbb Z \xrightarrow{m/n} \mathbb Z\right)
\end{align*}
Note that unlike e.g. coproducts, the underlying set of a filtered colimit is easy to describe. In all varieties, the forgetful functor to $\catname{Set}$ creates filtered colimits.
\end{Example}

The trademark feature of filtered colimits is that they commute with small limits in $\catname{Set}$.

\begin{Lemma}\label{prop:smallvsfiltered}
For every diagram $D : \cat I \times \cat J \to \catname{Set}$ with $\cat I$ $\lambda$-small and $\cat J$ $\lambda$-filtered, the induced morphism
\[ \colim_j \lim_i D(i,j) \to \lim_i \colim_j D(i,j) \]
is an isomorphism.
\end{Lemma}

\subsection{Presentable objects}
\begin{Definition}[Presentable object]\label{def:presentableobject}
An object $A$ of a category $\cat K$ is called \emph{$\lambda$-presentable} if its covariant Hom-functor 
\[ \hom(A,-) : \cat K \to \catname{Set} \]
preserves $\lambda$-filtered colimits.
\end{Definition}
Spelled out, this means that every morphism $A \xrightarrow{f} C = \colim_i D_i$ into a $\lambda$-filtered colimit already factors through one of the coprojections
\[ f = A \xrightarrow{f_i} D_i \xrightarrow{c_i} C \]
and the factorisation is essentially unique, so if \[ f = c_i\cdot f_i = c_j \cdot f_j, \]
there is an index $k$ in the diagram and maps $i \to k, j \to k$ such that
\[ D(i \to k)\cdot f_i = D(j \to k) \cdot f_j. \]

Note that every $\lambda$-presentable object is $\lambda'$-presentable for $\lambda' \geq \lambda$. In the case $\lambda=\aleph_0$, we simply say \emph{finitely presentable}. \\

\begin{Lemma}\label{prop:smallcolim} A $\lambda$-small colimit of $\lambda$-presentable objects is again $\lambda$-presentable.
\end{Lemma}
\begin{Proof}
Take a $\lambda$-small diagram $(A_i)$ of $\lambda$-presentable objects, and a $\lambda$-filtered diagram $(D_j)$. By the property \ref{prop:smallvsfiltered}, we get
\begin{align*}
\hom(\colim_i A_i, \colim_j D_j) &\cong \lim_i \hom(A_i, \colim_j D_j) \\
&\cong \lim_i \colim_j \hom(A_i, D_j) \\
&\cong \colim_j \lim_i \hom(A_i, D_j) \\
&\cong \colim_j \hom(\colim_i A_i, D_j),
\end{align*}
therefore $\hom(\colim_i A_i, -)$ preserves $\lambda$-filtered colimits as claimed.
\end{Proof}

\begin{Example}
In $\catname{Set}$, the singleton set $1$ is finitely presentable as $\hom(1,X) \cong X$ for every set, therefore
\[ \hom(1,\colim_i A_i) \cong \colim_i \hom(1,A_i) \]
for every diagram $(A_i)$. By \ref{prop:smallcolim}, a nonempty set $A$ is $\lambda$-presentable iff $|A| < \lambda$.
\end{Example}

The definition of presentable objects really captures what the name suggests: 

\begin{Proposition}\label{prop:varietypresentable} In a variety, an object $A$ is $\lambda$-presentable iff it has a presentation with less than $\lambda$ generators and relations in the usual algebraic sense.
\end{Proposition}
\textit{Idea: } Take $A$ with presentation on less than $\lambda$ generators and relations and take map $f : A \to C = \colim_i D_i$. Then all generators have to map set-theoretically into some $D_i$. Take a cocone over these, so all generators map into a single $D$, but the relations don't have to hold in $D$ yet, so we can't extend to a homomorphism of algebras. However the relations hold in $C$, so each relation has to start holding after some $D \to D_j$. Take a cocone over these to some $D^*$, now we can factor $f$ through it. \\

Conversely, let $A$ be $\lambda$-presentable.
\begin{enumerate}
\item \label{item:varietygen} $A$ is generated by less than $\lambda$ generators. Note that $A$ is the $\lambda$-directed union of its subalgebras with less than $\lambda$ generators, ordered by inclusion. The identity $A \xrightarrow{\id} A$ factors through one of these subalgebras as $A \to A_0 \to A$, but $A_0 \to A$ is an inclusion, thus we have an isomorphism. Denote the set of generators by $X$ where $|X| < \lambda$
\item Let \[ E = \{ (t_1 = t_2) \,|\, A \models (t_1 = t_2) \} \] be the set of equations in the generators $X$ that hold in $A$. $A$ is the $\lambda$-directed colimit of the diagram
\[ D = \{ \langle X|E_0\rangle : E_0 \subseteq E, |E_0| < \lambda \} \]
where the maps $\langle X|E_i\rangle \xrightarrow{\pi_{ij}} \langle X|E_j\rangle$ are the canonical projections corresponding to the inclusions $E_i \subseteq E_j$. Again we get a factorisation 
\[ A \xrightarrow{u} \langle X|E_0 \rangle \xrightarrow{\kappa_0} A \]
of the identity. Note that by \ref{item:varietygen}, $\langle X|E_0 \rangle$ is itself $\lambda$-presentable object and we have two factorisations $\kappa_0 = \kappa_0 \cdot u\kappa_0$ of a map into $A$. By the essential uniqueness in \ref{def:presentableobject}, we get a factorisation of $\kappa_0$ through some $\langle X|E_1 \rangle \xrightarrow{\kappa_1} A$ satisfying $\pi u \kappa_0 = \pi$. Now, $\kappa_1$ is an isomorphism as
\begin{align*}
\kappa_1 (\pi u) &= \kappa_0 u = \id_A \\
(\pi u) \kappa_1 \cdot \pi &= \pi u \kappa_0 = \id_A \cdot \pi
\end{align*}
thus $(\pi u) \kappa_1 = \id_A$ as $\pi$ is an epimorphism.
\end{enumerate}

\begin{Example}
The group $\mathbb Z$ is finitely presentable and by Example \ref{ex:Q}, $\mathbb Q$ is a $\aleph_1$-small colimit of copies of $\mathbb Z$, thus it is $\aleph_1$-presentable by Proposition \ref{prop:smallcolim}; note that we can indeed read off a presentation from the diagram as
\[ \mathbb Q \cong \left\langle x_n : n \in \mathbb N\,|\,x_n = k \cdot x_{nk} \right \rangle\]
\end{Example}

% Remove this?!
\begin{Example}
An element $c$ of a lattice $L$ is finitely presentable iff it is a \emph{compact element}, i.e. whenever
\[ c \leq \bigvee_i d_i \]
for a directed join, we have $c \leq d_i$ for some $i$. This generalises the topological notion of a compact set, where we take $L$ to be the lattice of open subsets.
\end{Example}


\subsection{Locally presentable and accessible categories}

%% Coole einleitende Sätze!

\begin{Definition}[Locally presentable category]
A category $\cat K$ is \emph{locally $\lambda$-presentable} if
\begin{enumerate}
\item $\cat K$ is cocomplete \label{item:cocomplete}
\item there is a set $\mathcal A$ of $\lambda$-presentable objects such that every object of $\cat K$ is a $\lambda$-filtered colimit of objects of $\mathcal A$.
\end{enumerate}
\end{Definition}

Accessibility is a weakening on condition \ref{item:cocomplete}:

\begin{Definition}[Accessible category]
A category $\cat K$ is \emph{$\lambda$-accessible} if
\begin{enumerate}
\item $\cat K$ is has $\lambda$-filtered colimits
\item there is a set $\mathcal A$ of $\lambda$-presentable objects such that every object of $\cat K$ is a $\lambda$-filtered colimit of objects of $\mathcal A$.
\end{enumerate}
\end{Definition}

We say a category is \emph{accessible} (\emph{locally presentable}) if it is $\lambda$-accessible (locally $\lambda$-presentable) for some $\lambda$. Again, for $\lambda = \aleph_0$, we say locally finitely presentable and finitely accessible. 

\begin{Example}
The category $\catname{Set}$ is locally finitely presentable, as every set is $\aleph_0$-directed colimit of its finite subsets by inclusion.
\end{Example}

Recall that in a category $\cat K$, a set $G$ of objects is called a \emph{strong generator} if morphisms out of $G$-objects can separate morphisms and proper subobjects in $\cat K$. If $\cat K$ is $\lambda$-accessible, the set $\mathcal A$ forms a strong generator for $\cat K$. For locally presentable categories, we have a converse that allows for a simpler definition:

\begin{Lemma}\label{lemma:stronggen}
A cocomplete category $\mathcal K$ is locally $\lambda$-presentable iff it has a strong generator of $\lambda$-presentable objects. 
\end{Lemma}

As an immediate corollary, we get that every locally $\lambda$-presentable category is also locally $\lambda'$-presentable for $\lambda' \geq \lambda$. The same strong generator will do! This situation stands in interesting contrast with accessible categories, where we only get the following: For every regular cardinal $\lambda$ there are arbitrarily large regular cardinals $\mu \geq \lambda$ such that every $\lambda$-accessible category is $\mu$-accessible. \\

Using \ref{lemma:stronggen} we can now give more examples

\begin{Example}\
\begin{enumerate}
\item Every variety is locally finitely presentable by the strong generator $\{ F(x) \}$ on the one-generator free algebra.

\item The category $\catname{Pos}$ of posets is locally finitely presentable, as it is cocomplete and $\{\mathbf 2\}$ is a strong generator. 

\item An infinitary ($\lambda$-ary) variety allows operations to have infinite arities $< \lambda$. For example, the theory of semilattices with countable suprema can be given as an equational theory with an operation
\[ \bigvee(x_1, x_2, \ldots) \]
forming an $\aleph_1$-ary variety. Analogously to finitary case, $\lambda$-ary varieties are locally $\lambda$-presentable categories. The only difference is that, while infinitary varieties are cocomplete, $\kappa$-filtered colimits are generally only created by the forgetful functor if $\kappa \geq \lambda$.

\item The category $\catname{Fld}$ of fields is finitely accessible. Even though the category is not cocomplete, it has \emph{filtered} colimits created by the forgetful functor to $\catname{Set}$. \\

Note that we captured the algebraic property that $F$ is finitely presentable as $\hom(F,-)$ preserving filtered colimits. In the same way, the statement that $F$ is finitely \emph{generated} translates into $\hom(F,-)$ perserving filtered colimits of monomorphisms. As every morphism in $\catname{Fld}$ is a monomorphism, this is no condition and the two notions coincide. Note that every field is the directed colimit of its finitely generated subfields by inclusion. \\

It remains to show that the class of finitely generated fields has a set of representatives: If $F$ is finitely generated, we have extensions $F/L/F_p$ where $F_p$ is the prime field of $F$, $L=F_p(\alpha_1,\ldots,\alpha_n)$ is purely transcendental and $F/L$ is finitely generated and algebraic, thus finite and the ring homomorphism
\[ L[x_1,\ldots,x_m] \to F \]
is surjective. Therefore the finitely generated fields are up to isomorphism quotients of the rings $F_p(\alpha_1, \ldots,\alpha_n)[x_1,\ldots,x_m]$ for the prime fields $F_0 = \mathbb Q$ and $F_p = \mathbb F_p$ for $p$ prime.
\end{enumerate}
\end{Example}

Another important example is the following
\begin{Example}
The functor category $\fun{\cat A}{\catname{Set}}$ is locally finitely presentable for every small category $\cat A$. We'll show that representable functors are finitely presentable objects. Let $yA = \hom(A,-)$ then $\hom(yA,-)$ actually preserves \emph{all} colimits as 
\[ \hom(yA, \colim_i F_i) \cong (\colim_i F_i)(A) \cong \colim_i F_i(A) \cong \colim_i \hom(yA, F_i). \]
By the Yoneda lemma, representables form generator of the functor category. To see that this generator is strong, note that the monomorphisms $F \xrightarrow{\alpha} G$ are pointwise-injective natural transformations because $\catname{Set}$ has pullbacks. Thus a proper subobject $\alpha$ leads to some element $x \in GA \setminus \alpha_A(FA)$ and the transformation $yA \xrightarrow{x} G$ doesn't factor through $F$. \\

\textbf{Remark: } We'll see in [??] that $\fun{\cat A}{\catname{Set}}$ is equivalent to a variety, so local presentability is no surprise.
\end{Example}

 \pagebreak
\section{Sketches}
\label{sec:sketches}

\subsection{Definition of sketches}

Sketches have been introduces by Ehresmann. % More to say \\

\begin{Definition}[Sketch]
A sketch is tuple $\mathbb S = (\cat A, \dist L, \dist C)$ where $\cat A$ is a small category, $\dist L$ is a set of cones in $\cat A$ and $\dist C$ a set of cocones in $\cat A$.
\end{Definition}
% Todo better explanation
Note that for convenience, we treat our cones not as natural transformations but as diagrams $D : \cat I \to \cat A$ of a special form. $\cat I$ consists of a distinguished apex $\cat I^+$ and a basis diagram $\cat I_0$ with unique morphisms $(\cat I^+ \to B)_{B \in \cat I_0}$ and no morphisms from the basis into $\cat I^+$; analogously for cocones. \\

Sketches form a category $\catname{Sk}$ by letting morphisms $(\cat A, \dist L, \dist C) \to (\cat A', \dist{L'}, \dist{C'})$ be functors $F : \cat A \to \cat A'$ that respect the distinguished cones and cocones, i.e. for all $D \in \dist L$, we have $F \circ D \in \dist{L'}$ and similarly $F \circ D \in \dist{C'}$ for all $D \in \dist C$. \\

\textbf{Remark: } The way we will actually write down sketches to present a theory will look slightly different: Instead of the category $\cat A$, we will usually specify an underlying graph\footnote{directed multigraph} $G$, together with a set $D$ of commutativity conditions between pairs of paths of the same starting- and endpoint. We will then prescribe the cones and cocones accordingly. Note that we get back to our original definition by taking $\cat A$ to be the category freely presented by $G$ modulo the congruence given by $D$. 

\begin{Definition}[Model of a sketch]
Given a sketch $\mathbb S = (\cat A, \dist L, \dist C)$ and a category $\cat K$, a \emph{model} of $\mathbb S$ in $\cat K$ is a functor
\[ F : \cat A \to \cat K \]
that turns all distinguished cones into limit cones and cocones into colimit cones in $\cat K$, i.e. \begin{itemize}
\item for all $D \in \dist L$, $F \circ D$ is limiting,
\item for all $D \in \dist C$, $F \circ D$ is colimiting.
\end{itemize}
A \emph{morphism of models} is just a natural transformation between the functors, making the category \[ \mod(\mathbb S, \cat K) \]
of $\mathbb S$-models in $\cat K$ a full subcategory of $\fun{\cat A}{\cat K}$. We write
\[ \mod(\mathbb S) := \mod(\mathbb S, \catname{Set}) \]
for the category of set-valued models of $\mathbb S$. Set-valued models will in fact be so important that we call a category \emph{sketchable} if it is equivalent to $\mod(\mathbb S)$ for a sketch $\mathbb S$.
\end{Definition}

% Normal sketches??

\textbf{Remark: } By ignoring set-theoretic restrictions, we could think of taking model categories as a functor
\begin{equation}
\label{eq:modfunctor} \mod(-, \cat K) : \catname{Sk}^\op \to \catname{CAT}
\end{equation}
Isomorphism of sketches is a stronger notion than equivalence of their categories of models, though. Note that we can assign to every category $\cat K$ an ``underlying large sketch'' 
\[ U(\cat K) = (\cat K, \dist L, \dist C) \]
where $\dist L$ collects all limit cones, $\dist C$ all colimit cones in $\cat K$. Then a model $F : \mathbb S \to \cat K$ is nothing but a sketch morphism $F : \mathbb S \to U(\cat K)$ and (\ref{eq:modfunctor}) is just given by composition of morphisms. \\

We can now give names to the the syntactic (rather: geometric) properties of sketches.
\begin{Definition}
We call $\mathbb S = (\cat A, \dist L, \dist C)$
\begin{enumerate}
\item a \emph{limit sketch} if $\dist C$ is empty
\item a \emph{product sketch} if $\dist C$ is empty and all cones in $\dist L$ are \emph{discrete}, i.e. an apex connected to a discrete diagram
\item a \emph{finite product sketch} if $\dist C$ is empty and all cones in $\dist L$ are discrete and finite
\item a \emph{sum sketch} if all cocones are discrete
\item \emph{$\lambda$-small} if all cones and cocones have underlying $\lambda$-small diagrams
\end{enumerate}
We furthermore say \emph{mixed sketch} to stress that both $\dist L$ and $\dist C$ are nonempty.
\end{Definition}

\subsection{Limit sketches}

In this section, we will see examples on how to sketch theories for known some categories. Note that we have to be careful to distinguish between the vertices and edges of the sketch, and their ``meaning'' as objects and morphisms in a model. \\

We have already seen the sketch for binary algebras in the introduction. Let's extend it in the following way:

\begin{Example}[Unital algebras]\label{ex:unitality}
We want to sketch the variety $\cat V$ with a neutral element $e$ such that
\[ m(x, e) = x. \]
We start with a graph on vertices $1, a, a^2$ and edges as described
\[
\xymatrix{
  & a^2 \ar[rd]^{\pi_2} \ar[d]_{m} \ar[ld]_{\pi_1} & & 1 \ar[d]_{e} \\
a & a & a & a
}\]
Note that even though $1$ and $a^2$ are suggestively named, these are just symbols at the moment. We will make them get their intended meaning in the models of the sketch. To make $a^2$ become the actual product of $a$, we needed to add
\[
\xymatrix{
  & p \ar[rd]^{\pi_2} \ar[ld]_{\pi_1} & \\
a & & a
}\]
to $\dist L$. Furthermore, we add to $\dist L$ the empty cone with apex $1$, making it a terminal object. We add an edge
\[ (\id,e) : a \to a^2 \]
and now need to pin down its meaning. Let's introduce another edge $! : a \to 1$. As $1$ will be a terminal object, its meaning is already uniquely defined\footnote{and I will by slight abuse of notation write $!$ for all edges into terminal objects}. Now we can add commutativity relations
\[
\xymatrix{
  & & a \ar[d]^{(\id,e)} \ar@/_/[lldd]^{\id} \ar@/^/[rd]^{!}  & & \\
  & & a^2 \ar[lld]^{\pi_1} \ar[rrd]_{\pi_2} & 1 \ar@/^/[rd]^{e} & \\
a & & & & a
}\]
As $a^2$ will be a limit, the morphism $(\id,e)$ into it will be determined through its projections and thus be uniquely defined as well. Lastly, we can add a commutativity relation\footnote{note that the identity morphism $\id_a$ is uniquely given through the empty path $()$ from $a$ to $a$}
\[ m \cdot (\id,e) = \id_a. \]
Note that natural transformations between models will precisely correspond to the algebraic notion of homomorphisms. By this construction, we have a finite product sketch $\mathbb S$ with two cones such that
\[ \mod(\mathbb S) \simeq \cat V. \]
\end{Example}

\begin{Example}[Groups]
I will just sketch how to sketch the category of groups. For the associative law, we construct ourselves double and triple products
\[
\xymatrix{
  & a^2 \ar[ld]_{\pi_1} \ar[rd]^{\pi_2} &    & & a^3 \ar[ld]_{p_1} \ar[d]^{p_2} \ar[rd]^{p_3} & \\
a &     & a & a & a & a
}\]
with the corresponding discrete limit cones. We want to write down the commutativity
\[ m \circ (m \circ (p_1, p_2), p_3) = m \circ (p_1, m \circ (p_2,p_3)) \]
between paths $a^3 \to a$. All the intermediate edges have to be added to the sketch and their meaning defined uniquely via their projections, e.g by the following commuting diagram
\[
\xymatrix{
& & a^3 \ar[d]^{(p_1,p_2)} \ar@/_2pc/[lldd]_{p_1} \ar@/^2pc/[rrdd]^{p_2} & \\
& & a^2 \ar[lld]_{\pi_1} \ar[rrd]^{\pi_2} & \\
a & & & & a
}\]
Unitality was discussed in \ref{ex:unitality}, and the inversion operation amounts to an edge $\iota : a \to a$ such that \[ m \cdot (\iota, \id) = e \cdot ! = m \cdot (\id, \iota) \]
as paths $a^2 \to a$, where again $!: a^2 \to 1$. 
\end{Example}

This procedure gives an idea on how to turn any set of identities into a finite product sketch. \\

Note that the theory of varieties allows for two straightforward extensions: 

\begin{itemize}
\item An infinitary ($\lambda$-ary) variety allows operations to have infinite arities $< \lambda$, for example the theory of semilattices with countable suprema can be given in terms of an operation
\[ \bigvee(x_1, x_2, \ldots) \]
and equations, forming an $\aleph_1$-ary variety. $\lambda$-ary varieties are locally $\lambda$-presentable categories.

\item A \emph{many-sorted variety} consists of different sorts of underlying sets and the signature of operations are ``typed`` with respect to the sorts. For example, a graph is a two-sorted algebra on sorts $e,v$ with two operations $s, t : e \to v$. 
\end{itemize}

Note than an $S$-sorted set $A$ is just a functor $A : S \to \catname{Set}$ with $S$ viewed a discrete category. Prescribing product cones for $A$ just allows us to declare certain sorts as products of other sorts, allowing us to reduce all operations to unary ones at the cost of introducing more sorts.
 
\begin{Proposition}
Model categories of finite product sketches are precisely the many-sorted varieties.
\end{Proposition}
\begin{Proof}
Given a finite product sketch $\mathbb S = (\cat A, \dist L)$, we need to define a variety equivalent to its models: Start by defining a signature with sorts $\obj(\cat A)$ and add unary operation symbols $f : s \to t$ for every morphism $f : s \to t$ in $\cat A$. \\

Add equations
\[ \id_r(x) = x \]
and
\[ f(g(x)) = (f \circ g)(x)  \]
for all morphisms $g : r \to s, f : s \to t$ in $\cat A$ where $x$ is a variable of sort $r$. We can now think of algebras as functors $F : \cat A \to \catname{Set}$. The limit conditions require us to exhibit certain sorts as the products of others. Thus for each cone $C = (s \xrightarrow{\pi_i} s_i) \in \dist L$, add a ``packing operation'' operation $\hat c$ of arity
\[ \hat c : s_1 \times \cdots \times s_n \to s \]
that is inverse to the projections, i.e.
\begin{align*}
\hat c(\pi_1(x), \ldots, \pi_n(x)) &= x \\
\pi_i(\hat c(x_1, \ldots, x_n)) &= x_i
\end{align*}
for all $i=1,\ldots,n$ where $x : s, x_i : s_i$.
Every functor $F : \cat A \to \catname{Set}$ that satisfies these equations will induce an isomorphism
\[ F(s) \xrightarrow{(F\pi_1,\ldots, F\pi_n)} F(s_1) \times \cdots \times F(s_n) \]
in $\catname{Set}$, i.e. be a model of the sketch.
\end{Proof}

By analogy, product sketches correspond to infinitary varieties.

\begin{Example}[Torsion-free groups]
Let's take a quasivariety like torsion-free groups and look at one implicational axiom like
\[ x + x = 0 \quad \Rightarrow \quad x = 0. \]
We can sketch this using equalisers. Take again vertices $1,a,a^2$ as before and $0_a : 1 \to a$. We can describe the set
\[ e = \{ x : x + x = 0 \} \]
as an equalizer cone
\[
\xymatrix{
& e \ar@/_1pc/[ld]_{j} \ar@/^1pc/[rd] & \\
a \ar@/^/[rr]^{x + x} \ar@/_/[rr]_{0_a \circ !} & & a
}\]
where \[ x + x = + \circ (\id,\id). \]
Now we add commutativity condition expressing $e = \{0\}$ by \[ j = 0_a\cdot !. \]
\end{Example}

\begin{Example}[Simple graphs]
A simple graph is a graph, i.e. a two-sorted algebra on sorts $v,e$ with operations
\[ s, t : e \to v \]
where there is at most one edge between two vertices, leading to the implicational condition
\begin{equation}\label{eq:graph} s(x) = s(y),\, t(x) = t(y) \quad \Rightarrow \quad x = y \end{equation}
The right hand side can be expressed as a pullback of the map \[ (s,t) : e \to v^. \]
In fact the condition (\ref{eq:graph}) just says that $(s,t)$ is injective, which is a pullback condition
\[
\xymatrix{
 & e \ar[ld]_{\id} \ar[rd]^{id} & \\
e \ar[rd]_{(s,t)} &  & e \ar[ld]^{(s,t)} \\
 & v^2 &
}\]
\end{Example}

% Just call things essentially algebraic

\textbf{Remark: } Recall that we could think of models of product sketches as many-sorted algebras where certain sorts encoded products of other sorts. Prescribing more general cones determines certain sorts as limits of other sorts, thus subsets of a product subject to equations. For example, the pullback above means that we have a ``partial packing operation'' $\hat c : e \times e \to e$ which is only defined on \emph{compatible inputs}, i.e.
\[ \hat c(x,y) \text{ defined iff } (s,t)(x) = (s,t)(y). \]
This leads to the characterisation of limit sketches as so called \emph{essentially algebraic theories}, which are equational theories with total and partial operations, where the domains of the partial operations are given by equations in the total ones. \\

In view of the sketchability theorem, we can identify the classes of locally presentable and essentially algebraic categories. \\

Note that models of a finite limit sketch can be axiomatised in propositional logic. This will contrast finite mixed sketches. \\ % TODO

We note that in essentially algebraic class can still be computed on the level of the underlying sets, sort-by-sort. This translates back to the following observation:

\begin{Proposition}
For every limit sketch $\mathbb S = (\cat A, \dist L)$, $\mod(S)$ is closed under limits in $\fun{\cat A}{\catname{Set}}$.
\end{Proposition}
\begin{Proof}
Limits are computed pointwise and commute with limits. % Annotate proof
\end{Proof}

\subsection{Mixed sketches}
The introduction of cocones allows for more interesting categories

\begin{Example}[Fields]
We can almost sketch fields as varieties on the signature $\{0_k,1_k,+,\times,(-)^{-1}\}$ apart from the pathological axiom
\[ x = 0_k \,\vee\,x \times x^{-1} = 1_k. \] 
% More work to be done? => Johnstone
Introduce a sketch as usual on vertices $1,k,k^2,k^3$ but also add a symbol $k^*$ that we want to represent the nonzero elements of the field. This of course means
\[
\xymatrix{
& k & \\
k^* \ar[ru]^{j} & & 1 \ar[lu]_{0_k}
}\]
has to become a coproduct, so add it to $\dist C$. Then define a map 
\[ (x,x^{-1}) : k^* \to k^2 \]
in the obvious way and add a condition on paths $k^* \to k$
\[ \times \circ (x,x^{-1}) = 1_k \circ !. \]
\end{Example}

The category of fields can be neither complete nor cocomplete, as there are only homomorphisms between fields of the same characteristic. However, it has connected limits, for example the intersection of two subfields is again a subfield. This can be traced to the fact that the sketch we gave for $\catname{Fld}$ is a sum sketch. % Show accessibiliity

\begin{Proposition}
For every sum sketch $\mathbb S = (\cat A, \dist L, \dist C)$, $\mod(S)$ is closed under connected limits in $\fun{\cat A}{\catname{Set}}$.
\end{Proposition}
\begin{Proof}
Coproducts commute with connected limits in $\catname{Set}$.
\end{Proof}

This is not the case any more for arbitrary cocones in $\dist C$. Compare with our sketch for connected graphs in the introduction. Intersections of connected subgraphs needn't be connected. \\

Worse, while models limit sketches with all finite cones could be described in first order logic, this is no longer true for mixed sketches. Connected graphs are not first-order axiomatisable by the compactness theorem. % Coequaliser description

% Linear orders machen \pagebreak
\section{From categories to sketches}
\label{seq:catstoskeches}

In this section, we will see how to embed accessible categories into the functor category $\fun{\cat A}{\catname{Set}}$ for some small category $\cat A$. More precisely, $\lambda$-accessible categories will be precisely those equivalent to the subcategories $\flat_\lambda(\cat A)$ of so-called \emph{$\lambda$-flat} functors $\cat A \to \catname{Set}$. \\

We will then show that the condition of flatness can be captured by a sketch.

\subsection{Canonical colimits}

Let $\cat K$ be a $\lambda$-accessible category. We know that every object is $\lambda$-filtered colimit of some set $\cat A$ of $\lambda$-presentable objects. In fact, we can pick a canonical such set generating set by considering all $\lambda$-presentable objects.

\begin{Proposition}
For every $\lambda$-accessible category $\cat K$, the full subcategory of all $\lambda$-presentable objects is essentially small. 
\end{Proposition}
\begin{Proof}
Take any $\lambda$-presentable object $B$ and write it as a $\lambda$-filtered colimit $B \xrightarrow{\sim} \colim_i A_i$ of objects in $\cat A$. The isomorphism factors through one of the $A_i$, exhibiting $B$ as a retract $B \xrightarrow{j} A_i \xrightarrow{r} B$ of $A_i$. Note that $jr$ is an idempotent endomorphism of $A_i$, and retracts that induce the same endomorphism are isomorphic. Thus we have an class injection
\[ \{ \text{retracts of } A_i \}/\text{isomorphism} \hookrightarrow \hom(A_i,A_i) \]
where the codomain is a set (as $\cat K$ is locally small). Now take the union of the sets of representatives for each $A_i \in \cat A$.
\end{Proof}

Let's write $\cat K_\lambda$ for any set of representatives of the $\lambda$-presentable objects up to isomorphism. Given an object $K$ and small full subcategory $\cat A$ of $\cat K$, the canonical diagram with respect to $\cat A$ is the diagram of all morphisms of $\cat A$-objects into $K$, formally given by the forgetful functor
\[ U : \cat A/K \to \cat K. \]
We say that $K$ is the canonical colimit of $\cat A$-objects if the canonical cocone
\[ (A \xrightarrow{f} K) \xrightarrow{f} K \]
is colimiting. We call $\cat A$ \emph{dense} in $\cat K$ if every object of $\cat K$ is the canonical colimit of $\cat A$-objects.

\begin{Proposition}\label{prop:presdense}
For every $\lambda$-accessible category $\cat K$ and $\cat A = K_\lambda$, every object is canonical colimit of $\cat A$-objects and its canonical diagram is $\lambda$-filtered.
\end{Proposition}
\begin{Proof}
Let $K$ be an object of $\cat K$. We know that it is $\lambda$-filtered colimit of some $\lambda$-presentable objects $(B_i)$. The digram $B$ sits inside the canonical diagram $A : \cat A/K \to \cat K$ like this
\[
\xymatrix{
  & & K & & \\
& B_i \ar[rr] \ar[ru] & & B_j \ar[lu] & \\
A_r \ar[rrrr] \ar@/^2pc/[rruu] \ar@{.>}[ru]_{\exists} & & & & A_s \ar@{.>}[lu]^{\exists} \ar@/_2pc/[lluu]
}\]
Because the $A_r$ are all locally $\lambda$-presentable, we get the dashed factorisations through some of the $B_i$. The larger diagram therefore factors through $B$ in a compatible way, in the terminology of [...AR], $B$ is cofinal in $A$, and the diagrams have the same colimit. \\

For $\lambda$-filteredness, note that each $\lambda$-small subcategory of $\cat A/K$ has factorisations through less than $\lambda$ $B$-objects, but by $\lambda$-directedness of $B$, everything then factors through a single $B$-object, so the subcategory has a cocone over it. 
\end{Proof}

% Use actual confinal diag B in A/K, index stuff with U_f

Given a small full subcategory $\cat A$ of $\cat K$, we have a functor 
\[ E : \cat K \to \fun{\cat A^\op}{\catname{Set}} \]
that sends an object $K$ to the domain restriction of $\hom(-, K)$ to $\cat A^\op$. We can restate the properties of $\cat A$ in terms of $E$ in the following way:

\begin{Proposition}\
\label{prop:canonicalproperties}
\begin{enumerate}
\item $E$ is fully faithful if $\cat A$ is dense in $\cat K$. \label{item:fullyfaithful}
\item $E$ preserves $\lambda$-filtered colimits if all objects of $\cat A$ are $\lambda$-presentable. \label{item:limits}
\end{enumerate}
\end{Proposition}
% Make stuff iff again

\begin{Proof}
\ref{item:fullyfaithful}: Let $U : \cat A/K \to \cat K$ be the canonical diagram; we claim that a cocone $(U_f \to K')$ corresponds to a natural transformation $EK \to EK'$. Indeed such a cocone has for each $A \xrightarrow{f} K$ a morphism $A \xrightarrow{\hat f} K'$, such that for all commutative triangles
\[
\xymatrix{
A \ar[rr]^{h} \ar[rd]_{f} & & A' \ar[ld]^{g} \\
& K
}\]
we have $\hat f = \hat g \cdot h$. This amounts precisely to a family of maps $(\widehat{-})_A: \hom(A,K) \to \hom(A,K')$, natural in $A$. We get the desired equation
\[ \hom(EK,EK') \cong \textrm{Cocones}(U, K') \cong \hom(K,K') \] 
if and only if the cocone into $K$ is universal, thus $\cat A$ is dense. \\

For \ref{item:limits}, note that colimits of functors are computed pointwise. For all $\lambda$-filtered colimits, we have
\[ E(\colim_i K_i)(A) \cong \hom(A,\colim_i K_i) \cong \colim_i \hom(A,K_i) \]
as $A$ is $\lambda$-presentable, therefore $E(\colim_i K_i) \cong \colim_i EK_i$.
\end{Proof}

Together with Proposition \ref{prop:presdense}, we see that for $\lambda$-accessible categories $\cat K$, we get a canonical embedding
\[ E : \cat K \to \fun{\cat A}{\catname{Set}} \text{ where } \cat A = \cat K^\op_\lambda. \]

\subsection{Flat functors}
We want to characterise precisely which subcategories of $\fun{\cat A}{\catname{Set}}$ occur via embeddings of accessible categories.

\begin{Definition}\
We call a functor $F : \cat A \to \catname{Set}$
\begin{enumerate}
\item \emph{$\lambda$-flat} if it is $\lambda$-filtered colimit of representable functors
\item \emph{$\lambda$-continuous} if it preserves $\lambda$-small limits
\end{enumerate}
and we denote the full subcategories of $\fun {\cat A}{\catname{Set}}$ of $\lambda$-flat and $\lambda$-continuous functors as 
\[ \flat_\lambda(\cat A), \cont_\lambda(\cat A) \text{ respectively}. \]
\end{Definition}

We'll prove several properties of flat functors, that will be discussed in terms of Kan extensions. \\

Recall that for every small category $\cat A$, the Yoneda embedding $y : \cat A \to \pre{\cat A}$ has the universal property of being the free cocompletion for $\cat A$. Any functor $F : \cat A \to \cat C$ into a cocomplete category $\cat C$ extends uniquely to a cocontinuous functor $F_!: \pre{\cat A} \to \cat C$. More precisely, there is an adjunction $F_! \dashv F^*$ with $F_! \circ y \cong F$ of the form
\[
\xymatrix{
\pre{\cat A} \ar@/^/[rr]^{F_!} & & \cat C \ar@/^/[ll]^{F^*} \\
& \cat A \ar[lu]^{y} \ar[ru]_{F}
}\]

We call $F_!$ the \emph{left Kan extension} or \emph{Yoneda extension} of $F$. The uniqueness of $F_!$ comes from the fact that every presheaf $P$ in $\pre{\cat A}$ is the canonical colimit of representables, which allows us to extend $F$ to $F_!$ up to isomorphism by cocontinuity
\[ F_!(P) \cong F_!(\colim_i yA_i) \cong \colim_i F_!(yA_i) \cong \colim_i FA_i. \]
Note that by the claimed adjunction, we get
\[ \hom(F_!(yA),B) \cong \hom(yA,F^*(B)) \cong F^*(B)(A), \]
so $F^*(B)$ has to be defined as $\hom(F(-),B)$. \\

We'll give the canonical diagram of representables an explicit description: For a presheaf $P : \cat A^\op \to \catname{Set}$, we define the ``category of elements'' $\el(P)$ be the category $(y \downarrow P)$. By the Yoneda lemma, an element $x : yA \to P$ of $\el(P)$ is equivalent to a pair $(A,x), x \in PA$ where morphisms $(A,x) \xrightarrow{h} (B,y)$ are morphisms $A \xrightarrow{h} B$ satisfying $Ph(y) = x$. We have a forgetful functor $\pi : \el(P) \to \cat A$ and $P \cong \colim y\circ \pi$. We denote the contravariant Yoneda embedding by $y' : \cat A^\op \to \fun{\cat A}{\catname{Set}}$.

\begin{Proposition}\label{prop:flat}
For a functor $F : \cat A \to \catname{Set}$, the following are equivalent
\begin{enumerate}
\item $F$ is $\lambda$-flat \label{item:fflat}
\item $F_!$ preserves $\lambda$-small limits \label{item:kancont}
\item $F_!$ preserves $\lambda$-small limits of representables \label{item:kanrepcont}
\item $\el(F)$ is $\lambda$-filtered (considering $F$ as a presheaf in $\pre{\cat A^\op}$) \label{item:elfiltered}
\end{enumerate}
\end{Proposition}
\begin{Proof}
\ref{item:elfiltered} $\Rightarrow$ \ref{item:fflat} is immediate. For \ref{item:fflat} $\Rightarrow$ \ref{item:kancont}, we first note that the Kan extension of a representable functor is again representable. Let $F=\hom(A,-)$, there is the cocontinuous extension \[ \widehat F(P) := P(A) \cong \pre{\cat A}(yA,P), \]
thus by uniqueness $F_! \cong \widehat F$. In particular, $F_!$ preserves all limits. \\

Now let $G = \colim_i F_i$ be $\lambda$-filtered colimit of representables, and $(P_j)$ a $\lambda$-small diagram of presheaves. Because taking the Kan extension is cocontinuous and $\lambda$-small limits commute with $\lambda$-filtered colimits, we get
\begin{align*}
G_!(\lim_j P_j) &\cong \colim_i (F_i)_!(\lim_j P_j) \\
&\cong \colim_i \lim_j (F_i)_!(P_j) \\
&\cong \lim_j  \colim_i (F_i)_!(P_j) \\
&\cong \lim_j G_!(P_j).
\end{align*}

For \ref{item:kanrepcont} $\Rightarrow$ \ref{item:elfiltered}, we take a $\lambda$-small subcategory of $\el(F)$, schematically
\[ (A_i,a_i) \xrightarrow{g_{ij}} (A_j,a_j) \]
and construct a cocone over it. We look at the diagram of presheaves $(yA_i)$ and take its limit $H$ in $\pre{\cat A}$
\[
\xymatrix{
& H \ar[ld]_{h_i} \ar[rd]^{h_j} \\
yA_i & & yA_j \ar[ll]_{yg_{ij}}
}\]
By assumption, applying $F_!$ preserves this limit, so by $F_!(yA_i) \cong FA_i$, we get the following limit diagram in $\catname{Set}$
\[
\xymatrix{
& F_!(H) \ar[ld]_{F_!h_i} \ar[rd]^{F_!h_j} \\
FA_i & & FA_j \ar[ll]_{Fg_{ij}}
}\]
Explicitly, $F_!(H)$ consists of compatible families $(\alpha_i \in FA_i), Fg_{ij}\alpha_j = \alpha_i$. Our family of elements $(a_i)$ is compatible by assumption, so there is an element $h \in F_!(H)$ with $a_i = (F_!h_i)(h)$. \\

By definition of the Kan extension, we can compute $F_!(H)$ via the canonical colimit 
\[ F_!(H) \cong \bigcolim_{yB \to H} FB, \]
so there is morphism $yB \xrightarrow{f} H$ and $b \in B$ such that $(F_!f)(b) = h$. The morphisms $h_i\circ f : yB \to yA_i$ now satisfy
\[ F_!(h_i \circ f)(b) = h_i \]
and by the Yoneda lemma get induced by morphisms $B \to A_i$. This means the diagram
\[
\xymatrix{
& (B,b) & \\
(A_i,a_i) \ar[ru]^{h_i \circ f} \ar[rr]^{g_{ij}} & & (A_j,a_j) \ar[lu]_{h_j \circ f} 
}\]
is the desired cocone in $\el(F)$.
\end{Proof}

\begin{Corollary}
$\flat_\lambda(\cat A)$ is closed unter $\lambda$-filtered colimits in $\fun{\cat A}{\catname{Set}}$.
\end{Corollary}
\begin{Proof}
By the same argument as in \ref{item:fflat} $\Rightarrow$ \ref{item:kancont} for \ref{prop:flat}, a $\lambda$-filtered colimit of functors that satisfy \ref{item:kancont} also satisfies \ref{item:kancont}.
\end{Proof}

\subsection{Embedding theorems}

Our first claim is that the $\lambda$-accessible categories are precisely those equivalent to $\flat_\lambda(\cat A)$ for small categories $\cat A$.

\begin{Proposition}
Let $\cat K$ be $\lambda$-accessible and $\cat A = \cat K^\op_\lambda$. Then the essential image of the canonical embedding consists of the $\lambda$-flat functors, and we have an equivalence
\[ \cat K \simeq \flat_\lambda(\cat A). \]
Conversely, for $\cat A$ small, the category $\flat_\lambda(\cat A)$ is $\lambda$-accessible.
\end{Proposition}
\begin{Proof}
The first statement follows pretty much from the definition of flatness and the properties in \ref{prop:canonicalproperties}; $E$ preserves $\lambda$-filtered colimits, and every object $K$ can be written as one, so
\[ EK \cong E(\colim_i A_i) \cong \colim_i \hom(-,A_i), \]
where the latter is a colimit of representables in $\fun{\cat A^\op}{\catname{Set}}$.
% comment on both directions
Conversely, we have seen that $\flat_\lambda(\cat A)$ has $\lambda$-directed colimts. Furthermore representable functors are $\lambda$-presentable objects and by definition every $\lambda$-flat functor is $\lambda$-filtered colimits of these. So $\flat_\lambda(\cat A)$ is $\lambda$-accessible.
\end{Proof}


% orthogonality small vs. filtered
We can sharpen this characterisation in the locally presentable case.
\begin{Proposition}
Let $\cat K$ be $\lambda$-accessible \emph{and cocomplete}, i.e. locally $\lambda$-presentable, and $\cat A = \cat K^\op_\lambda$. Then the essential image of the canonical embedding restricts to the $\lambda$-continuous functors, so 
\[ \cat K \simeq \flat_\lambda(\cat A) \simeq \cont_\lambda(\cat A). \]
\end{Proposition}
\begin{Proof}
We first show that all functors coming from the canonical embedding are $\lambda$-continuous. The key fact is: All $\lambda$-small limits in $\cat A$ exist and correspond to the $\lambda$-small colimits in $\cat K$, which exist by cocompleteness, and by \ref{prop:smallcolim} already lie in $\cat A$. \\

The functors $EK = \hom(-, K)$ preserve all colimits in $\cat K$, thus in particular $\lambda$-small limits from $\cat A$. \\

Now we let $F$ be $\lambda$-continuous and deduce that it's $\lambda$-flat by showing that $\el(F)$ is $\lambda$-filtered (compare with the filteredness proof in \ref{prop:flat}). Take a $\lambda$-small subcategory of $\el(F)$, schematically
\[ (A_i,a_i) \xrightarrow{g_{ij}} (A_j,a_j). \]
We can take the limit of the $A_i$ in $\cat A$
\[
\xymatrix{
& A \ar[ld]_{f_i} \ar[rd]^{f_j} & \\
A_i & & A_j \ar[ll]_{g_{ij}}
}\]
By assumption, $F$ turns this into a limit in $\catname{Set}$
\[
\xymatrix{
& FA \ar[ld]_{Ff_i} \ar[rd]^{Ff_j} & \\
FA_i & & FA_j \ar[ll]_{Fg_{ij}}
}\]
so $FA$ consists of families $(\alpha_i \in FA_i)$ such that $Fg_{ij}(\alpha_j) = \alpha_i$. The familiy of elements $(a_i)$ is compatible, thus there exists $a \in FA$ with $(Ff_i)(a) = a_i$. This means \[
\xymatrix{
& (A,a) & \\
(A_i,a_i) \ar[ru]^{f_i} \ar[rr]^{g_{ij}} & & (A_j,a_j) \ar[lu]_{f_j}
}\]
is the desired cocone in $\el(F)$.
\end{Proof}

During the proof, it was actually enough to require that $\cat K$ have $\lambda$-small colimits. We can draw several corollaries

\begin{Corollary}
Let $\cat K$ be a $\lambda$-accessible category. Then the following are equivalent
\begin{enumerate}
\item $\cat K$ is complete \label{item:complete}
\item $\cat K$ is cocomplete \label{item:cocomplete}
\item $\cat K$ has $\lambda$-small colimits \label{item:lcocomplete}
\end{enumerate}
\end{Corollary}
\begin{Proof}
By the previous proposition, \ref{item:lcocomplete} implies that we have an equivalence $K \simeq \cont_\lambda(\cat A)$. The latter category is closed under limits in $\fun{\cat A}{\catname{Set}}$, as limits commute with limits; thus it is complete. \\

For \ref{item:complete} $\Rightarrow$ \ref{item:cocomplete}, we note that $\cat K$ now has all limits and $E : \cat K \to \fun{\cat A}{\catname{Set}}$ preserves them. We can use the Adjoint Functor Theorem to show that $E$ has a left adjoint. This exhibits $\cat K$ as a reflective subcategory of a cocomplete category, thus it is cocomplete. The solution set condition is verified in [...]. 
\end{Proof}

From this it follows that every locally presentable category is complete and well-powered, as it has a strong generator and pullbacks. Furthermore

\begin{Corollary}\
Every locally $\lambda$-presentable category is sketchable by a $\lambda$-small limit sketch.
\end{Corollary}
\begin{Proof}
Note that by definition
\[ \cont_\lambda(\cat A) \simeq \mod(\mathbb S) \]
where the sketch $\mathbb S$ collects all $\lambda$-small limit cones in $\cat A$.
\end{Proof}

It remains to show how we can sketch accessible categories. This reduces to the question of sketching $\flat_\lambda(\cat A)$. Note that by \ref{prop:flat}, we have an equivalence between $\lambda$-flat functors $F : \cat A \to \catname{Set}$ and cocontinuous, $\lambda$-continuous functors $F_! : \pre{\cat A} \to \catname{Set}$. This condition is alredy sketchable by a ``large sketch'' on the large category $\pre{\cat A}$. We can reduce it to a small sketch by virtue of \ref{prop:flat}.\ref{item:kanrepcont}. \\


\begin{Theorem}
Every accessible category is sketchable.
\end{Theorem}
\begin{Proof}
Let $A \simeq \cat D_0$ be the full subcategory of $\pre{\cat A}$ of representable presheaves, i.e. the image of the Yoneda embedding. We can construct a sketch $\mathbb S = (\cat D, \dist L, \dist C)$ in the following way:

\begin{enumerate}
\item Let $\dist L$ be a system of representatives of all $\lambda$-small limit cones over $\cat D_0$, which might produce a set of new limits $\cat D'$.
\item Let $\cat D = \cat D_0 \cup \cat D'$
\item Let $\dist C$ be the set of all canonical cocones into the new limits with respect to $\cat D_0$.
\end{enumerate}

Preserving the colimits in $\dist C$ ensures that models $F : \cat D_0 \cup \cat D' \to \catname{Set}$ of $\mathbb S$ can still be extended by cocontinuity to functors $F_! : \pre{\cat A} \to \catname{Set}$. Those then preserve $\lambda$-small limits of representables by $\dist L$. \\

On the other hand, every cocontinuous functor $F_! : \pre{\cat A} \to \catname{Set}$ preserving $\lambda$-small limits induces a model of $\mathbb S$ when restricted to $\cat D$. This gives an equivalence
\[ \mod(\mathbb S) \simeq \cat \flat_\lambda(\cat A). \]
\end{Proof}

 \pagebreak
\section{Categories of models are accessible}
\label{sec:sketchesaccessible}

The goal of this section is the remaining direction in the sketchability theorem: To show that $\mod(\mathbb S)$ is accessible for every sketch $\mathbb S$. The structure of the argument follows \cite[\nopp~3.3.5]{MakkaiPare} and \cite[\nopp~D2.3.11]{elephant}. \\

As an outline, let us examine the following crude proof that the category of groups is $\aleph_1$-accessible. For every infinite regular cardinal $\kappa$, every group $G$ of cardinality less than $\kappa$ is $\kappa$-presentable by \ref{prop:varietypresentable}, because we have a presentation of $G$ by its own elements and multiplication table
\[ G \cong \langle e_g : g \in G \,|\, e_g e_h = e_{gh} \rangle. \]

Every group is the colimit (union) of its subgroups of cardinality less than $\kappa$, because the subgroups generated by each singleton element are at most countable and thus among them. So representatives of these groups up to isomorphism would be a candidate generating set in the definition of accessibility. Write $\sub_\kappa(G)$ for the poset of subgroups of cardinality less than $\kappa$. It remains to find out when this poset is actually $\kappa$-directed. \\

This is certainly not the case for $\kappa=\aleph_0$. Consider the infinite group $G=\langle a, b \,|\, a^2 = b^2 = 1 \rangle$ and the poset of its \emph{finite} subgroups. Then the subgroups $\langle a \rangle$ and $\langle b \rangle$ do not have an upper bound. 
However for $\kappa > \aleph_0$, if $\{H_i\}$ is a family of less than $\kappa$ subgroups of cardinality less than $\kappa$, the set \[ A = \bigcup_i H_i \]
is still smaller than $\kappa$ by regularity. The subgroup $\langle A \rangle$ generated by it has at most as many elements as there are words in $A \cup A^{-1}$, thus
\begin{equation} |\langle A \rangle| \leq 2 \cdot |A|\cdot \aleph_0 < \kappa. \label{eq:skolembound} \end{equation}
Therefore $\sub_\kappa(G)$ is $\kappa$-directed. $\blacksquare$ \\

We will give an analogous argument for the models of a sketch. The hard part will obtaining the bound on the size of ``generated submodels'' like \eqref{eq:skolembound}. The rather explicit construction of the generated subgroup is a special case of the classical Skolem hull-construction in model theory, which builds up a model by inductively including all witnesses for existential formulae. This leads to a formulation of the (infinitary) downward Löwenheim-Skolem theorem for models of sketches. \\

During this section, we fix notation for sketches as $\mathbb S = (\cat S, \dist L, \dist C)$ and let $S=\obj(\cat S)$ be a set of sorts. Some futher terminology:

\begin{Definition}
We will consider models $M : \cat S \to \catname{Set}$ of $\mathbb S$ as $S$-sorted sets, endowed with unary operations. We write $s_M$ for the set $M(s)$ as well as $f_M : s_M \to s'_M$ for the operation $M(f)$ if $f : s \to s'$ is a morphism in $\cat S$.
\begin{enumerate}
\item The cardinality of an $S$-sorted set $A$ is \[ |A| = \sum_{s \in S} |s_A|. \]
\item Subsets and submodels $B \subseteq A$ and other set-theoretic notions like unions will be defined sort-by-sort.
\item We call a (many-sorted) set $A$ $\lambda$-small if $|A|<\lambda$.
\end{enumerate}
\end{Definition}

\begin{Theorem}[Downward Löwenheim-Skolem for Sketches]\label{thm:ls}
Let $\mathbb S$ be a sketch. Then there is a regular cardinal $\kappa > \mu_{\mathbb S}$, such that for every model $M \in \mod(\mathbb S)$ the following holds: Every subset $A \subseteq M$ of cardinality less than $\kappa$ is contained in a submodel $\bar A \subseteq M$ of cardinality less than $\kappa$.
\end{Theorem}
We will prove the theorem in the next subsection and describe the cardinal $\kappa$ explicitly.

\begin{Theorem}
$\mod(\mathbb S)$ is $\kappa$-accessible for every sketch $\mathbb S$.
\end{Theorem}
\begin{Proof}
Let $\kappa > \mu_\mathbb S$ be the cardinal from \ref{thm:ls}.
\begin{enumerate}
\item Recall from \ref{prop:modsdirectedcolimits} that $\mod(\mathbb S)$ is closed under $\kappa$-directed colimits, computed sort-by-sort.
\item Every $\kappa$-small model is $\kappa$-presentable. We show that this is true in general for $\fun{\cat S}{\catname{Set}}$, so the same has to hold in the full subcategory $\mod(\mathbb S)$ that computes the same relevant colimits: \\

Let $F : \cat S \to \catname{Set}$ be $\kappa$-small. We know that $F$ is the canonical colimit of representable functors, which are finitely presentable objects by \ref{ex:representablepresentable}. If we can show the bound $|\el(F)| < \kappa$ for the canonical diagram, $F$ will be $\kappa$-presentable by \ref{prop:smallcolim}. Here $|\el(F)|$ denotes the cardinality of the set of morphisms of $\el(F)$. \\

The morphisms of the canonical diagram $\el(F)$ all have the form $(s,x) \xrightarrow{f} (t,y)$ where
\[ x \in F(s),\, y \in F(t),\, h : t \to s,\, x, y \in S. \]
The number of such morphisms is bounded by
\[ |\el(F)| \leq |F|^2 \cdot |S|^3 = |F|\cdot |S| < \kappa. \]

\item For every model $M$, write $\sub_\kappa(M)$ for the poset of its $\kappa$-small submodels ordered by inclusion. By the Löwenheim-Skolem theorem, this poset is $\kappa$-directed, because if $\{A_i\}$ is a $\kappa$-small family of such submodels, the set
\[ A = \bigcup_i A_i \]
has cardinality smaller than $\kappa$, so $A$ is contained in a submodel from $\sub_\kappa(M)$.

\item $M$ is the $\kappa$-directed colimit (union) of $\sub_\kappa(M)$, as this colimit is computed set-theoretically and again by Löwenheim-Skolem, every singleton from the underlying set of $M$ is contained in a submodel from $\sub_\kappa(M)$. 

\item The class of all $\mathbb S$-models of cardinality less than $\kappa$ is essentially small. Their representatives form a generating set of $\mod(\mathbb S)$ in the definition of $\kappa$-accessibility.
\end{enumerate}
\end{Proof}

\begin{Corollary}
Locally presentable categories are precisely the model categories of limit sketches.
\end{Corollary}
\begin{Proof}
It remains to show that $\mod(\mathbb S)$ is locally presentable for limit sketches $\mathbb S$. But $\mod(\mathbb S)$ is closed under limits in $\fun{\cat A}{\catname{Set}}$ as limits are computed pointwise and limits commute with limits. Therefore it is complete and accessible, thus locally presentable.
\end{Proof}

\subsection{The Löwenheim-Skolem theorem}

I will motivate the cardinal bounds that occur in the infinitary version of Löwenheim-Skolem and then adapt the theorem to models of sketches. The categorical approach to this theorem is due to \cite[§3.3]{MakkaiPare}. \\

Consider a $\lambda$-ary algebra $M$ with $\mu$ operations and a subset $A_0 \subseteq M$. Close $A_0$ under the operations of the algebra by the following transfinite process:

\[
\begin{cases}
A_{i+1} = A_i^+, \\
A_{\alpha} = \cup_{i < \alpha}\,A_i
\end{cases}
\]

where

\[ A^+ = \{ f(\vec a) : \vec a \in A^\nu \, | \, f \text{ operation of arity } \nu \}. \] 

For convenience, we may assume that $M$ has an identity map among its operations, so $A \subseteq A^+$. The process stabilises at $A_\lambda$, for if $\vec a \in (A_\lambda)^\nu, \nu < \lambda$, each component $a_i$ lies in some $A_{\alpha_i}$. By regularity of $\lambda$, the supremum $\alpha$ of the $\alpha_i$ is $< \lambda$, thus already $\vec a \in (A_\alpha)^\nu$ and $f(\vec a) \in A_{\alpha+1} \subseteq A_\lambda$ by construction. $A_\lambda$ is the smallest subalgebra containing $A$. \\

Because all operations of the algebra have arity $< \lambda$, the set of possible inputs $\vec a$ ranges over the set
\[ A^{< \lambda} := \bigsqcup_{\nu < \lambda} A^\nu. \]

We use the following notation for the cardinality of that set
\begin{Definition}
For a cardinal $\mu$, let
\[ \mu^{< \lambda} := \sum_{\nu < \lambda} \mu^\nu. \]
\end{Definition}

The cardinality of $A^+$ is thus bounded by
\[ |A^+| \leq |\{ (f,\vec a) : \vec a \in A^{< \lambda} \}| = \mu \cdot |A|^{< \lambda}. \]

Just as a finite sequence of finite sequences can be compressed into a single finite sequence, repeated steps can be bounded by the following lemma from \cite[\nopp~2.10]{AdamekRosicky}:

\begin{Lemma}
For $\mu \geq \lambda$ and $\lambda$ regular, 
\[ \left(\mu^{< \lambda}\right)^{< \lambda} = \mu^{< \lambda}. \]
\end{Lemma}
\begin{Proof}
We have $(\mu^{< \lambda})^\alpha = \mu^{< \lambda}$ for all $\alpha < \lambda$. A function $\alpha \to \mu^{< \lambda}$ is equivalent to giving a function $c : \alpha \to \mathfrak L$, where $\mathfrak L$ is the set of cardinals $< \lambda$, and for each $i< \alpha$ an element of $\mu^{c(i)}$. Therefore 
\begin{align*}
(\mu^{< \lambda})^\alpha = \sum_{c : \alpha \to \mathfrak L} \prod_{i < \alpha} \mu^{c(i)} 
= \sum_{c : \alpha \to \mathfrak L} \mu^{\sum_{i < \alpha} c(i)}
\end{align*}
Now $|\mathfrak L^\alpha| \leq \lambda^\alpha \leq \mu^{< \lambda}$ and by regularity, $\sum_{i < \alpha} c(i) < \lambda$, thus the whole sum is bounded by $\mu^{< \lambda}$. Therefore we get
\[ (\mu^{< \lambda})^{< \lambda} = \sum_{\alpha < \lambda} (\mu^{< \lambda})^\alpha \leq \mu^{< \lambda}. \]
\end{Proof}

As a corollary, we can compute the bound \eqref{eq:skolembound} $\lambda$-ary algebras: Let $\beta \geq \mu + \lambda$ and say $|A_0| \leq \beta^{< \lambda}$, then we can show inductively that $|A_i| \leq \beta^{< \lambda}$ for all $i \leq \lambda$.
\begin{description}
\item[Successor step] If $|A_i| \leq \beta^{< \lambda}$, then
\[ |A_i^+| \leq \mu \cdot |A_i|^{< \lambda} \leq \left(\beta^{< \lambda}\right)^{< \lambda} = \beta^{< \lambda}. \]
\item[Limit step] If $|A_i| \leq \beta^{< \lambda}$ for each $i < \alpha \leq \lambda$ then
\[ |A_\alpha| \leq \sum_{i < \alpha} |A_i| \leq \beta^{< \lambda} \]
by regularity.
\end{description}

As noted in \cite[\nopp~3.3.3]{MakkaiPare}, we can use the same closure process for models of a sum-sketch. This is because by \ref{prop:sumconnectedlimits}, intersections of submodels are still submodels, therefore the description of the ``generated submodel'' stays the same as for varieties.
\begin{Proposition}
Let $\mathbb S$ be a sum sketch. Take the regular cardinals $\lambda = \lambda^+_\mathbb S, \mu = (\lambda + \mu_\mathbb S)^+$ and $\kappa = (\mu^{< \lambda})^+$. Then every subset $A$ of cardinality $< \kappa$ of a model of $\mathbb S$ is contained in a submodel of cardinality $< \kappa$.
\end{Proposition}
\begin{Proof}
Let $M$ be a model of $\mathbb S$ and $A \subseteq M$ a subset. A is a submodel of $M$ if and only if the following three closure conditions hold:
\begin{enumerate}
\item\label{item:f} For every $f : s \to t$ and $a \in s_A$, $f_M(a) \in t_A$
\item\label{item:cones} For every distinguished cone $(s \xrightarrow{\pi_i} s_i)$, we have $a \in s_A$ if and only if $(\pi_i)_M(a) \in (s_i)_A$ for all $i$.
\item\label{item:cocones} For every distinguished cocone $(s_i \xrightarrow{c_i} s)$, we have $a \in (s_i)_A$ if and only if $(c_i)_M(a) \in s_A$ for some $i$.
\end{enumerate}
This leads to the following closure process: We define $A^+$ as the union of the following elements
\begin{enumerate}
\item $f_M(a)$ where $f : s \to t$, $a \in A$
\item $a \in s_M$ such that $(\pi_i)_M(a) = a_i$ where $(a_i \in (s_i)_A)$ is a compatible collection with respect to a distinguished cone $(s \xrightarrow{\pi_i} s_i)$
\item $a \in (s_i)_M$ where $b \in s_A, (c_i)_M(a) = b$ for a distinguished cocone  $(s_i \xrightarrow{c_i} s)$
\end{enumerate}
This process stabilises for $A_\lambda$ with the smallest submodel containing $A$. We show again by induction that $|A_i| < \kappa$ for all $i \leq \lambda$: 
\begin{description}
\item[Successor step] Let $|A_i|$ be $< \kappa$. The contributions of \ref{item:f} and \ref{item:cocones} are easily bounded $< \kappa$. The contribution of \ref{item:cones} is at most $|\dist L|\cdot |A_i|^{< \lambda}$. But $|A_i| \leq \mu^{< \lambda}$ by assumption, so \[ |A_i|^{< \lambda} \leq \left(\mu^{< \lambda}\right)^{< \lambda} = \mu^{< \lambda} < \kappa. \]
\item[Limit step] follows by regularity of $\kappa$.
\end{description}
\end{Proof}

The final step of the proof of Löwenheim-Skolem is to generalise the result for sum sketches to arbitrary sketches, as done in \cite[55]{MakkaiPare} \\

Every colimit specification in a sketch can be formulated using only coproducts and coequalisers. By a technique called Morleyization\footnote{see \url{https://ncatlab.org/nlab/show/coherent+logic}}, we can even restrict the latter to \emph{effective coequalisers} (\emph{regular epi specifications}). Those are distinguished coequaliser cocones of the form
\[
\xymatrix{
a \ar@/^/[r]^{f_1} \ar@/_/[r]_{f_2} & b \ar[r]^{g} & c
}\]
where at the same time $(f_1,f_2)$ is specified as a kernel pair (pullback) cone of $c$. Furthermore, the construction does not change the cardinals $\lambda_\mathbb S, \mu_\mathbb S$ (see \cite[\nopp~3.2.4, \nopp~3.2.7]{MakkaiPare}). \\

We want to show that we can drop the effective coequaliser specifications and use our result for the remaining sum sketch. The idea is that in $\catname{Set}$, given a map and its kernel pair, stating that it is an effective epimorphism (coequaliser of the kernel pair) just requires stating that is surjective. Furthermore in $\catname{Set}$, every epimorphism splits. We create a new sketch $\mathbb S'$ by adding to every effective coequaliser specification a splitting morphism
\[
\xymatrix{
a \ar@/^/[r]^{f_1} \ar@/_/[r]_{f_2} & b \ar[r]^{g} & c \ar@/^/[l]^{r}
}\]
together with the condition that $g \cdot r = \id_c$. Let $\mathbb S''$ be the sketch $\mathbb S'$ with all effective coequaliser specifications removed. \\

Given a model $M \in \mod(\mathbb S)$ and $A \subseteq M$, we obtain a model $M'$ of $\mathbb S'$ by choosing appropriate splittings. As described above, we can identify the models of $\mathbb S'$ and $\mathbb S''$ as the kernel pairs are enough to determine the coequalisers given surjectivity. However $\mathbb S''$ is a sum sketch, so we can apply the closure process to $A$. The generated submodel will be a submodel of $M$ by forgetting the splittings. \pagebreak
\section{Conclusion}
\label{sec:conclusion}

% Acknowledge: AK, Ohad Kammar \pagebreak

% Acknowledgement
\section*{Acknowledgement}
I'd like to thank Andreas Kostakiotis, Ohad Kammar and Sean Moss for the interesting discussions that sometimes follow from innocent questions (``is there an algebraic theory that has precisely three models up to isomorphism?'').

% Print bibliography

% Print all of it
\nocite{*}

Add ncatlab

\addcontentsline{toc}{section}{Bibliography}
\printbibliography[title=Bibliography]

\end{document}
